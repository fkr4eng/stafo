\section{Sets}

% snippet(1)
A {\it set} $\{x,y,\ldots\}$ is a collection of elements.
A set can include either a finite or infinite number of elements.
The set $\SX$ is {\it finite} if it has a finite number of elements; otherwise, $\SX$ is {\it infinite}.
%
% snippet(2)
The set $\SX$ is {\it countably infinite} if $\SX$ is infinite and its elements are in one-to-one correspondence with the positive integers. The set $\SX$ is {\it countable} if it is either finite or countably infinite.

% snippet(3)
Let $\SX$ be a set.
Then, \begin{align}x\in \SX\end{align} means that $x$ is an {\it element}
\label{insym}%
\index{element!definition}%
of $\SX$. If $w$ is not an element of $\SX$, then we write
\begin{align}w\not\in \SX.\end{align}
\label{notinsym}

% snippet(3)
No set can be an element of itself.  Therefore, there does not exist a set that includes every set.  The set with no elements, denoted by $\varnothing\mspace{-1mu},$ is the {\it empty set}.
%
\label{varnothingsym}%
\index{empty set!definition}%
\index{nonempty set!definition}%
If $\SX\not=\varnothing\mspace{-1mu},$ then $\SX$ is {\it nonempty}.



Let $\SX$ and $\SY$ be sets. The {\it intersection}
\index{intersection!definition}%
of $\SX$ and $\SY$ is the set of common elements of $\SX$ and
$\SY$, which is given by
\begin{align}
\SX\cap \SY
%
\isdef\{x\mspace{-1mu}\colon\ x\in \SX \mbox{ and } x\in \SY \}
%
=  \{ x\in \SX \mspace{-2mu}\colon\ x\in \SY \}
%
=\{x\in \SY\mspace{-2mu}\colon\ x\in \SX \}
%
= \SY \cap \SX,\end{align}
