
\chapter[Beobachterentwurf mittels  Linearisierung der Fehlerdynamik]{Beobachterentwurf mittels exakter bzw. näherungsweiser Linearisierung
der Fehlerdynamik\label{cha:Beobachter-Normalform}}

Normalformen spielen beim Entwurf nichtlinearer Beobachter eine große
Rolle. Kann man ein System in eine Form überführen, bei der die Nichtlinearitäten
ausschließlich von den Messgrößen abhängen, ist der Entwurf eines
Beobachters mit exakt linearer Fehlerdynamik vergleichsweise einfach.
Eine solche Form ist die Beobachternormalform. Die Transformation
in diese Form und damit verbunden die Linearisierung des Beobachtungsfehlers
ist aufgrund restriktiver Existenzbedingungen bzw. einer aufwendigen
Berechnung in der regelungstechnischen Praxis kaum anzutreffen. Bei
genauerer Betrachtung eröffnen sich jedoch etliche Möglichkeiten zur
Berechnung bzw. zur Approximation eines Beobachters mit linearer Fehlerdynamik.
In diesem Kapitel werden Existenzbedingungen und Berechnungsmethoden
vorgestellt.\footnote{Das Kapitel basiert sich an dem Übersichtsbeitrag Röbenack, K.: Entwurf
nichtlinear Beobachter mit linearer und näherungsweise linearer Fehlerdynamik,
Automatisierungstechnik, De Gruyter-Verlag, Berlin, Boston, Jahrgang~58,
Heft~9, S.~489-497, für den der De~Gruyter-Verlag freundlicherweise
eine Nutzungsgenehmigung erteilt hat.}

\section{Linearisierung des Beobachtungsfehlers durch Aufschaltung\label{sec:Linearisierung-durch-Aufschaltung}}

In diesem Abschnitt soll zunächst die grundsätzliche Idee der exakten
Linearisierung des Beobachtungsfehlers durch Aufschaltung illustriert
werden. Angenommen, das zu beobachtende System kann in der Form
\begin{equation}
\dot{x}=Ax+\alpha(y,u),\quad y=c^{T}x\label{eq:lin-beob-fehler-strecke}
\end{equation}
mit dem Zustand~$x$, dem Eingang~$u$, und dem gemessenen Ausgang~$y$
dargestellt werden. Dabei sei der durch das Paar $(A,c^{T})$ beschriebene
lineare Teil beobachtbar. Zusätzlich seien alle auftretenden Nichtlinearitäten
in Abhängigkeit vom Ausgang darstellbar. Diese Nichtlinearitäten werden
in der Eingangs-Ausgangs-Aufschaltung~$\alpha$ zusammengefasst.
Setzt man den Beobachter in der Form
\begin{equation}
\dot{\hat{x}}=A\hat{x}+\alpha(y,u)+l\cdot\left(y-c^{T}\hat{x}\right)\label{eq:lin-beob-fehler-beobachter}
\end{equation}
mit einer Beobachterverstärkung $l\in\R^{n}$ an. so genügt der Beobachtungsfehler
$\tilde{x}=x-\hat{x}$ der linearen zeitinvarianten Differentialgleichung
\begin{equation}
\begin{array}{lcl}
\dot{\tilde{x}} & = & \dot{x}-\dot{\hat{x}}\\
 & = & Ax+\alpha(y,u)-A\hat{x}-\alpha(y,u)-l\cdot\left(y-c^{T}\hat{x}\right)\\
 & = & \left(A-lc^{T}\right)\tilde{x}.
\end{array}\label{eq:lin-beob-fehler-dynamik}
\end{equation}
Über die Verstärkung~$l$ kann man für die Systemmatrix $A-lc^{T}$
der Fehlerdynamik~(\ref{eq:lin-beob-fehler-dynamik}) ein beliebiges
charakteristisches Polynom

\begin{equation}
\det(sI-A+lc^{T})=a_{0}+a_{1}s+\cdots+a_{n-1}s^{n-1}+s^{n}\label{eq:cp}
\end{equation}
vorgeben. Diese Herangehensweise soll an zwei Beispielen vorgestellt
werden:

\begin{example}
\label{exa:inverses-pendel-motor-nf-beobachter}Wir betrachten das
Modell des inversen Pendels mit Gleich\-strom\-antrieb aus Beispiel~\ref{exa:inverses-pendel-gleichstrommotor}:

\begin{equation}
\begin{array}{lcl}
\dot{x}_{1} & = & x_{2}\\
\dot{x}_{2} & = & \frac{mg\ell}{J}\sin x_{1}-\frac{d}{J}x_{2}+\frac{K}{J}x_{3}\\
\dot{x}_{3} & = & -\frac{K}{L}x_{2}-\frac{R}{L}x_{3}+\frac{1}{L}u\\
y & = & x_{1}.
\end{array}\label{eq:inverses-pendel-motor-nf-strecke}
\end{equation}
Als Messgröße steht der Winkel~$x_{1}$, der beispielsweise mit einem
Inkrementgeber erfasst werden kann, zur Verfügung. Die einzige im
System auftretende Nichtlinearität hängt ausschließlich vom Ausgang~$x_{1}$
ab. Der Beobachter setzt sich dann aus einer Kopie des Systems, wobei
die Nichtlinearität vom gemessenen Ausgang gespeist wird, und einem
linearen Korrekturterm zusammen: 
\begin{equation}
\left(\begin{array}{l}
\dot{\hat{x}}_{1}\\
\dot{\hat{x}}_{2}\\
\dot{\hat{x}}_{3}
\end{array}\right)=\left(\begin{array}{c}
\hat{x}_{2}\\
\frac{mg\ell}{J}\sin y-\frac{d}{J}\hat{x}_{2}+\frac{K}{J}\hat{x}_{3}\\
-\frac{K}{L}\hat{x}_{2}-\frac{R}{L}\hat{x}_{3}+\frac{1}{L}u
\end{array}\right)+\left(\begin{array}{l}
l_{1}\\
l_{2}\\
l_{3}
\end{array}\right)\cdot\left(y-\hat{x}_{1}\right)\label{eq:inverses-pendel-motor-nf-beobachter}
\end{equation}
Der Vergleich von~(\ref{eq:inverses-pendel-motor-nf-strecke}) und~(\ref{eq:inverses-pendel-motor-nf-beobachter})
führt auf eine lineare Fehlerdynamik~(\ref{eq:lin-beob-fehler-dynamik}).
Für ein vorgegebenes charakteristisches Polynom~(\ref{eq:cp}) mit
den Koeffizienten $a_{0},a_{1},a_{2}>0$ berechnet man durch Koeffizientenvergleich
oder mit Hilfe der Ackermann-Formel~\cite{acker77,ackermann77} die
Beobachterverstärkung 
\begin{equation}
\begin{array}{lcl}
l_{1} & = & -\frac{R}{L}-\frac{d}{J}+a_{2},\\
l_{2} & = & \frac{-JK^{2}L+\left(a_{1}J^{2}-a_{2}dJ+d^{2}\right)L^{2}-J\left(a_{2}J-d\right)LR+J^{2}R^{2}}{J^{2}L^{2}},\\
l_{3} & = & -\frac{\left(a_{2}J-d\right)K^{2}L^{2}-a_{0}J^{2}L^{3}+JL\left(a_{1}JL-2K^{2}\right)R-a_{2}J^{2}LR^{2}+J^{2}R^{3}}{JKL^{3}}.
\end{array}\label{eq:inverses-pendel-motor-nf-verstaerkung}
\end{equation}
\end{example}

\begin{example}
\label{exa:inverses-pendel-elastisch}Wir betrachten wiederum einen
Manipulator in der Form eines inversen Pendels (siehe Abb.~\ref{fig:Inverses-Pendel-elastisch}).
Der Antrieb erfolgt über ein eingeprägtes Drehmoment~$u$, welches
den Winkel~$q_{1}$ beeinflusst. Der Antrieb ist über eine elastische
Welle, die als Torsionsfeder mit der Federkonstanten~$k$ modelliert
wird, mit dem Manipulatorarm verbunden. Die Position des Manipulatorarms
wird durch den Relativwinkel~$q_{2}$ beschrieben. Der Manipulator
habe beim Antrieb das Trägheitsmoment~$J_{1}$ und hinsichtlich des
Manipulatorarms das Trägheitsmoment~$J_{2}$. Am Ende des Manipulatorarms
der Länge~$\ell$ befinde sich eine Last mit der Masse~$m$.

Das System hat die kinetische Energie
\[
T=\frac{J_{1}}{2}\dot{q}_{1}^{2}+\frac{J_{2}}{2}\left(\dot{q}_{1}+\dot{q}_{2}\right)^{2}
\]
und die potentielle Energie
\[
V=m\ell\left(\cos(q_{1}+q_{2})-1\right)+\frac{k}{2}q_{2}^{2}.
\]
Aus der zugehörigen Bewegungsgleichung erhält man mit $x=(q_{1},\dot{q}_{1},q_{2},\dot{q}_{2})^{T}$
das Zustandsraummodell
\begin{equation}
\left(\begin{array}{l}
\dot{x}_{1}\\
\dot{x}_{2}\\
\dot{x}_{3}\\
\dot{x}_{4}
\end{array}\right)=\left(\begin{array}{c}
x_{2}\\
\frac{k}{J_{1}}x_{3}+\frac{1}{J_{1}}u\\
x_{4}\\
\frac{mg\ell}{J_{2}}\sin(x_{1}+x_{3})-\left(\frac{1}{J_{1}}+\frac{1}{J_{2}}\right)kx_{3}-\frac{u}{J_{1}}
\end{array}\right).\label{eq:inverses-pendel-elastisch}
\end{equation}
Werden die Winkel~$x_{1}$ und~$x_{3}$ oder deren Summe $x_{1}+x_{3}$
gemessen, dann kann die in~(\ref{eq:inverses-pendel-elastisch})
auftretende Nichtlinearität $\sin(x_{1}+x_{3})$ beim Beobachterentwurf
vollständig kompensiert werden~\cite{nicosia88}.
\end{example}
\begin{figure}
\begin{centering}
\resizebox{0.4\textwidth}{!}{\input{inv_pendel_elastisch.pdftex_t}}
\par\end{centering}
\caption{Inverses Pendel, welches elatisch mit dem Antrieb gekoppelt ist\label{fig:Inverses-Pendel-elastisch}}
\end{figure}


\section{Normalfrom-Beobachter\label{sec:Normal-Form-Observer}}

\subsection{Beobachter-Struktur}

Die in Abschnitt~\ref{sec:Linearisierung-durch-Aufschaltung} beschriebene
Herangehensweise wird nachfolgend systematisiert. In der Regel wird
das zu untersuchende System nicht unmittelbar in der Form~(\ref{eq:lin-beob-fehler-strecke})
vorliegen. In etlichen Fällen ist aber eine derartige Darstellung
in anderen Koordinaten möglich. Dazu wird in diesem Abschnitt die
Beobachternormalform eingeführt.

Man betrachte ein nichtlineares System 
\begin{equation}
\dot{x}=F(x,u),\quad y=h(x).\label{eq:plant}
\end{equation}
Die Menge $\mathcal{M}\subseteq\R^{n}$ sei offen und die Menge $\mathcal{U}\subseteq\R^{m}$
beliebig. Die Abbildungen $F:\mathcal{M}\times\mathcal{U}\to\R^{n}$
und $h:\mathcal{M}\to\R$ seien ausreichend glatt. Damit ist~$F$
ein vom Eingang abhängiges Vektorfeld und~$h$ ein Skalarfeld. Angenommen,
in der Umgebung eines Punktes $p\in\mathcal{M}$ gibt es einen Diffeomorphismus
\begin{equation}
z=T(x)\quad\textrm{mit}\quad x=S(z),\label{eq:TS}
\end{equation}
der das System~(\ref{eq:plant}) in die Form 
\begin{equation}
\dot{z}=Az+\alpha(c^{T}z,u),\quad y=\gamma(c^{T}z)\label{eq:nf}
\end{equation}
überführt, wobei das Paar $(A,c^{T})$ in der \emph{dualen} \emph{Brunovský-Normalform}\index{Brunovský-Normalform!duale}\index{Normalform!duale Brunovský-}~\cite{brunovsky70}
vorliege: 
\begin{equation}
A=\left(\begin{array}{cccc}
0 & 0 & \cdots & 0\\
1 & 0 & \ddots & \vdots\\
 & \ddots & \ddots & 0\\
0 &  & 1 & 0
\end{array}\right),\quad c^{T}=\left(\begin{array}{cccc}
0 & \cdots & 0 & 1\end{array}\right).\label{eq:AC}
\end{equation}
Die Form~(\ref{eq:nf}) entspricht der \emph{Beobachter\-normal\-form}
(engl. \emph{observer canonical form})~\cite{zeitz85,zeitz89}\index{Beobachternormalform}\index{Normalform!Beobachter-}
mit der zusätzlichen Ausgangstransformation~$\gamma$ (siehe Abb.~\ref{fig:Beobachter-Normalform}).
Die Abbildung~$\gamma$ muss dabei invertierbar sein. Die vektorwertige
Abbildung~$\alpha$ berücksichtigt die im System auftretenden Nichtlinearitäten.
Dabei fungiert $\alpha(\gamma^{-1}(y),u)$ als Eingangs-Ausgangs-Aufschaltung. 

\begin{figure}
\begin{centering}
\resizebox{0.7\textwidth}{!}{\input{bild-nf2.pdftex_t}}
\par\end{centering}
\caption{Beobachternormalform~(\ref{eq:nf}), Quelle:~\cite{roebenack2010at}\label{fig:Beobachter-Normalform}}
\end{figure}

Für ein System der Form~(\ref{eq:nf}) ist der Beobachterentwurf
sehr einfach. Ein Beobachter der Form 
\begin{equation}
\begin{array}{ccl}
\dot{\hat{z}} & = & A\hat{z}+\alpha(\gamma^{-1}(y),u)+l(\gamma^{-1}(y)-c^{T}\hat{z})\\
\hat{x} & = & S(\hat{z})
\end{array}\label{eq:obs-in-nf}
\end{equation}
mit dem Vektor $l\in\R^{n}$ (siehe Abb.~\ref{fig:Normalform-Beobachter})
führt auf einen Beobachtungsfehler $\tilde{z}=z-\hat{z}$, welcher
dem linearen zeit\-invarianten Differential\-gleichungs\-system
\begin{equation}
\dot{\tilde{z}}=(A-lc^{T})\tilde{z}\label{eq:error-dgl-nf}
\end{equation}
genügt. Da das Paar $(A,c^{T})$ beobachtbar ist, lassen sich die
Eigenwerte von $A-lc^{T}$ beliebig zuweisen. Für ein gewünschtes
charakteristisches Polynom~(\ref{eq:cp}) der Fehlerdynamik~(\ref{eq:error-dgl-nf})
ist der Vektor~$l$ entsprechend $l=(a_{0},\ldots,a_{n-1})^{T}$
zu wählen. Legt man~$l$ derart fest, dass alle Wurzeln von~(\ref{eq:cp})
in der offenen linken Halbebene liegen, dann ist die Ruhelage $\tilde{z}=0$
von~(\ref{eq:error-dgl-nf}) exponentiell (und damit auch asymptotisch)
stabil. 

\begin{figure}
\begin{centering}
\resizebox{0.85\textwidth}{!}{\input{bild-nfbeob1.pdftex_t}}
\par\end{centering}
\caption{Normalform-Beobachter~(\ref{eq:obs-in-nf})\label{fig:Normalform-Beobachter}}
\end{figure}

Der Beobachter~(\ref{eq:obs-in-nf}) basiert unmittelbar auf der
Normalform~(\ref{eq:nf}). Daher nennt man ihn \emph{Normalform-Beobachter}\index{Beobachter!Normalform-}.\index{Normalform-Beobachter}
Mitunter ist es wünschenswert, den Beobachter~(\ref{eq:obs-in-nf})
direkt in den Originalkoordinaten zu implementieren:
\begin{equation}
\dot{\hat{x}}=F(\hat{x},u)+k_{\infty}(\hat{x},y,u).\label{eq:ons-orig}
\end{equation}
Der Korrekturterm hat dann die Form 
\begin{equation}
\begin{array}{ccl}
k_{\infty}(\hat{x},y,u) & = & \left(T^{\prime}(\hat{x})\right)^{-1}\left[\alpha(\gamma^{-1}(y),u)-\alpha(\gamma^{-1}(h(\hat{x})),u)\right.\\
 &  & \left.+\,l(\gamma^{-1}(y)-\gamma^{-1}(h(\hat{x})))\right].
\end{array}\label{eq:correction-term-nf-obs}
\end{equation}

\begin{remark}
Mit dem beschriebenen Zugang ergibt sich eine Fehlerdynamik, die in
den transformierten Koordinaten einer linearen zeit\-invarianten
Differentialgleichung~(\ref{eq:error-dgl-nf}) genügt. Daher spricht
man auch von einer \emph{exakten Linearisierung des Beobachtungsfehlers}.
Dieses Konzept ist auch bei Systemen mit mehreren Ausgängen anwendbar~\cite{krener85observer,xia88,xia89}
und kann mit einer Parameter\-adaption kombiniert werden~\cite{marino1990,marino1992,marino1995,marino1995book}.
Die Linearisierung des Beobachtungsfehlers entspricht in gewisser
Weise der vollständigen Linearisierung (Eingangs-Zustands-Linearisierung)
beim Reglerentwurf. Ähnlich wie bei der Regelung ist auch beim Beobachterentwurf
mitunter nur eine \emph{partielle Linearisierung des Beobachtungsfehlers}
möglich. Entsprechende Entwurfsverfahren werden beispielsweise in~\cite{jo2000b,jo2002,roebenack2005syncod,roebenack2006amcs,roebenack2007iet,roebenack2007ndst}
beschrieben.
\end{remark}

\begin{remark}
Mitunter wird für das System~(\ref{eq:nf}) mit dem Beobachter~(\ref{eq:obs-in-nf})
eine andere Darstellungsform favorisiert. Seien 
\[
\check{A}:=A-lc^{T}\quad\mbox{und}\quad\check{\alpha}(z_{n},u):=\alpha(z_{n},u)+lz_{n},
\]
wobei der Vektor $l\in\R^{n}$ so zu wählen ist, dass die Matrix~$\check{A}$
nur Eigenwerte in der linken offenen Halbebene besitzt. Das System~(\ref{eq:nf})
lässt sich dann in der Form 
\[
\begin{array}{ccl}
\dot{z} & = & Az+\alpha(z_{n},u)\\
 & = & Az-lc^{T}z+lc^{T}z+\alpha(z_{n},u)\\
 & = & \check{A}z+\check{\alpha}(\gamma^{-1}(y),u)
\end{array}
\]
angeben. Der Beobachter~(\ref{eq:obs-in-nf}) vereinfacht sich zu
\[
\begin{array}{ccl}
\dot{\hat{z}} & = & A\hat{z}+\alpha(z_{n},u)+l(z_{n}-c^{T}\hat{z})\\
 & = & (A-lc^{T})\hat{z}+\alpha(z_{n},u)+lz_{n}\\
 & = & \check{A}\hat{z}+\check{\alpha}(\gamma^{-1}(y),u)
\end{array}
\]
mit $\hat{x}=S(\hat{z})$. Als Fehlerdifferentialgleichung erhält
man wiederum Gl.~(\ref{eq:error-dgl-nf}), d.\,h. $\dot{\tilde{z}}=\check{A}\tilde{z}$.
Diese Form von Strecke und Beobachter ist bei verschiedenen neueren
Zugängen zur exakten bzw. approximativen Linearisierung des Beobachtungsfehlers
zu finden (siehe~\cite{kazantzis1998,krener2002,xiao2006,deutscher2006,deutscher2007}
sowie Anmerkung~\ref{rem:Boebachter-mit-Hilfssatz-von-Ljapunov}).
\end{remark}

\begin{remark}
\label{rem:endliche-Einstellzeit}Bei der Fehlerdifferentialgleichung~(\ref{eq:error-dgl-nf})
des Normalformbeobachters~ (\ref{eq:obs-in-nf}) kann man durch geeignete
Wahl des Vektors $l\in\R^{n}$ die Eigenwerte der Systemmatrix $A-lc^{T}$
in der komplexen Ebene beliebig weit links anordnen und damit die
Konvergenzgeschwindigkeit vorgeben. Bei einem Anfangswert $\tilde{z}(0)\neq0$
wird die Ruhelage $\tilde{z}=0$ von~(\ref{eq:error-dgl-nf}) jedoch
erst in unendlicher Zeit erreicht. Für lineare Systeme kann man durch
geeignete Nutzung eines Totzeitgliedes eine endliche Einstellzeit
vorgeben~\cite{engel2002}. Dieser Zugang lässt sich auch für nichtlineare
Systeme, die in die Beobachternormalform~(\ref{eq:nf}) transformierbar
sind, erweitern~\cite{menold2003cdc}.
\end{remark}

\subsection{Reduzierter Normalform-Beobachter}

Bei der Normalform~(\ref{eq:nf}) lässt sich die letzte Komponente
des Zustands mit $z_{n}=\gamma^{-1}(y)$ direkt aus dem Ausgang rekonstruieren
und muss daher nicht geschätzt werden. Folglich ist es naheliegend,
zur Rekonstruktion der verbleibenden $n-1$ Zustandsgrößen $z_{1},\ldots,z_{n-1}$
einen \emph{reduzierten} Beobachter\index{Beobachter!reduzierter}
einzusetzen. 

Die Herleitung eines reduzierten Zustandsbeobachters ist in Analogie
zum linearen Fall möglich~\cite[Abschnitt~{7.4.2}]{ludyk1995-2}.
Durch Streichen der letzten Differentialgleichung in der Normalform~(\ref{eq:nf})
erhält man das System 
\begin{equation}
\dot{\bar{z}}=\bar{A}\bar{z}+\bar{\alpha}(\bar{c}^{T}z,u)\label{eq:reduzierte-nf}
\end{equation}
mit 
\[
\bar{z}=\left(\begin{array}{c}
z_{1}\\
\vdots\\
z_{n-1}
\end{array}\right)\quad\mbox{und}\quad\bar{\alpha}(z_{n},u)=\left(\begin{array}{c}
\alpha_{1}(z_{n},u)\\
\vdots\\
\alpha_{n-1}(z_{n},u)
\end{array}\right).
\]
Das Paar $(\bar{A},\bar{c}^{T})$ mit $\bar{A}\in\R^{(n-1)\times(n-1)}$
und $\bar{c}\in\R^{n-1}$ liegt in der dualen Brunovský-Normalform
der Ordnung $n-1$ vor (siehe Gl.~(\ref{eq:AC})). Der Beobachtungsfehler
$\tilde{\bar{z}}=\bar{z}-\hat{\bar{z}}$ soll einer linearen zeit\-invarianten
Differentialgleichung 
\begin{equation}
\dot{\tilde{\bar{z}}}=\left(\bar{A}-\bar{l}\,\bar{c}^{T}\right)\tilde{\bar{z}}\label{eq:reduziert-error-dgl}
\end{equation}
genügen, wobei der Vektor $\bar{l}=(p_{0},\ldots,p_{n-2})^{T}\in\R^{n-1}$
die Koeffizienten eines vorgegebenen charakteristischen Polynoms vom
Grad $n-1$ enthält. Der Vergleich von~(\ref{eq:reduzierte-nf})
und~(\ref{eq:reduziert-error-dgl}) liefert für den Zustand $\hat{\bar{z}}=\bar{z}-\tilde{\bar{z}}$
die Differentialgleichung
\begin{equation}
\dot{\hat{\bar{z}}}=\bar{A}\hat{\bar{z}}+\bar{\alpha}(z_{n},u)+\bar{l}\tilde{\bar{z}}_{n-1}\label{eq:red-nf-beob-schritt1}
\end{equation}
mit $\tilde{\bar{z}}_{n-1}=\bar{z}_{n-1}-\hat{\bar{z}}_{n-1}$. In
dieser Gleichung wird $\bar{z}_{n-1}=z_{n-1}$ benötigt. Diese Größe
lässt sich aus der letzten Differentialgleichung 
\[
\dot{z}_{n}=z_{n-1}+\alpha_{n}(z_{n},u),\quad z_{n}=\gamma^{-1}(y)
\]
der Normalform~(\ref{eq:nf}) rekonstruieren: 
\[
\bar{z}_{n-1}=z_{n-1}=\frac{\d}{\d t}\left(\gamma^{-1}(y)\right)-\alpha_{n}(\gamma^{-1}(y),u).
\]
Durch Einsetzen in~(\ref{eq:red-nf-beob-schritt1}) erhält man 
\[
\begin{array}{ccl}
\dot{\hat{\bar{z}}} & = & \bar{A}\hat{\bar{z}}+\bar{\alpha}(\gamma^{-1}(y),u)+\bar{l}\left(\frac{\d}{\d t}\left(\gamma^{-1}(y)\right)-\alpha_{n}(\gamma^{-1}(y),u)-\bar{c}^{T}\hat{\bar{z}}\right)\\
 & = & \left(\bar{A}-\bar{l}\,\bar{c}^{T}\right)\hat{\bar{z}}+\bar{\alpha}(\gamma^{-1}(y),u)+\bar{l}\left(\frac{1}{\gamma^{\prime}(y)}\dot{y}-\alpha_{n}(\gamma^{-1}(y),u)\right).
\end{array}
\]
Diese Gleichung kann wegen der auf der rechten Seite auftretenden
Zeit\-ab\-leitung~$\dot{y}$ noch nicht für die Zustandsbeobachtung
eingesetzt werden. Zur Elimination von~$\dot{y}$ wird für den Beobachter
die Koordinatentransformation 
\[
\begin{array}{ccccc}
w & = & \hat{\bar{z}} & - & \bar{l}\gamma^{-1}(y)\\
\hat{\bar{z}} & = & w & + & \bar{l}\gamma^{-1}(y)
\end{array}
\]
eingeführt. Dann gilt 
\[
\begin{array}{ccl}
\dot{w} & = & \frac{\d}{\d t}\left(\hat{\bar{z}}-\bar{l}\gamma^{-1}(y)\right)\\
 & = & \dot{\hat{\bar{z}}}-\bar{l}\frac{\d}{\d t}\left(\gamma^{-1}(y)\right)\\
 & = & \left(\bar{A}-\bar{l}\,\bar{c}^{T}\right)\hat{\bar{z}}+\bar{\alpha}(\gamma^{-1}(y),u)+\bar{l}\left(\frac{\d}{\d t}\left(\gamma^{-1}(y)\right)-\alpha_{n}(\gamma^{-1}(y),u)\right)-\bar{l}\frac{\d}{\d t}\left(\gamma^{-1}(y)\right)\\
 & = & \left(\bar{A}-\bar{l}\,\bar{c}^{T}\right)\hat{\bar{z}}+\bar{\alpha}(\gamma^{-1}(y),u)-\bar{l}\alpha_{n}(\gamma^{-1}(y),u)\\
 & = & \left(\bar{A}-\bar{l}\,\bar{c}^{T}\right)\left(w+\bar{l}\gamma^{-1}(y)\right)+\bar{\alpha}(\gamma^{-1}(y),u)-\bar{l}\alpha_{n}(\gamma^{-1}(y),u)\\
 & = & \left(\bar{A}-\bar{l}\,\bar{c}^{T}\right)w+\left(\bar{A}-\bar{l}\,\bar{c}^{T}\right)\bar{l}\gamma^{-1}(y)+\bar{\alpha}(\gamma^{-1}(y),u)-\bar{l}\alpha_{n}(\gamma^{-1}(y),u).
\end{array}
\]
Die letzte Gleichung liefert den \emph{reduzierten Normalform-Beobachter}\index{Normalform-Beobachter!reduzierter}
\begin{equation}
\dot{w}=\left(\bar{A}-\bar{l}\,\bar{c}^{T}\right)w+\underbrace{\left(\bar{A}-\bar{l}\,\bar{c}^{T}\right)\bar{l}\gamma^{-1}(y)+\bar{\alpha}(\gamma^{-1}(y),u)-\bar{l}\alpha_{n}(\gamma^{-1}(y),u)}_{{\displaystyle \kappa(\gamma^{-1}(y),u)}}\label{eq:red-nf-beob}
\end{equation}
mit dem Korrekturterm~$\kappa$, wobei der Originalzustand mittels
\begin{equation}
\hat{z}=\left(\begin{array}{c}
w+\bar{l}\gamma^{-1}(y)\\
\gamma^{-1}(y)
\end{array}\right),\quad\hat{x}=S(\hat{z})\label{eq:red-nf-ruecktransf}
\end{equation}
rekonstruiert wird. Das zugehörige Strukturbild ist in Abb.~\ref{fig:Reduzierter-Normalformbeobachter}
dargestellt.

\begin{figure}
\begin{centering}
\resizebox{0.85\textwidth}{!}{\input{bild-reduzierter-nf-beob1.pdftex_t}}
\par\end{centering}
\caption{Reduzierter Normalformbeobachter~(\ref{eq:red-nf-beob}) mit Rücktransformation~(\ref{eq:red-nf-ruecktransf})\label{fig:Reduzierter-Normalformbeobachter}}
\end{figure}


\section{Existenz und Berechnung der Beobachternormalform}

\subsection{Existenz und Berechnung der Normalform nach Byrnes und Isidori\label{subsec:Berechnung-nach-Byrnes-Isidori}}

Die Schwierigkeit beim Entwurf eines Normalform-Beobachters besteht
in der Existenz der Normalform~(\ref{eq:nf}) sowie in der konkreten
Berechnung der Transformation~(\ref{eq:TS}). Für die folgenden Betrachtungen
ist es zweckmäßig, das zu beobachtende System~(\ref{eq:plant}) in
der Form 
\begin{equation}
\dot{x}=f(x)+g(x,u),\quad y=h(x),\label{eq:plant-fg}
\end{equation}
mit den Abbildungen $f:\mathcal{M}\to\R^{n}$ und $g:\mathcal{M}\times\mathcal{U}\to\R^{n}$
anzugeben. Diese Abbildungen seien durch 
\[
f(x)=F(x,0)\quad\mbox{und}\quad g(x,u)=F(x,u)-F(x,0)
\]
festgelegt. Die Beobachtbarkeitsmatrix\index{Beobachtbarkeitsmatrix}~$Q_{B}$
ist die Jacobimatrix der Beobachtbarkeitsabbildung\index{Beobachtbarkeitsabbildung}~$q:\mathcal{M}\to\R^{n}$,
d.\,h. 
\begin{equation}
q(x)=\left(\begin{array}{c}
h(x)\\
L_{f}h(x)\\
\vdots\\
L_{f}^{n-1}h(x)
\end{array}\right),\quad Q_{B}(x)=q^{\prime}(x)=\left(\begin{array}{c}
dh(x)\\
L_{f}dh(x)\\
\vdots\\
L_{f}^{n-1}dh(x)
\end{array}\right),\label{eq:Q}
\end{equation}
vgl. Abschnitt~\ref{subsec:Beobachtbarkeit-autonom}. Der sogenannte
\emph{Start\-vektor} bzw. das sog. \emph{Start\-vektor\-feld}\index{Startvektorfeld}\footnote{Mit dem Vektorfeld~$v$ wird zyklisch der gesamte Raum für das transformierte
System aufgespannt.} ist die Lösung des Gleichungssystems
\begin{equation}
Q_{B}(x)\,v(x)=\beta(h(x))\,\frac{\partial}{\partial x_{n}}\label{eq:v}
\end{equation}
mit einer skalarwertigen Funktion~$\beta$. Der folgende Satz gibt
die Existenzbedingungen für die Normalform~(\ref{eq:nf}) an~\cite{krener83,marino1990}:
\begin{theorem}
\label{thm:existence}In einer Umgebung von $p\in\mathcal{M}$ existiert
genau dann ein lokaler Diffeomorphismus~(\ref{eq:TS}), der das System~(\ref{eq:plant-fg})
in die Normalform~(\ref{eq:nf}) mit einem lokalen Diffeomorphismus~$\gamma$
überführt, wenn für eine geeignete Funktion~$\beta$ mit $\beta(h(p))\neq0$
die Bedingungen
\begin{enumerate}
\item \label{enu:ki1}$\rank\,Q_{B}(p)=n$

\item \label{enu:ki2}$[\ad_{-f}^{i}v,\ad_{-f}^{j}v](x)=0$ für $0\leq i,j\leq n-1$
\item \label{enu:ki3}$[g,\ad_{-f}^{i}v](x,u)=0$ für $0\leq i\leq n-2$
\end{enumerate}
für alle $x$ in einer Umgebung von~$p$ und für alle $u\in\mathcal{U}$
erfüllt sind. Es gibt einen globalen Diffeomrophismus~(\ref{eq:TS}),
wenn die Bedingungen~\ref{enu:ki1}-\ref{enu:ki3} in $\mathcal{M}=\R^{n}$
gelten und zusätzlich
\begin{enumerate}[start=4]
\item \label{enu:ki4}$\ad_{-f}^{i}v$ für $0\leq i<n$ vollständige Vektorfelder\index{Vektorfeld!vollständiges}
sind.
\end{enumerate}
\end{theorem}
Der folgende Beweis widmet sich der Existenz einer lokalen Koordinatentransformation. 
\begin{svmultproof2}
\hinreichend\ Die Bedingungen \ref{enu:ki1}-\ref{enu:ki3} seien
in einer Umgebung von $p\in\mathcal{M}$ erfüllt. Wegen Bedingung~\ref{enu:ki1}
ist das lineare Gls.~(\ref{eq:v}) (eindeutig) lösbar bezüglich~$v$.
Zeilenweise kann man Gls.~(\ref{eq:v}) auch in der Form 
\begin{equation}
L_{v}L_{f}^{i}h(x)=\left\langle \d L_{f}^{i}h(x),v(x)\right\rangle =\left\{ \begin{array}{ccl}
0 & \mbox{für} & i=0,\ldots,n-2,\\
\beta(h(x)) & \mbox{für} & i=n-1.
\end{array}\right.\label{eq:proof-obsnf-gl0}
\end{equation}
schreiben. Wegen 
\begin{equation}
\begin{array}{l}
\left(\begin{array}{c}
\d h(p)\\
\vdots\\
\d L_{f}^{n-1}h(p)
\end{array}\right)\left(v(p),\,\ad_{-f}v(p),\,\cdots,\,\ad_{-f}^{n-1}v(p)\right)=\\
=\left(\begin{array}{cccc}
0 & \cdots & 0 & \beta(h(p))\\
\vdots & \qdots & \qdots & *\\
0 & \qdots & \qdots & \vdots\\
\beta(h(p)) & * & \cdots & *
\end{array}\right)
\end{array}\label{eq:proof-obsnf-gl1}
\end{equation}
sind die Vektorfelder $v,\ad_{-f}v,\ldots,\ad_{-f}^{n-1}v$ in einer
Umgebung von~$p$ linear unabhängig (vgl. Lemma~\ref{lem:Lin-Unabh-Kovektoren}).
In Verbindung mit Bedingung~\ref{enu:ki2} garantiert der Begradigungssatz
(Satz~\ref{thm:simultane-begradigung-von-VF}) die Existenz eines
lokalen Diffeomorphismus~(\ref{eq:TS}) mit 
\begin{equation}
T_{*}\ad_{-f}^{i}v=\frac{\partial}{\partial z_{i+1}}\quad\mbox{für}\quad i=0,\ldots,n-1.\label{eq:proof-obsnf-gl2}
\end{equation}
 Der Diffeomorphismus~(\ref{eq:TS}) transformiert die im System~(\ref{eq:plant-fg})
auftretenden Abbildungen in die Form 
\begin{equation}
\begin{array}{lcl}
\bar{f}(z) & = & T_{*}f(S(z)),\\
\bar{g}(z,u) & = & T_{*}g(S(z),u),\\
\bar{h}(z) & = & h(S(z)).
\end{array}\label{eq:proof-obsnf-bar-groessen}
\end{equation}
Wegen Gls.~(\ref{eq:proof-obsnf-gl1}) gilt
\begin{equation}
\frac{\partial}{\partial z_{i+1}}\bar{h}=\left\langle \d h,\ad_{-f}^{i}v\right\rangle =\left\{ \begin{array}{ccl}
0 & \mbox{für} & i=0,\ldots,n-2,\\
\beta(h(x)) & \mbox{für} & i=n-1,
\end{array}\right.\label{eq:proof-obsnf-dh-beta}
\end{equation}
d.\,h. der Gradient von~$\bar{h}$ hat die Form 
\[
\bar{h}^{\prime}(z)=\left(\begin{array}{cccc}
0 & \cdots & 0 & *\end{array}\right).
\]
Folglich hat die Ausgangsabbildung~$\bar{h}$ die in Gl.~(\ref{eq:nf})
angegebene Form d.\,h. es besteht nur eine Abhängigkeit von der letzten
Zustandskomponente~$z_{n}$.

Als nächstes wird das Vektorfeld 
\[
\bar{f}=\bar{f}_{1}\frac{\partial}{\partial z_{1}}+\cdots+\bar{f}_{n}\frac{\partial}{\partial z_{n}}
\]
betrachtet. Mit~(\ref{eq:proof-obsnf-gl2}) gilt 
\[
\begin{array}{ccl}
\tfrac{\partial}{\partial z_{i+1}} & = & T_{*}\ad_{-f}^{i}v\\
 & = & T_{*}[-f,\ad_{-f}^{i-1}v]\\
 & = & [-T_{*}f,T_{*}\ad_{-f}^{i-1}v]\\
 & = & [-\bar{f},\tfrac{\partial}{\partial z_{i}}]\\
 & = & \sum_{j=1}^{n}\left(\left(\tfrac{\partial}{\partial z_{i}}\bar{f}_{j}\right)\tfrac{\partial}{\partial z_{j}}\right)
\end{array}
\]
für $i=1,\ldots,n-1$. Ein Vergleich beider Seiten der Gleichung liefert
\[
\frac{\partial\bar{f}_{j}}{\partial z_{i}}=\left\{ \begin{array}{ccccc}
0 & \mbox{für} & j & \neq & i+1\\
1 & \mbox{für} & j & = & i+1
\end{array}\right.
\]
mit $1\leq i\leq n-1$ und $1\leq j\leq n$. Die $n\times n$-Jacobimatrix
von~$\bar{f}$ hat folglich die Form 
\begin{equation}
\bar{f}^{\prime}(z)=\left(\begin{array}{ccccc|c}
0 & 0 & \cdots & 0 & 0 & *\\
1 & 0 &  & \vdots & 0 & *\\
0 & 1 & \ddots & \vdots & \vdots & \vdots\\
\vdots & 0 & \ddots & 0 & \vdots & \vdots\\
0 & \vdots & \ddots & 1 & 0 & *\\
0 & 0 & \cdots & 0 & 1 & *
\end{array}\right).\label{eq:proof-obsnf-Form-DF}
\end{equation}

Bei dem vom Eingang abhängigen Vektorfeld~$\bar{g}$ geht man ähnlich
vor. Wegen Bedingung~\ref{enu:ki3} und Gl.~(\ref{eq:proof-obsnf-gl2})
gilt 
\[
\begin{array}{ccl}
0 & = & T_{*}[g,\ad_{-f}^{i}v]\\
 & = & [T_{*}g,T_{*}\ad_{-f}^{i}v]\\
 & = & [\bar{g},\frac{\partial}{\partial z_{i+1}}]\\
 & = & -\sum_{j=1}^{n}\left(\left(\tfrac{\partial}{\partial z_{i+1}}\bar{g}_{j}\right)\tfrac{\partial}{\partial z_{j}}\right)
\end{array}
\]
für $i=0,\ldots,n-2$, d.\,h. 
\[
\frac{\partial}{\partial z_{i+1}}\bar{g}_{j}=0\quad\mbox{für}\quad1\leq j\leq n\quad\mbox{und}\quad0\leq i\leq n-2.
\]
Die Jacobimatrix von~$\bar{g}$ hat somit die Form 
\begin{equation}
\frac{\partial\bar{g}}{\partial z}(z,u)=\left(\begin{array}{ccc|c}
0 & \cdots & 0 & *\\
\vdots & \ddots & \vdots & \vdots\\
0 & \cdots & 0 & *
\end{array}\right).\label{eq:proof-obsnf-Form-DG}
\end{equation}
Aus Gln.~(\ref{eq:proof-obsnf-Form-DF}) und~(\ref{eq:proof-obsnf-Form-DG})
folgt, dass das kombinierte Vektorfeld $(\bar{f}+\bar{g})$ die in
Gl.~(\ref{eq:nf}) angegebene Gestalt besitzt.

\notwendig\ Angenommen, es gäbe lokale Diffeomorphismen~(\ref{eq:TS})
und~$\gamma$, die das System~(\ref{eq:plant}) in einer Umgebung
des Punktes~$p$ in die Normalform~(\ref{eq:nf}) überführen. Simultan
wird für das transformierte System~(\ref{eq:nf}) auch die Form~(\ref{eq:proof-obsnf-bar-groessen})
verwendet. Ohne Einschränkung sei $T(p)=0$ (andernfalls Verschiebung
des Koordinatensystems). Die Beobachtbarkeitsmatrix der Normalform
lässt sich leicht berechnen. Zu der Beobachtbarkeitsmatrix~(\ref{eq:Q})
in den Originalkoordinaten besteht folgender Zusammenhang: 
\begin{equation}
\begin{array}{rcl}
Q_{B}(x) & = & {\displaystyle \frac{\partial}{\partial x}}\left(\begin{array}{c}
h(x)\\
\vdots\\
L_{f}^{n-1}h(x)
\end{array}\right)\\
 & = & {\displaystyle \frac{\partial}{\partial z}}\left(\begin{array}{c}
h(S(z))\\
\vdots\\
L_{f}^{n-1}h(S(z))
\end{array}\right){\displaystyle \frac{\partial z}{\partial x}}\\
 & = & {\displaystyle \frac{\partial}{\partial z}}\left(\begin{array}{c}
\bar{h}(z)\\
\vdots\\
L_{\bar{f}}^{n-1}\bar{h}(z)
\end{array}\right){\displaystyle \frac{\partial z}{\partial x}}\\
 & = & \left.\left(\begin{array}{cccc}
0 & \cdots & 0 & \gamma^{\prime}(z_{n})\\
\vdots & \qdots & \qdots & *\\
0 & \qdots & \qdots & \vdots\\
\gamma^{\prime}(z_{n}) & * & \cdots & *
\end{array}\right)\right|_{z=T(x)}\cdot T^{\prime}(x).
\end{array}\label{eq:proof-obnf-Qx-Qz}
\end{equation}
Die Matrizen auf der rechten Seite sind im Punkt~$p$ regulär, weil~$\gamma$
und~$T$ Diffeomorphismen sind. Damit muss auch die Matrix $Q_{B}(p)$
regulär sein, d.\,h. Bedingung~\ref{enu:ki1} ist erfüllt. 

Als nächstes transformieren wir das Vektorfeld~$v$ mittels Pushforward
in die Normalform-Koordinaten: 
\[
\bar{v}(z)=T_{*}v(S(z)).
\]
In den Originalkoordinaten kann das in Gl.~(\ref{eq:v}) definierte
Startvektorfeld auch als Lösung von Gl.~(\ref{eq:proof-obsnf-gl0})
beschrieben werden. Mit Gl.~(\ref{eq:proof-obnf-Qx-Qz}) erhält man
analog in den transformierten Koordinaten die Darstellung 
\[
L_{\bar{v}}L_{\bar{f}}^{i}\bar{h}(z)=\left\langle \d L_{\bar{f}}^{i}\bar{h}(z),\bar{v}(z)\right\rangle =\left\{ \begin{array}{ccl}
0 & \mbox{für} & i=0,\ldots,n-2,\\
\gamma^{\prime}(z_{n}) & \mbox{für} & i=n-1.
\end{array}\right.
\]
Daraus folgt $\beta(h(p))=\gamma^{\prime}(0)\neq0$, so dass die Ausgangstransformation~$\gamma$
ein lokaler Diffeomorphismus ist. Für die Normalform~(\ref{eq:nf})
lässt sich unmittelbar nachrechnen, dass die Bedingungen 
\[
\begin{array}{rclcl}
[\ad_{-\bar{f}}^{i}\bar{v},\ad_{-\bar{f}}^{j}\bar{v}](z) & = & 0 & \mbox{für} & 0\leq i,j\leq n-1\\{}
[\bar{g},\ad_{-\bar{f}}^{i}\bar{v}](z,u) & = & 0 & \mbox{für} & 0\leq i\leq n-2
\end{array}
\]
in einer Umgebung von $z=0$ und für alle $u\in\mathcal{U}$ erfüllt
sind. Aus Proposition~\ref{pro:Lie-Klammer-und-Push-Forward} folgt,
dass die zugehörigen Bedingungen~\ref{enu:ki2} und~\ref{enu:ki3}
auch in den Originalkoordinaten erfüllt sind.
\end{svmultproof2}

Bedingung~\ref{enu:ki1} aus Satz~\ref{thm:existence} ist die lokale
Beobachtbarkeitsbedingung (siehe Abschnitt~\ref{subsec:Beobachtbarkeit-autonom}
und Anmerkung~\ref{rem:stetige-Beobachter}). Die Integrabilitätsbedingung\index{Integrabilitätsbedingung}~\ref{enu:ki2}
ist bei vielen Systemen nicht erfüllt. Allerdings kann man die Klasse
der in die Normalform~(\ref{eq:nf}) transformierbaren Systeme vergrößern,
indem man z.\,B. Zeitskalierungen berücksichtigt~\cite{guay2002,respondek2004}\@.
Bei autonomen Systemen entfällt die Bedingung~\ref{enu:ki3}. 

Selbst wenn die Bedingungen von Satz~\ref{thm:existence} erfüllt
sind, ist die konkrete Berechnung der Normalform~(\ref{eq:nf}) bzw.
des Beobachters~(\ref{eq:ons-orig}) in der Regel sehr schwierig.
Prinzipiell lässt sich dabei wie folgt vorgehen: Ausgehend von der
Beobachtbarkeitsmatrix~(\ref{eq:Q}) bestimmt man den Startvektor~$v$
entsprechend Gl.~(\ref{eq:v}) und die Lie-Klammern $\ad_{-f}v,\ldots,\ad_{-f}^{n-1}v$
für eine geeignete Abbildung~$\beta$. Die Abbildung~$\beta$ ist
dabei so zu wählen, dass Bedingung~\ref{enu:ki2} des Satzes erfüllt
ist. Anschließend fasst man die vektorwertigen Gln.~(\ref{eq:proof-obsnf-gl2})
spaltenweise zu der matrixwertigen Gleichung 
\begin{equation}
I=T^{\prime}(x)\cdot\Pi(x)\label{eq:pde-T}
\end{equation}
mit
\begin{equation}
\Pi(x)=\left(\begin{array}{cccc}
v(x) & \ad_{-f}v(x) & \cdots & \ad_{-f}^{n-1}v(x)\end{array}\right)\label{eq:PI}
\end{equation}
zusammen~\cite{schaffner99}. Die Matrix~$\Pi$ ist dabei die von
den Vektorfeldern~$f$ und~$v$ erzeugte Steuerbarkeitsmatrix (siehe
Anmerkung~\ref{rem:Steuerbarkeitsmatrix}). Bei Gl.~(\ref{eq:pde-T})
handelt es sich um eine partielle Differentialgleichung, die bezüglich~$T$
zu lösen ist. In \textsc{Maple} steht dazu die Routine \texttt{pdesolve}
aus der Bibliothek \texttt{PDEtools} zur Verfügung~\cite{chep-terrab1995}. 

Alternativ kann man die gesuchte Koordinatentransformation~(\ref{eq:TS})
auch unmittelbar auf Basis des im Beweis von Satz~\ref{thm:existence}
verwendeten Begradigungssatzes berechnen. Dazu bestimmt man die Flüsse
der in Gl.~(\ref{eq:PI}) auftretenden Vektor\-felder und bildet
die nachfolgende Flussverknüpfung\index{Fluss!Verkettung} (siehe
Beweis von Satz~\ref{thm:simultane-begradigung-von-VF}):
\begin{equation}
x=S(z)=\varphi_{z_{1}}^{v}\circ\varphi_{z_{2}}^{\ad_{-f}v}\circ\cdots\circ\varphi_{z_{n}}^{\ad_{-f}^{n-1}v}(p).\label{eq:verknuepfung-der-fluesse}
\end{equation}
Mit zunehmender Ordnung der Lie-Klammern werden die Vektor\-felder
$v,ad_{-f}v,\ldots,ad_{-f}^{n-1}v$ in der Regel komplizierter. Bei
dem potentiell kompliziertesten Vektor\-feld $ad_{-f}^{n-1}v$ benötigt
man jedoch nicht den allgemeinen Fluss, sondern nur die Lösung einer
Anfangswertaufgabe, die sich durch geeignete Wahl des Anfangswertes~$p$
oft vereinfachen lässt. Die Umkehrung von~$S$ liefert die Hintransformation~$T$.

Aus der Kenntnis der Transformation~(\ref{eq:TS}) erhält man die
Normalform~(\ref{eq:nf}) bzw. die Abbildungen~$\alpha$ und~$\gamma$
mit Hilfe von Gl.~(\ref{eq:proof-obsnf-bar-groessen}), d.\,h. 
\[
Az+\alpha(z_{n},u)=T_{*}F(S(z))=\left(S^{\prime}(z)\right)^{-1}F(S(z))\quad\mbox{und}\quad\gamma(z_{n})=h(S(z)).
\]
Zwischen den Funktionen~$\beta$ und~$\gamma$ besteht wegen Gl.~(\ref{eq:proof-obsnf-dh-beta})
der Zusammenhang 
\begin{equation}
\gamma^{\prime}(z_{n})=\beta(h(x))=\beta(h(S(z))).\label{eq:dgamma-beta}
\end{equation}
Die Ausgangstransformation~$\gamma$ lässt sich daher bis auf eine
Integrationskonstante durch 
\[
\gamma(z_{n})=\int\beta(h(S(z)))\,\d z_{n}
\]
bestimmen.

\begin{example}
\label{exa:van-der-Pol-Krener-Isidori}Man betrachte den van der Pol-Oszillator
\begin{equation}
\begin{array}{lcl}
\dot{x}_{1} & = & x_{2}\\
\dot{x}_{2} & = & -x_{1}+x_{2}(1-x_{1}^{2})+x_{1}u\\
y & = & x_{1}
\end{array}\label{eq:vdp}
\end{equation}
mit dem zusätzlichen Eingang~$u$ und dem Ausgang~$y$. Das van
der Pol-System~(\ref{eq:vdp}) wird in der Schaltungstechnik häufig
zur Beschreibung von Relaxationsschwingungen in Oszillatoren verwendet~\cite{van-der-pol1926,guckenheimer83,prochaska2004}.
Für $u=0$ enthält die Phasenebene eine instabile Ruhelage im Ursprung
und einen stabilen Grenzzyklus (siehe Abb.~\ref{fig:Phasenportrait-vdP}).
\end{example}
\begin{figure}
\begin{centering}
\resizebox{0.85\textwidth}{!}{\input{vdp-sim.pdftex_t}}
\par\end{centering}
\caption{Phasenebene des van der Pol-Oszillators~(\ref{eq:vdp}) für $u=0$\label{fig:Phasenportrait-vdP}}
\end{figure}

Die zum System gehörende Beobachtbarkeitsmatrix ist die Einheitsmatrix.
Dadurch ergibt sich der Startvektor nach Gl.~(\ref{eq:v}) und die
daraus berechnete Lie-Klammer:
\begin{equation}
v(x)=\left(\begin{array}{c}
0\\
\beta(x_{1})
\end{array}\right),\quad\ad_{-f}v(x)=\left(\begin{array}{c}
\beta(x_{1})\\
\left(1-x_{1}^{2}\right)\beta(x_{1})+x_{2}\beta^{\prime}(x_{1})
\end{array}\right).\label{eq:vdp-Lie-Klammern1}
\end{equation}
Zur Überprüfung der Bedingung~\ref{enu:ki2} aus Satz~\ref{thm:existence}
betrachten wir die weitere Lie-Klammer
\[
[v,\ad_{-f}v](x)=\left(\begin{array}{c}
0\\
-2\beta(x_{1})\beta^{\prime}(x_{1})
\end{array}\right).
\]
Diese Lie-Klammer ist genau dann die Nullfunktion, wenn~$\beta$
konstant ist. Mit $\beta(x_{1})\equiv1$ ist auch die Bedingung~\ref{enu:ki3}
erfüllt, d.\,h. $[g,v]=0$. Die Lie-Klammern~(\ref{eq:vdp-Lie-Klammern1})
vereinfachen sich zu 
\begin{equation}
v(x)=\left(\begin{array}{c}
0\\
1
\end{array}\right)\quad\text{und}\quad\ad_{-f}v(x)=\left(\begin{array}{c}
1\\
1-x_{1}^{2}
\end{array}\right).\label{eq:vdp-Lie-Klammern2}
\end{equation}
Zu diesen Vektorfeldern berechnet man die Flüsse
\[
\varphi_{z_{1}}^{v}(x)=\left(\begin{array}{c}
x_{1}\\
x_{2}+z_{1}
\end{array}\right)\quad\text{und}\quad\varphi_{z_{2}}^{\ad_{-f}v}(x)=\left(\begin{array}{c}
x_{1}+z_{2}\\
x_{2}+(1-x_{1}^{2})z_{2}-x_{1}z_{2}^{2}-\frac{1}{3}z_{2}^{3}
\end{array}\right).
\]
Aus der Flussverkettung~(\ref{eq:verknuepfung-der-fluesse}) erhält
man für $p=(0,0)^{T}$ die Rücktransformation 
\[
\left(\begin{array}{l}
x_{1}\\
x_{2}
\end{array}\right)=S(z)=\varphi_{z_{1}}^{v}\circ\varphi_{z_{2}}^{\ad_{-f}v}(p)=\left(\begin{array}{c}
z_{2}\\
z_{1}+z_{2}-\frac{1}{3}z_{2}^{3}
\end{array}\right).
\]
Die zugehörige Hintransformation lautet
\begin{equation}
\left(\begin{array}{l}
z_{1}\\
z_{2}
\end{array}\right)=T(x)=\left(\begin{array}{c}
x_{2}-x_{1}+\frac{1}{3}x_{1}^{3}\\
x_{1}
\end{array}\right).\label{eq:vdp-hintransformation}
\end{equation}
Durch Anwendung dieser Transformation auf System~(\ref{eq:vdp})
erhält man die Beobachternormalform
\begin{equation}
\left(\begin{array}{l}
\dot{z}_{1}\\
\dot{z}_{2}
\end{array}\right)=\underbrace{\left(\begin{array}{cc}
0 & 0\\
1 & 0
\end{array}\right)}_{{\displaystyle A}}\left(\begin{array}{l}
z_{1}\\
z_{2}
\end{array}\right)+\underbrace{\left(\begin{array}{c}
uz_{2}-z_{2}\\
z_{2}-\frac{1}{3}z_{2}^{3}
\end{array}\right)}_{{\displaystyle \alpha(z_{n},u)}},\quad y=z_{2}.\label{eq:vdp-beobachternormalform}
\end{equation}
Daraus ergibt sich unmittelbar der Normalform-Beobachter
\begin{eqnarray*}
\left(\begin{array}{l}
\dot{\hat{z}}_{1}\\
\dot{\hat{z}}_{2}
\end{array}\right) & = & \left(\begin{array}{cc}
0 & 0\\
1 & 0
\end{array}\right)\left(\begin{array}{l}
\hat{z}_{1}\\
\hat{z}_{2}
\end{array}\right)+\left(\begin{array}{c}
uy-y\\
y-\frac{1}{3}y
\end{array}\right)+\left(\begin{array}{l}
a_{0}\\
a_{1}
\end{array}\right)(y-\hat{z}_{2})\\
\hat{x} & = & S(\hat{z})
\end{eqnarray*}
nach Gl.~(\ref{eq:obs-in-nf}), wobei das charakteristische Polynom~(\ref{eq:cp})
der Fehlerdynamik~(\ref{eq:error-dgl-nf}) über die Koeffizienten
$a_{0},a_{1}>0$ vorgegeben wird. Mit Kenntnis der Koordinatentransformation
und der Eingangs-Ausgangs-Aufschaltung kann man den Beobachter auch
in den Originalkoordianten implementieren. Der zum Beobachter~(\ref{eq:ons-orig})
gehörende Korrekturterm~(\ref{eq:correction-term-nf-obs}) lautet:
\begin{equation}
\begin{array}{lcl}
k_{\infty}(\hat{x},y,u) & = & \left(\begin{array}{cc}
0 & 1\\
1 & 1-\hat{x}_{1}^{2}
\end{array}\right)\left(\left(\begin{array}{c}
uy-y\\
y-\frac{1}{3}y^{3}
\end{array}\right)\!-\!\left(\begin{array}{c}
u\hat{x}_{1}-\hat{x}_{1}\\
\hat{x}_{1}-\frac{1}{3}\hat{x}_{1}^{3}
\end{array}\right)+\left(\begin{array}{c}
a_{0}\\
a_{1}
\end{array}\right)(y-\hat{x}_{1})\right)\end{array}\!.\label{eq:vdp-kinf}
\end{equation}

\begin{example}
\label{exa:Roessler-Krener-Isidori}Man betrachte das Rössler-System~\cite{roessler76,roessler79}
\begin{equation}
\begin{array}{lcl}
\dot{x}_{1} & = & -x_{2}-x_{3}\\
\dot{x}_{2} & = & x_{1}+ax_{2}\\
\dot{x}_{3} & = & c+x_{3}(x_{1}-b)
\end{array}\label{eq:roessler-attraktor}
\end{equation}
mit den Parametern $a,b,c>0$ und der Ausgangsgleichung 
\begin{equation}
y=x_{3}.\label{eq:roessler-ausgang1}
\end{equation}
System~(\ref{eq:roessler-attraktor}) besitzt für bestimmte Anfangs-
und Parameterwerte einen chaotischen Attraktor~\cite{jetschke89,arrowsmith90}.

Für den Ausgang~(\ref{eq:roessler-ausgang1}) erhält man die Beobachtbarkeitsmatrix
\[
Q_{B}(x)=\left(\begin{array}{ccc}
0 & 0 & 1\\
x_{3} & 0 & x_{1}-b\\
c+2x_{3}(x_{1}-b) & \;-x_{3}\; & -x_{2}-2x_{3}+(x_{1}-b)^{2}
\end{array}\right).
\]
Mit Gl.~(\ref{eq:v}) berechnet man das Startvektorfeld~$v$ mit
einem noch nicht festgelegten Skalarfeld~$\beta$, welches wegen
Gl.~(\ref{eq:roessler-ausgang1}) von~$x_{3}$ abhängt:
\[
v(x)=\left(\begin{array}{c}
0\\
-\frac{\beta(x_{3})}{x_{3}}\\
0
\end{array}\right).
\]
Dieser Teil der Berechnungen lässt sich ohne Schwierigkeiten mit \textsc{Maxima}
nachvollziehen, wobei auf die in Alg.~\ref{alg:Berechnung-Beobachtbarkeitsmatrix}
definierte Routine zur Berechnung der Beobachtbarkeitsmatrix zurückgegriffen
wird:

\begin{maxima}\noindent
%%%%%%%%%%%%%%%
%%% INPUT:
\begin{minipage}[t]{8ex}\color{red}\bf
\begin{verbatim}
(%i7) 
\end{verbatim}
\end{minipage}
\begin{minipage}[t]{\textwidth}\color{blue}
\begin{verbatim}
f:[-x2-x3,x1+a*x2,c+x3*(x1-b)];
h:x3$
x:[x1,x2,x3]$
n:length(x)$
\end{verbatim}
\end{minipage}
%%% OUTPUT:

\noindent
\begin{math}\displaystyle
\parbox{10ex}{$\color{labelcolor}\mathrm{\tt (\%o4) }\quad $}
[-\mathit{x3}-\mathit{x2},a\cdot \mathit{x2}+\mathit{x1},\left( \mathit{x1}-b\right) \cdot \mathit{x3}+c]\mbox{}
\end{math}
%%%%%%%%%%%%%%%


\noindent
%%%%%%%%%%%%%%%
%%% INPUT:
\begin{minipage}[t]{8ex}\color{red}\bf
\begin{verbatim}
(%i8) 
\end{verbatim}
\end{minipage}
\begin{minipage}[t]{\textwidth}\color{blue}
\begin{verbatim}
Qb:ObservabilityMatrix(f,h,x);
\end{verbatim}
\end{minipage}
%%% OUTPUT:

\noindent
\begin{math}\displaystyle
\parbox{8ex}{$\color{labelcolor}\mathrm{\tt (\%o8) }\quad $}
\begin{pmatrix}0 & 0 & 1\cr \mathit{x3} & 0 & \mathit{x1}-b\cr 2\cdot \left( \mathit{x1}-b\right) \cdot \mathit{x3}+c & -\mathit{x3} & -2\cdot \mathit{x3}-\mathit{x2}+{{\left( \mathit{x1}-b\right) }^{2}}\end{pmatrix}\mbox{}
\end{math}
%%%%%%%%%%%%%%%


\noindent
%%%%%%%%%%%%%%%
%%% INPUT:
\begin{minipage}[t]{8ex}\color{red}\bf
\begin{verbatim}
(%i11) 
\end{verbatim}
\end{minipage}
\begin{minipage}[t]{\textwidth}\color{blue}
\begin{verbatim}
depends(β,x3)$
v:col(invert(Qb)*β,n)$
v:list_matrix_entries(v);
\end{verbatim}
\end{minipage}
%%% OUTPUT:

\noindent
\begin{math}\displaystyle
\parbox{8ex}{$\color{labelcolor}\mathrm{\tt (\%o11) }\quad $}
[0,-\frac{\beta}{\mathit{x3}},0]\mbox{}
\end{math}
\end{maxima}

Aus dem Startvektor bestimmt man die durch Lie-Klammern definierten
Vektorfelder $\ad_{-f}v,\ad_{-f}^{2}v$, bei denen schon vergleichsweise
große symbolische Ausdrücke auftreten. Zu überprüfen ist die Bedingung~\ref{enu:ki2}
aus Satz~\ref{thm:existence}. Es gilt $[v,\ad_{-f}v]=0$, $[v,\ad_{-f}^{2}v]=0$
und 
\[
[\ad_{-f}v,\ad_{-f}^{2}v](x)=\left(\begin{array}{c}
\frac{3}{x_{3}^{2}}\,\beta(x_{3})\,(\beta(x_{3})-x_{3}\beta^{\prime}(x_{3}))\\
*\\
0
\end{array}\right).
\]
Der erste Term verschwindet, wenn die Differentialgleichung $\beta(x_{3})-x_{3}\beta^{\prime}(x_{3})=0$
erfüllt ist. Die allgemeine Lösung lautet $\beta(x_{3})=C_{1}x_{3}$
mit der Konstanten~$C_{1}$. Für $\beta(x_{3})=x_{3}$ erhält man
die Vektor\-felder 
\begin{equation}
v(x)=\left(\begin{array}{r}
0\\
-1\\
0
\end{array}\right),\;\ad_{-f}v(x)=\left(\begin{array}{r}
1\\
-a\\
0
\end{array}\right),\;\ad_{-f}^{2}v(x)=\left(\begin{array}{c}
a\\
1-a^{2}\\
x_{3}
\end{array}\right).\label{eq:Roessler-Lie-Klammern}
\end{equation}
Nach Anpassung von~$\beta$ im \textsc{Maxima}-Skript erhält man
diese Vektorfelder aus den Spalten der Steuerbarkeitsmatrix (vgl.
Alg.~\ref{alg:Berechnung-Steuerbarkeitsmatrix}):

\begin{maxima}\noindent
%%%%%%%%%%%%%%%
%%% INPUT:
\begin{minipage}[t]{8ex}\color{red}\bf
\begin{verbatim}
(%i11) 
\end{verbatim}
\end{minipage}
\begin{minipage}[t]{\textwidth}\color{blue}
\begin{verbatim}
β:x3$
v:col(invert(Qb)*β,n)$
v:list_matrix_entries(v);
\end{verbatim}
\end{minipage}
%%% OUTPUT:

\noindent
$\displaystyle
\parbox{8ex}{$\color{labelcolor}\mathrm{\tt (\%o11) }\quad $}
[0,-1,0]\mbox{}
$
%%%%%%%%%%%%%%%


\noindent
%%%%%%%%%%%%%%%
%%% INPUT:
\begin{minipage}[t]{8ex}\color{red}\bf
\begin{verbatim}
(%i13) 
\end{verbatim}
\end{minipage}
\begin{minipage}[t]{\textwidth}\color{blue}
\begin{verbatim}
ControllabilityMatrix(f,v,x);
\end{verbatim}
\end{minipage}
%%% OUTPUT:

\noindent
$\displaystyle
\parbox{8ex}{$\color{labelcolor}\mathrm{\tt (\%o13) }\quad $}
\begin{pmatrix}0 & 1 & a\cr -1 & -a & 1-{{a}^{2}}\cr 0 & 0 & \mathit{x3}\end{pmatrix}\mbox{}
$
%%%%%%%%%%%%%%%
\end{maxima}

Die Bedingung~\ref{enu:ki2} aus Satz~\ref{thm:existence} ist jetzt
erfüllt, d.\,h. es gilt zusätzlich $[\ad_{-f}v,\ad_{-f}^{2}v]=0$.
Aus der Verknüpfung der Flüsse
\[
\varphi_{z_{1}}^{v}(x)=\left(\begin{array}{c}
x_{1}\\
x_{2}-z_{1}\\
x_{3}
\end{array}\right),\quad\varphi_{z_{2}}^{\ad_{-f}v}(x)=\left(\begin{array}{c}
x_{1}+z_{2}\\
x_{2}-az_{2}\\
x_{3}
\end{array}\right)
\]
und
\[
\varphi_{z_{3}}^{\ad_{-f}^{2}v}(x)=\left(\begin{array}{c}
x_{1}+az_{3}\\
x_{2}+(1-a^{2})z_{3}\\
x_{3}\exp(z_{3})
\end{array}\right)
\]
erhält man entsprechend Gl.~(\ref{eq:verknuepfung-der-fluesse})
für den Bezugspunkt $p=(0,0,1)^{T}$ die Koordinaten\-transformation
\begin{equation}
\begin{array}{lcl}
x_{1} & = & z_{2}+az_{3}\\
x_{2} & = & -z_{1}-az_{2}+(1-a^{2})z_{3}\\
x_{3} & = & \exp(z_{3})
\end{array}\label{eq:roessler-S}
\end{equation}
mit der Umkehrung 
\begin{equation}
\begin{array}{lcl}
z_{1} & = & -ax_{1}-x_{2}+\ln x3\\
z_{2} & = & x_{1}-a\ln x_{3}\\
z_{3} & = & \ln x_{3}
\end{array}\label{eq:roessler-T}
\end{equation}
für $x_{3}>0$, vgl.~\cite{nijmeijer97}. Mit Gl.~(\ref{eq:roessler-S})
bekommt man die in der Normalform~(\ref{eq:nf}) auftretenden Abbildungen
\begin{equation}
\alpha(z_{3})=\left(\begin{array}{c}
-b+a\e^{z_{3}}+c\e^{-z_{3}}\\
-z_{3}+ab-\e^{z_{3}}-ac\e^{-z_{3}}\\
-b+az_{3}+c\e^{-z_{3}}
\end{array}\right)\quad\mbox{und}\quad\gamma(z_{3})=\exp z_{3}.\label{eq:roessler-alpha-gamma}
\end{equation}
Bei Kenntnis von~(\ref{eq:roessler-S}) und~(\ref{eq:roessler-alpha-gamma})
kann man mit der Umkehrabbildung $z_{3}=\gamma^{-1}(y)=\ln y$ den
Normalform-Beobachter~(\ref{eq:obs-in-nf}) angeben. 
\end{example}

\begin{example}
Für das inverse Pendel mit Gleichstrommotor aus Beispiel~\ref{exa:inverses-pendel-motor-nf-beobachter}
berechnet man die Beobachtbarkeitsmatrix 
\[
Q_{B}(x)=\left(\begin{array}{ccl}
1 & 0 & 0\\
0 & 1 & 0\\
\frac{Mg\ell}{J}\cos x_{1} & -\frac{d}{J} & \frac{K}{J}
\end{array}\right).
\]
Mit der Wahl $\beta=1$ (und damit $\gamma(z_{3})=z_{3}$, vgl. Gl.~(\ref{eq:dgamma-beta}))
erhält man den Startvektor $v=\tfrac{J}{K}\tfrac{\partial}{\partial x_{3}}$
und die konstante Matrix 
\[
\Pi=\left(\begin{array}{ccc}
0 & 0 & 1\\
0 & 1 & -\frac{R}{L}-\frac{d}{J}\\
\frac{J}{K} & -\frac{JR}{KL} & \frac{JR^{2}}{KL^{2}}-\frac{K}{L}
\end{array}\right).
\]
Daraus erhält man die \textit{lineare} Transformation
\[
\left(\begin{array}{c}
z_{1}\\
z_{2}\\
z_{3}
\end{array}\right)=\left(\begin{array}{ccc}
0 & 0 & 1\\
0 & 1 & -\frac{R}{L}-\frac{d}{J}\\
\frac{J}{K} & -\frac{JR}{KL} & \frac{JR^{2}}{KL^{2}}-\frac{K}{L}
\end{array}\right)\left(\begin{array}{c}
x_{1}\\
x_{2}\\
x_{3}
\end{array}\right).
\]
Die Anwendung der Transformation auf das System~(\ref{eq:inverses-pendel-motor-nf-strecke})
führt auf die Eingangs-Ausgangs-Aufschaltung
\[
\alpha(y,u)=\frac{1}{JL}\left(\begin{array}{c}
Mg\ell\sin y+Ku\\
Mg\ell\sin y-(dR+K^{2})y\\
-(dL+JR)y
\end{array}\right),
\]
mit der man den Korrekturterm~(\ref{eq:correction-term-nf-obs})
des Beobachters~(\ref{eq:ons-orig}) berechnen kann. Zwischen dem
Simulationsterm $F(\hat{x},u)$ und dem Korrekturterm $k_{\infty}(\hat{x},y,u)$
entfällt der Term $\sin\hat{x}_{1}$, so dass man den schon berechneten
Beobachter~(\ref{eq:inverses-pendel-motor-nf-beobachter}) mit den
Verstärkungsfaktoren~(\ref{eq:inverses-pendel-motor-nf-verstaerkung})
erhält.
\end{example}

\begin{remark}
Die in Satz~\ref{thm:existence} formulierten Existenzbedingungen
sind sehr restriktiv. Für Systeme mit mehreren Ausgängen kann man
ähnliche Bedingungen angeben~~\cite{krener85observer,xia88,xia89}.
Diese Restriktionen lassen sich beispielsweise dadurch abschwächen,
dass man Zeitskalierungen~\cite{guay2002,respondek2004,wang2010}
zulässt. Anstelle einer reinen Zustandstransformation kann man die
Auswahl der Normalform-Koordinaten auch vom Eingang und seinen Zeit\-ab\-leitungen
abhängig machen~\cite{zeitz87,zeitz87correspondence,proychev1993,glumineau1996,plestan1997,mishkov2005}.
Statt einer statischen Aufschaltung des Ausgangs ist auch eine dynamische
Aufschaltung denkbar, bei der bestimmte Zeitableitungen des Ausgangs
berücksichtigt werden. Man spricht dann von einer dynamischen Linearisierung
des Beobachtungsfehlers~\cite{noh2001,noh2004}. Bei Mehrgrößensystemen
ist oft ein dezentraler Beobachterentwurf auf der Basis einer Block-Dreicks-Form
vorteilhaft~\cite{rudolph1994,schaffner97,schaffner97ecc,schaffner98,schaffner99ijss,schaffner99,wang2006}.
Mitunter ist die exakte Linearisierung des Beobachtungsfehlers erst
durch Einbettung eines gegebenen nichtlinearen Systems in einen höherdimensionalen
Raum möglich~\cite{back2002,back2005}. Falls die exakte Linearisierung
des Beobachtungsfehlers nicht möglich ist, existieren zahlreiche Ansätze,
um eine näherungseise Linearisierung zu erzielen (siehe~\cite{karahan89,nicosia89,krener91poincare,bortoff95,banaszuk97,moreno2005syncod}
und Abschnitt~\ref{sec:Approximation-of-observers}).
\end{remark}

\begin{remark}
\label{rem:Boebachter-mit-Hilfssatz-von-Ljapunov}Ein anderer, vielversprechender
Ansatz zur exakten Linearisierung des Beobachtungsfehlers wurde in~\cite{kazantzis1998,krener2002,xiao2006}
vorgestellt. Dieser Entwurfsansatz basiert auf einem Hilfssatz von
Ljapunov und lässt in der Normalform eine nicht\-lineare Ausgangsabbildung
zu. Jedoch ist die Implementierung des Beobachters in der Regel nur
auf Basis numerischer Näherungsverfahren möglich~\cite{deutscher2006,deutscher2007}.
\end{remark}

\begin{remark}
\label{rem:stetige-Beobachter}Die Regularitätsbedingungen aus Satz~\ref{thm:existence}
(d.\,h. $\beta(h(p))\neq0$ und die Beobachtbarkeitsbedingung~\ref{enu:ki1})
garantieren, dass die Umkehrabbildungen von~$\gamma$ und~$T$ existieren
und eine entsprechende Glattheit aufweisen (Satz über die Umkehrabbildung\index{Satz!über die Umkehrabbildung}),
d.\,h. dass sie Diffeomorphismen sind. Bei manchen Systemen ist die
Beobachtbarkeitsmatrix zwar in einer punktierten Umgebung von $p\in\mathcal{M}$
regulär, nicht aber im Punkt~$p$ selber. Dann kann das System~(\ref{eq:plant})
nicht mit Diffeomorphismen in die Normalform~(\ref{eq:nf}) überführt
werden, möglicherweise aber mit Semidiffeomorphismen. Unter einem
\emph{Semidiffeomorphismus}\index{Semidiffeomorphismus} versteht
man eine \index{Abbildung!bijektive}bijektive glatte Abbildung, deren
Umkehrabbildung stetig (aber nicht notwendigerweise stetig differenzierbar)
ist. Ist eine solche semidiffeomorphe Transformation in die Normalform~(\ref{eq:nf})
möglich, dann ist der Einsatz des Beobachters~(\ref{eq:obs-in-nf})
denkbar, den man in diesem Fall \emph{stetigen Beobachter}\index{Beobachter!stetiger}
nennt~\cite{xia97,zeitz98,schaffner99}.
\end{remark}


\subsection{Existenz und Berechnung der Normalform nach Nam\label{subsec:Berechnung-nach-Nam}}

Für die Existenz der Beobachter-Normalform~(\ref{eq:nf}) und die
konkrete Konstruktion der Koordinatentransformation~(\ref{eq:TS})
ist Bedingung~\ref{enu:ki2} aus Satz~\ref{thm:existence} entscheidend.
Dabei fasst man die Spalten der in Gl.~(\ref{eq:PI}) definierten
Matrix~$\Pi$ als Vektorfelder auf. Eine äquivalente Existenzaussage
ist auch über die Zeilen der inversen Matrix 
\begin{equation}
\left(\begin{array}{c}
\omega^{1}(x)\\
\vdots\\
\omega^{n}(x)
\end{array}\right):=\Pi^{-1}(x)\label{eq:Pinv}
\end{equation}
möglich, wobei man $\omega^{1},\ldots,\omega^{n}:\mathcal{M}\to(\R^{n})^{*}$
als Kovektorfelder bzw. Differentialformen ersten Grades betrachtet~\cite[Proposition~1]{nam97}:
\begin{theorem}
\label{thm:existence-nam}In einer Umgebung von $p\in\mathcal{M}$
existiert genau dann ein lokaler Diffeomorphismus~(\ref{eq:TS}),
der das System~(\ref{eq:plant-fg}) in die Normalform~(\ref{eq:nf})
mit einem lokalen Diffeomorphismus~$\gamma$ überführt, wenn für
eine geeignete Funktion~$\beta$ mit $\beta(h(p))\neq0$ die Bedingungen
\begin{enumerate}
\item \label{enu:nam1}$\rank\,Q_{B}(p)=n$

\item \label{enu:nam2}$\d\omega^{k}(x)=0$ für $1\leq i,j\leq n$
\item \label{enu:nam3}$(L_{g}\omega^{k}\wedge\d h)(x,u)=0$ für $1\leq i\leq n$
\end{enumerate}
für alle $x$ in einer Umgebung von~$p$ und für alle $u\in\mathcal{U}$
erfüllt sind. 
\end{theorem}
Wir zeigen nur eine Richtung. In der umgekehrten Beweisrichtung lassen
sich die Bedingungen unmittelbar an der Normalform überprüfen.
\begin{svmultproof2}
Die Bedingungen~\ref{enu:nam1}-\ref{enu:nam3} seien erfüllt. Aufgrund
von Bedingung~\ref{enu:nam1} folgen wir zunächst dem Beweis von
Satz~\ref{thm:existence} bis zu Gl.~(\ref{eq:proof-obsnf-gl2}).
Diese Gleichung lässt sich in der Matrixform~(\ref{eq:pde-T}) schreiben.
Daraus ist ersichtlich, dass~(\ref{eq:Pinv}) mit der Jacobimatrix
der Hintransformation~$T$ übereinstimmen muss. Nach dem Lemma von
Poincaré\index{Lemma!von Poincaré} (Lemma~\ref{lem:poincare-formen})
existieren Skalarfelder $t_{1},\ldots,t_{n}$ mit $\d t_{k}=\omega^{k}$
für $k=1,\ldots,n$. Die Funktionen $t_{1},\ldots,t_{n}$ sind wegen~(\ref{eq:pde-T})
die Komponenten des Diffeomorphismus~$T$ aus Gl.~(\ref{eq:TS})
und können durch elementare Integration 
\[
t_{k}(x)=\int\limits _{0}^{1}\left(\sum_{i=1}^{n}\omega_{i}^{k}(\tau x+(1-\tau)p)(x_{i}-p_{i})\right)\d\tau
\]
aus den Komponenten $\omega_{1}^{k},\ldots,\omega_{n}^{k}$ der Differentialformen~$\omega^{k}$
bestimmt werden (vgl. Beweis von Lemma~\ref{lem:poincare}).

Zum Verständnis von Bedingung~\ref{enu:nam3} betrachten wir eine
beliebige Zeile $i\in\{1,\ldots,n\}$ des transformierten Systems:
\[
\begin{array}{lcl}
\dot{z}_{i} & = & \frac{\d}{\d t}t_{i}(x)\\
 & = & \d t_{i}(x)\cdot\dot{x}\\
 & = & \d t_{i}(x)\cdot\left(f(x)+g(x,u)\right)\\
 & = & L_{f}t_{i}(x)+L_{g}t_{i}(x,u).
\end{array}
\]
Die Komponente $L_{g}t_{i}(x,u)$ des Eingangsvektorfeldes darf neben
dem Eingang nur vom Ausgang, also von $h(x)$ abhängen. Diese gewünschte
funktionale Abhängigkeit wird über die lineare Abhängigkeit der Gradienten
geprüft. Wegen
\[
\d L_{g}t_{i}(x,u)=L_{g}\d t_{i}(x,u)=L_{g}\omega^{i}(x,u)
\]
liegt die gewünschte Struktur genau dann vor, wenn die Kovektoren
$L_{g}\omega^{i}(x,u)$ und $\d h(x)$ linear abhängig sind. Das wird
in Bedingung~\ref{enu:nam3} entsprechend Prop.~\ref{prop:Lineare-Unabh-Kovektoren-Keil}
verifiziert.
\end{svmultproof2}

Die Überprüfung der linearen Abhängigkeit entsprechend Bedingung~\ref{enu:nam3}
von Satz~\ref{thm:existence-nam} kann auch ohne Verwendung des Keilprodukts\index{Keilprodukt}\index{Produkt!Keil-},
also im Sinne klassischen Matrizenrechnung, erfolgen~\cite[Corollary~1]{nam97}:
\begin{corollary}
Bedingung~\ref{enu:nam3} von Satz~\ref{thm:existence-nam} ist
gleichbedeutend mit
\[
\rank\left(\begin{array}{c}
\d h(x,u)\\
L_{g}\omega^{1}(x,u)\\
\vdots\\
L_{g}\omega^{n}(x,u)
\end{array}\right)=1
\]
für alle $x$ in einer Umgebung von~$p$ und für alle $u\in\mathcal{U}$.
\end{corollary}

\begin{example}
\label{exa:van-der-pol-nam}Die Regularität der Beobachtbarkeitsmatrix
des van der Pol-Systems~(\ref{eq:vdp}) aus Beispiel~\ref{exa:van-der-Pol-Krener-Isidori}
wurde bereits überprüft. Aus den Lie-Klammern~(\ref{eq:vdp-Lie-Klammern2})
ergibt sich
\begin{equation}
\Pi(x)=\left(\begin{array}{cc}
0 & 1\\
1 & 1-x_{1}^{2}
\end{array}\right)\quad\text{und}\quad\Pi^{-1}(x)=\left(\begin{array}{cc}
x_{1}^{2}-1 & 1\\
1 & 0
\end{array}\right)=:\left(\begin{array}{c}
\omega^{1}(x)\\
\omega^{2}(x)
\end{array}\right).\label{eq:vdp-nam-omega}
\end{equation}
Die Zeilen von~$\Pi^{-1}$ bilden nach Gl.~(\ref{eq:Pinv}) die
Kovektorfelder~$\omega^{1}$ und~$\omega^{2}$, welche die Bedingung~\ref{enu:nam2}
des Satzes erfüllen und die Potentiale~$t_{1}$ und~$t_{2}$ besitzen:
\[
\begin{array}{lclclcl}
\omega^{1}(x) & = & (x_{1}^{2}-1,1) & \quad\Rightarrow\quad & t_{1}(x) & = & \frac{1}{3}x_{1}^{3}-x_{1}+x_{2},\\
\omega^{2}(x) & = & (1,0) & \quad\Rightarrow\quad & t_{2}(x) & = & x_{1}.
\end{array}
\]
Damit erhält man die bereits in Gl.~(\ref{eq:vdp-hintransformation})
angegebene Hintransformation. Die Integration der Differentialformen
lässt sich mit der Funktion \texttt{potential} des \textsc{Maxima}-Pakets
\texttt{vect} durchführen (vgl. Beispiel~\ref{exa:Integration-Differentialform-Poincare}):

\begin{maxima}\noindent
%%%%%%%%%%%%%%%
%%% INPUT:
\begin{minipage}[t]{8ex}\color{red}\bf
\begin{verbatim}
(%i1) 
\end{verbatim}
\end{minipage}
\begin{minipage}[t]{\textwidth}\color{blue}
\begin{verbatim}
load(vect)$
\end{verbatim}
\end{minipage}


\noindent
%%%%%%%%%%%%%%%
%%% INPUT:
\begin{minipage}[t]{8ex}\color{red}\bf
\begin{verbatim}
(%i4) 
\end{verbatim}
\end{minipage}
\begin{minipage}[t]{\textwidth}\color{blue}
\begin{verbatim}
w1:[x1^2-1,1];
w2:[1,0];
x:[x1,x2]$
\end{verbatim}
\end{minipage}
%%% OUTPUT:

\noindent
$\displaystyle
\parbox{8ex}{$\color{labelcolor}\mathrm{\tt (\%o2) }\quad $}
[{{\mathit{x1}}^{2}}-1,1]$

\noindent
$\displaystyle
\parbox{8ex}{$\color{labelcolor}\mathrm{\tt (\%o3) }\quad $}
[1,0]\mbox{}$
%%%%%%%%%%%%%%%


\noindent
%%%%%%%%%%%%%%%
%%% INPUT:
\begin{minipage}[t]{8ex}\color{red}\bf
\begin{verbatim}
(%i6) 
\end{verbatim}
\end{minipage}
\begin{minipage}[t]{\textwidth}\color{blue}
\begin{verbatim}
t1:potential(w1,x);
t2:potential(w2,x);
\end{verbatim}
\end{minipage}
%%% OUTPUT:

\noindent
$\displaystyle
\parbox{8ex}{$\color{labelcolor}\mathrm{\tt (\%o5) }\quad $}
\frac{-3\cdot \mathit{x1}+{{\mathit{x1}}^{3}}+3\cdot \mathit{x2}}{3}$

\noindent
$\displaystyle
\parbox{8ex}{$\color{labelcolor}\mathrm{\tt (\%o6) }\quad $}
\mathit{x1}\mbox{}
$
%%%%%%%%%%%%%%%
\end{maxima}

Aus den Kovektorfeldern~(\ref{eq:vdp-nam-omega}) berechnet man die
Lie-Ableitungen 
\[
L_{g}\omega^{1}(x,u)=(u,0)\quad\text{und}\quad L_{g}\omega^{2}(x,u)=(0,0)
\]
in Richtung des eingangsabhängigen Vektorfeldes $g(x,u)=x_{1}u\tfrac{\partial}{\partial x_{2}}$.
Die Kovektorfelder $L_{g}\omega^{1},L_{g}\omega^{2}$ sind linear
abhängig von $\d h=\d x_{1}$, so dass auch die Bedingung~\ref{enu:nam3}
erfüllt ist.
\end{example}

\begin{example}
\label{exa:Roessler-Nam}Aus den in Gl.~(\ref{eq:Roessler-Lie-Klammern})
angegebenen Lie-Klammern erhält man nach Gl.~(\ref{eq:PI}) die Matrix
\[
\Pi(x)=\left(\begin{array}{ccc}
0 & 1 & a\\
-1 & -a & 1-a^{2}\\
0 & 0 & x_{3}
\end{array}\right).
\]
Aus ihrer Inversen~$\Pi^{-1}$ 
\[
\Pi^{-1}(x)=\left(\begin{array}{ccc}
-a & -1 & \frac{1}{x_{3}}\\
1 & 0 & -\frac{a}{x_{3}}\\
0 & 0 & \frac{1}{x_{3}}
\end{array}\right)=:\left(\begin{array}{c}
\omega^{1}(x)\\
\omega^{2}(x)\\
\omega^{3}(x)
\end{array}\right)
\]
liest man die Kovektorfelder bzw. Differentialformen $\omega^{1},\omega^{2},\omega^{3}$
ab. Diese Differentialformen sind geschlossen und daher (lokal) exakt.
Die zugehörigen Potentiale liefern die Transformation~(\ref{eq:roessler-T}).
\end{example}

\subsection{Berechnung nach Keller und Phelps\label{subsec:Berechnung-nach-Keller}}

Ein völlig anderer Ansatz zur Berechnung der Beobachter-Normalform~(\ref{eq:nf})
wurde in~\cite{keller86,keller86diss,keller87,phelps1988,phelps1991}
vorgestellt. Analog zur Aufspaltung des Vektorfeldes~$F$ in die
Vektorfelder~$f$ und~$g$ in Gl.~(\ref{eq:plant-fg}) zerlegen
wir auch die Aufschaltung~$\alpha$ des transformierten Systems~(\ref{eq:nf})
mit 
\begin{eqnarray}
Az+\bar{\alpha}(y) & = & T^{\prime}(x)f(x)\label{eq:keller-alpha-f}\\
\tilde{\alpha}(y,u) & = & T^{\prime}(x)g(x,u)\label{eq:keller-alpha-g}
\end{eqnarray}
in einen autonomen Teil~$\bar{\alpha}$ sowie einen eingangsabhängigen
Teil~$\tilde{\alpha}$ (vgl. Gl.~(\ref{eq:proof-obsnf-bar-groessen})).
Zusätzlich werden diese Größen in Abhängigkeit von~$y$ statt von~$z_{n}$
dargestellt. Das ist möglich, da die Ausgangstransformation~$\gamma$
als Diffeomorphismus angesetzt wurde.

Zur Bestimmung der Komponentenfuntionen $t_{1},\ldots,t_{n}:\mathcal{M}\to\R$
der Koordinatentransformation 
\begin{equation}
z=T(x)=\left(\begin{array}{c}
t_{1}(x)\\
\vdots\\
t_{n}(x)
\end{array}\right)\label{eq:keller-T-Ansatz}
\end{equation}
vergleicht man zeilenweise beide Seiten von Gl.~(\ref{eq:keller-alpha-f}):
\begin{equation}
\begin{array}{rclcrcl}
 &  & \bar{\alpha}_{1}(y) & = & t_{1}^{\prime}(x)f(x) & = & L_{f}t_{1}(x)\\
t_{1}(x) & + & \bar{\alpha}_{2}(y) & = & t_{2}^{\prime}(x)f(x) & = & L_{f}t_{2}(x)\\
 &  &  & \vdots &  & \vdots\\
t_{n-2}(x) & + & \bar{\alpha}_{n-1}(y) & = & t_{n-1}^{\prime}(x)f(x) & = & L_{f}t_{n-1}(x)\\
t_{n-1}(x) & + & \bar{\alpha}_{n}(y) & = & t_{n}^{\prime}(x)f(x) & = & L_{f}t_{n}(x).
\end{array}\label{eq:keller-vergleich-alpha-f}
\end{equation}
Sukzessives Einsetzen von der letzten bis zur zweiten Gleichung liefert
eine Darstellung der ersten $n-1$ Komponentenfunktionen: 
\begin{equation}
\begin{array}{rcl}
t_{n-1}(x) & = & L_{f}t_{n}(x)-\bar{\alpha}_{n}(y)\\
t_{n-2}(x) & = & L_{f}t_{n-1}(x)-\bar{\alpha}_{n-1}(y)\\
 & = & L_{f}^{2}t_{n}(x)-L_{f}\bar{\alpha}_{n}(y)-\bar{\alpha}_{n-1}(y)\\
 & \vdots\\
t_{1}(x) & = & L_{f}t_{2}(x)-\bar{\alpha}_{2}(y)\\
 & = & L_{f}^{n-1}t_{n}(x)-L_{f}^{n-2}\bar{\alpha}_{n}(y)-\cdots-L_{f}\bar{\alpha}_{3}(y)-\bar{\alpha}_{2}(y).
\end{array}\label{eq:Keller-trans1}
\end{equation}
Die letzte Komponentenfunktion~$t_{n}$ hat die Form 
\begin{equation}
z_{n}=t_{n}(x)=\lambda(y),\label{eq:Keller-transn}
\end{equation}
wobei~$\lambda=\gamma^{-1}$ die Umkehrfunktion der in der Normalform~(\ref{eq:nf})
auftretenden Ausgangstransformation~$\gamma$ ist. Die erste Zeile
von Gl.~(\ref{eq:keller-vergleich-alpha-f}) führt dann auf die partielle
Differentialgleichung 
\begin{equation}
L_{f}^{n}\lambda(y)=L_{f}^{n-1}\bar{\alpha}_{n}(y)+L_{f}^{n-2}\bar{\alpha}_{n-1}(y)+\cdots+L_{f}\bar{\alpha}_{2}(y)+\bar{\alpha}_{1}(y),\label{eq:ncg}
\end{equation}
die in~\cite{keller86} als \emph{nichtlineare charakteristische
Gleichung} bezeichnet wird. Diese Gleichung ist bezüglich der Funktionen
$\lambda,\bar{\alpha}_{1},\ldots,\bar{\alpha}_{n}$ zu lösen. Dann
kann man die Koordinatentransformation~(\ref{eq:keller-T-Ansatz})
explizit angeben: 
\[
z=T(x)=\left(\begin{array}{c}
L_{f}^{n-1}\lambda(y)-L_{f}^{n-2}\bar{\alpha}_{n}(y)-\cdots-L_{f}\bar{\alpha}_{3}(y)-\bar{\alpha}_{2}(y)\\
L_{f}^{n-2}\lambda(y)-L_{f}^{n-3}\bar{\alpha}_{n}(y)-\cdots-\bar{\alpha}_{3}(y)\\
\vdots\\
L_{f}\lambda(y)-\bar{\alpha}_{n}(y)\\
\lambda(y)
\end{array}\right).
\]

In den Originalkoordinaten ist es kaum möglich, Gl.~(\ref{eq:ncg})
bezüglich der Funktionen $\lambda,\bar{\alpha}_{1},\ldots\bar{\alpha}_{n}$
zu lösen. Die entscheidende Idee von~\cite{keller86,keller86diss,keller87,phelps1988,phelps1991}
besteht darin, in einem Zwischenschritt das System~(\ref{eq:plant-fg})
mit der Beobachtbarkeitsabbildung~$q$ nach Gl.~(\ref{eq:Q}) in
die Beobachtbarkeitsnormalform\index{Beobachtbarkeitsnormalform}\index{Normalform!Beobachtbarkeits-}
\begin{equation}
\left(\begin{array}{c}
\dot{\xi}_{1}\\
\vdots\\
\dot{\xi}_{n-1}\\
\dot{\xi}_{n}
\end{array}\right)=\overbrace{\left(\begin{array}{c}
\xi_{2}\\
\vdots\\
\xi_{n}\\
f_{n}^{*}(\xi)
\end{array}\right)}^{{\displaystyle f^{*}(\xi)}}+\overbrace{\left(\begin{array}{c}
g_{1}^{*}(\xi_{1},u)\\
\vdots\\
g_{n-1}^{*}(\xi_{1},\ldots,\xi_{n-1},u)\\
g_{n}^{*}(\xi_{1},\ldots,\xi_{n},u)
\end{array}\right)}^{{\displaystyle g^{*}(\xi,u)}},\quad y=\xi_{1}\label{eq:beobachtbarkeitsnormalform-keller}
\end{equation}
mit
\[
\begin{array}{lcl}
f_{n}^{*}(\xi) & = & L_{f}^{n}h(q^{-1}(\xi))\\
g_{i}^{*}(\xi_{1},\ldots,\xi_{i},u) & = & L_{g}L_{f}^{i-1}h(q^{-1}(\xi)),\quad i=1,\ldots,n
\end{array}
\]
zu überführen (siehe Abschnitt~\ref{subsec:Beobachterentwurf-mit-eingang}).
Die spezielle Form der Eingangsvektorfelder ist hinreichend, um die
gleichmäßige lokale Beobachtbarkeit zu sichern (Korollar~\ref{cor:Beobachtbarkeit-nicht-affin-MISO}). 

In den $\xi$-Koordinaten der Beobachtbarkeitsnormalform~(\ref{eq:beobachtbarkeitsnormalform-keller})
berechnet man jetzt die in~(\ref{eq:ncg}) auftretenden Lie-Ableitungen
in Richtung des transformierten Vektorfeldes~$f^{*}$. Diese kann
man als Zeitableitungen des autonomes Teil der Beobachtbarkeitsnormalform~(\ref{eq:beobachtbarkeitsnormalform-keller})
betrachten, d.\,h. von 
\begin{equation}
\dot{\xi}=f^{*}(\xi),\quad y=\xi_{1}.\label{eq:beobachtbarkeitsnormalform-keller-autonom}
\end{equation}
Die Berechnung der Normalform wird zunächst für Systeme der Ordnung~$2$
und~$3$ illustriert. Anschließend wird der allgemeine Fall behandelt.


\subsubsection*{Spezialfall: Ordnung~2}

Für ein nur von $\xi_{1}$ (also vom Ausgang) abhängiges Skalarfeld~$\nu$
haben die Lie-Ableitungen als Zeitableitungen von~(\ref{eq:beobachtbarkeitsnormalform-keller-autonom})
die Form
\begin{equation}
\begin{array}{lcl}
L_{f^{*}}\nu(\xi_{1}) & = & \frac{\d}{\d t}\nu(\xi_{1})\\
 & = & \nu^{\prime}(\xi_{1})\dot{\xi}_{1}\\
 & = & \nu^{\prime}(\xi_{1})\xi_{2}
\end{array}\label{eq:keller-Lie1}
\end{equation}
und
\[
\begin{array}{lcl}
L_{f^{*}}^{2}\nu(\xi_{1}) & = & \frac{\d}{\d t}L_{f^{*}}\nu(\xi_{1})\\
 & = & \nu^{\prime\prime}(\xi_{1})\dot{\xi}_{1}\xi_{2}+\nu^{\prime}(\xi_{1})\dot{\xi}_{2}\\
 & = & \nu^{\prime\prime}(\xi_{1})\xi_{2}^{2}+\nu^{\prime}(\xi_{1})f_{2}^{*}(\xi).
\end{array}
\]
Wertet man damit die Gl.~(\ref{eq:ncg}) mit den Lie-Ableitungen
für $\nu\in\{\lambda,\bar{\alpha}_{1},\bar{\alpha}_{2}\}$ aus, so
erhält man
\[
\lambda^{\prime\prime}(\xi_{1})\xi_{2}^{2}+\lambda^{\prime}(\xi_{1})f_{2}^{*}(\xi)=\bar{\alpha}_{2}^{\prime}(\xi_{1})\xi_{2}+\bar{\alpha}_{1}(\xi_{1}).
\]
Diesen Ausdruck kann man nach~$f_{2}^{*}(\xi)$ auflösen:
\begin{equation}
f_{2}^{*}(\xi)=\underbrace{\frac{\bar{\alpha}_{1}(\xi_{1})}{\lambda^{\prime}(\xi_{1})}}_{{\displaystyle k_{0}(\xi_{1})}}+\underbrace{\frac{\bar{\alpha}_{2}^{\prime}(\xi_{1})}{\lambda^{\prime}(\xi_{1})}}_{{\displaystyle k}_{1}(\xi_{1})}\,\xi_{2}\underbrace{-\frac{\lambda^{\prime\prime}(\xi_{1})}{\lambda^{\prime}(\xi_{1})}}_{{\displaystyle k}_{2}(\xi_{1})}\,\xi_{2}^{2}.\label{eq:keller-f2}
\end{equation}
Man würde zunächst die Beobachtbarkeitsnormalform~(\ref{eq:beobachtbarkeitsnormalform-keller})
und damit $f_{2}^{*}(\xi)$ berechnen. Die Faktoren $k_{0},k_{1},k_{2}$
erhält man mit einem Koeffizientenvergleich bzw. durch Ableitungsbildung
\[
k_{0}(\xi)=f_{2}^{*}(\xi),\quad k_{1}(\xi)=\frac{\partial f_{2}^{*}(\xi)}{\partial\xi_{2}},\quad k_{2}(\xi)=\frac{1}{2}\frac{\partial^{2}f_{2}^{*}(\xi)}{\partial\xi_{2}^{2}}
\]
 an der Stelle $\xi_{2}=0$. Daraus ergeben sich Gleichungen für die
gesuchten Funktionen 
\begin{equation}
\begin{array}{lcl}
\lambda^{\prime\prime}(\xi_{1}) & = & -\lambda^{\prime}(\xi_{1})\frac{1}{2}\frac{\partial^{2}f_{2}^{*}(\xi)}{\partial\xi_{2}^{2}},\\
\bar{\alpha}_{1}(\xi_{1}) & = & \lambda^{\prime}(\xi_{1})f_{2}^{*}(\xi),\\
\bar{\alpha}_{2}^{\prime}(\xi_{1}) & = & \lambda^{\prime}(\xi_{1})\frac{\partial f_{2}^{*}(\xi)}{\partial\xi_{2}}.
\end{array}\label{eq:Keller-Funktionen-n2}
\end{equation}


\subsubsection*{Spezialfall: Ordnung~3}

Für $n=3$ haben die Lie-Ableitungen einer nur von $\xi_{1}$ abhängigen
Funktion die Form~(\ref{eq:keller-Lie1}) sowie
\[
\begin{array}{lcl}
L_{f^{*}}^{2}\nu(\xi_{1}) & = & \frac{\d}{\d t}L_{f^{*}}\nu(\xi_{1})\\
 & = & \nu^{\prime\prime}(\xi_{1})\dot{\xi}_{1}\xi_{2}+\nu^{\prime}(\xi_{1})\dot{\xi}_{2}\\
 & = & \nu^{\prime\prime}(\xi_{1})\xi_{2}^{2}+\nu^{\prime}(\xi_{1})\xi_{3}.
\end{array}
\]
und
\[
\begin{array}{lcl}
L_{f^{*}}^{3}\nu(\xi_{1}) & = & \frac{\d}{\d t}L_{f^{*}}^{2}\nu(\xi_{1})\\
 & = & \nu^{\prime\prime^{\prime}}(\xi_{1})\dot{\xi}_{1}\xi_{2}^{2}+2\nu^{\prime\prime}(\xi_{1})\xi_{2}\dot{\xi}_{2}+\nu^{\prime\prime}(\xi_{1})\dot{\xi}_{1}\xi_{3}+\nu^{\prime}(\xi_{1})\dot{\xi}_{3}\\
 & = & \nu^{\prime\prime^{\prime}}(\xi_{1})\xi_{2}^{3}+3\nu^{\prime\prime}(\xi_{1})\xi_{2}\xi_{3}+\nu^{\prime}(\xi_{1})f_{3}^{*}(\xi).
\end{array}
\]
Man wertet wiederum Gl.~(\ref{eq:ncg}) mit den Lie-Ableitungen für
$\nu\in\{\lambda,\bar{\alpha}_{1},\bar{\alpha}_{2},\bar{\alpha}_{3}\}$
aus. Umstellen nach $f_{3}^{*}(\xi)$ liefert
\[
f_{3}^{*}(\xi)\!=\!\frac{\bar{\alpha}_{1}(\xi_{1})}{\lambda^{\prime}(\xi_{1})}\!+\!\frac{\bar{\alpha}_{2}^{\prime}(\xi_{1})}{\lambda^{\prime}(\xi_{1})}\xi_{2}\!+\!\frac{\bar{\alpha}_{3}^{\prime\prime}(\xi_{1})}{\lambda^{\prime}(\xi_{1})}\xi_{2}^{2}\!+\!\frac{\lambda^{\prime\prime\prime}(\xi_{1})}{\lambda^{\prime}(\xi_{1})}\xi_{2}^{3}\!+\!\frac{\bar{\alpha}_{3}^{\prime}(\xi_{1})}{\lambda^{\prime}(\xi_{1})}\xi_{3}\!-\!3\frac{\lambda^{\prime\prime}(\xi_{1})}{\lambda^{\prime}(\xi_{1})}\xi_{2}\xi_{3}.
\]
Aus den Koeffizientenfunktionen zu den Monomen $1,\xi_{2},\xi_{3}$
und $\xi_{2}\xi_{3}$ berechnet man die Funktionen $\bar{\alpha}_{1},\bar{\alpha}_{2},\bar{\alpha}_{3}$
und~$\lambda$~\cite{jelali1995bericht}.

\subsubsection*{Allgemeines Vorgehen}

Bei beliebiger Dimension $n\geq2$ bestimmt man zunächst die inverse
Ausgangstransformation~$\lambda$ durch Lösen der gewöhnlichen Differentialgleichung
\begin{equation}
\lambda^{\prime\prime}(\xi_{1})=\lambda^{\prime}(\xi_{1})\,\frac{1}{n}\,\frac{\partial^{2}f_{n}^{*}(\xi)}{\partial\xi_{2}\partial\xi_{n}}.\label{eq:keller-berechnung-inverse-ausgangstransformation}
\end{equation}
In ähnlicher Weise ergeben sich die Komponenten der Ausgangsaufschaltung
\begin{equation}
\begin{array}{lcl}
\bar{\alpha}_{1}(\xi_{1}) & = & \lambda^{\prime}(\xi_{1})\cdot f_{n}^{*}(\xi)\\
\bar{\alpha}_{2}^{\prime}(\xi_{1}) & = & \lambda^{\prime}(\xi_{1})\cdot\frac{\partial f_{n}^{*}(\xi)}{\partial\xi_{2}}\\
 & \vdots\\
\bar{\alpha}_{n}^{\prime}(\xi_{1}) & = & \lambda^{\prime}(\xi_{1})\cdot\frac{\partial f_{n}^{*}(\xi)}{\partial\xi_{n}}.
\end{array}\label{eq:Keller-berechnung-alpha}
\end{equation}
Die Funktion~$f_{n}^{*}$ und ihre partiellen Ableitungen sind für
$\xi_{2}=\cdots=\xi_{n}=0$ auszuwerten. Aus Gln.~(\ref{eq:Keller-trans1})
und~(\ref{eq:Keller-transn}) bestimmt man die Transformation~$T$
und kann damit die eingangsabhängige Aufschaltung~$\tilde{\alpha}$
nach Gl.~(\ref{eq:keller-alpha-g}) berechnen. 

\begin{example}
\label{exa:van-der-Pol-Keller}Das van der Pol-System~(\ref{eq:vdp})
aus Beispiel~\ref{exa:van-der-Pol-Krener-Isidori} liegt bereits
in der Beobachtbarkeitsnormalform~(\ref{eq:beobachtbarkeitsnormalform-keller})
vor, so dass man mit $\xi=x$ die Funktion $f_{2}^{*}(\xi)=-\xi_{1}+\xi_{2}(1-\xi_{1}^{2})$
abliest. Mit $\partial^{2}f_{2}^{*}(\xi)/\partial\xi_{2}^{2}\equiv0$
gilt $\lambda^{\prime\prime}(\xi_{1})\equiv0$. Zweifache Integration
liefert $\lambda^{\prime}(\xi_{1})=C_{1}$ und $\lambda(\xi_{1})=C_{1}\xi_{1}+C_{0}$
mit den Konstanten $C_{0},C_{1}\in\R$. Die Wahl $C_{0}=0$ und $C_{1}=1$
liefert $\lambda(\xi_{1})=\xi_{1}$ bzw. $z_{n}=\lambda(y)$ mit der
als Ausgangstransformation fungierenden Umkehrabbildung $y=\gamma(z_{n})$.
Mit Gl.~(\ref{eq:dgamma-beta}) erhält man damit die in Beispiel~\ref{exa:van-der-Pol-Krener-Isidori}
gewählte Funktion $\beta\equiv1$. Nach Gl.~(\ref{eq:Keller-Funktionen-n2})
ergeben sich die Komponenten 
\[
\bar{\alpha}_{1}(\xi_{1})=1\cdot\left.f_{2}^{*}(\xi)\right|_{\xi_{2}=0}=-\xi_{1}
\]
und 
\[
\bar{\alpha}_{2}^{\prime}(\xi_{1})=1\cdot\frac{\partial f_{2}^{*}(\xi)}{\partial\xi_{2}}=1-\xi_{1}^{2}\quad\text{\text{bzw.}}\quad\bar{\alpha}_{2}(\xi_{1})=\xi_{1}-\frac{1}{3}\xi_{1}^{3}
\]
der Ausgangsaufschaltung
\[
\bar{\alpha}(y)=\left(\begin{array}{c}
-y\\
y-\frac{1}{3}y^{3}
\end{array}\right),
\]
die mit dem autonomen Teil der in Gl.~(\ref{eq:vdp-beobachternormalform})
berechneten Aufschaltung übereinstimmt. Aus Gln.~(\ref{eq:Keller-trans1})
und~(\ref{eq:Keller-transn}) erhält man übereinstimmend mit Gl.~(\ref{eq:vdp-hintransformation})
die Koordinatentransformation
\[
\begin{array}{lcl}
t_{2}(x) & = & \lambda(x_{1})\\
 & = & x_{1}\\
t_{1}(x) & = & L_{f}t_{2}(x)-\bar{\alpha}_{2}(x_{1})\\
 & = & x_{2}-x_{1}+\frac{1}{3}x_{1}^{3}.
\end{array}
\]
\end{example}

\begin{example}
Das Rössler-System~(\ref{eq:roessler-attraktor}) aus Beispiel~\ref{exa:Roessler-Krener-Isidori}
lässt sich mit der Boebachtbarkeitsabbildung
\[
q(x)=\left(\begin{array}{c}
x_{3}\\
(x_{1}-b)x_{3}+c\\
(x_{1}-b)\left(c+(x_{1}-b)x_{3}\right)-(x_{2}+x_{3})x_{3}
\end{array}\right)
\]
in die Beobachtbarkeitsnormalform~(\ref{eq:beobachtbarkeitsnormalform-keller})
bzw.~(\ref{eq:beobachtbarkeitsnormalform-keller-autonom}) mit
\[
\begin{array}{ccl}
f_{3}^{*}(\xi) & = & ac\frac{\xi_{2}}{\xi_{1}}-c\frac{\xi_{3}}{\xi_{1}}+2c\frac{\xi_{2}^{2}}{\xi_{1}^{2}}+c-b\xi_{1}+a\xi_{3}-a\frac{\xi_{2}^{2}}{\xi_{1}}\\
 &  & +\,a\xi_{1}^{2}+3\frac{\xi_{2}\xi_{3}}{\xi_{1}}-2\frac{\xi_{2}^{3}}{\xi_{1}^{2}}-\xi_{1}\xi_{2}-\xi_{2}
\end{array}
\]
überführen. Die Differentialgleichung 
\[
\lambda^{\prime\prime}(\xi_{1})=\lambda^{\prime}(\xi_{1})\,\frac{1}{n}\,\left.\frac{\partial^{2}f_{n}^{*}(\xi)}{\partial\xi_{2}\partial\xi_{n}}\right|_{\xi_{2}=\xi_{3}=0}=\frac{\lambda^{\prime}(\xi_{1})}{\xi_{1}}
\]
hat die allgemeine Lösung $\lambda(\xi_{1})=C_{0}+C_{1}\ln\xi_{1}$.
Mit der Wahl $C_{0}=0,$ $C_{1}=1$ erhält man $\lambda(\xi_{1})=\ln\xi_{1}$.
Entsprechend Gl.~(\ref{eq:Keller-berechnung-alpha}) berechnet man
die Komponenten
\[
\bar{\alpha}_{1}(\xi_{1})=\frac{1}{\xi_{1}}\left.f_{3}^{*}(\xi)\right|_{\xi_{2}=\xi_{3}=0}=-b+a\xi_{1}+\frac{c}{\xi_{1}}
\]
sowie
\[
\begin{array}{lclclcl}
\bar{\alpha}_{2}^{\prime}(\xi_{1}) & = & -\frac{1}{\xi_{1}}+\frac{ac}{\xi_{1}^{2}}-1 & \quad\Rightarrow\quad & \bar{\alpha}_{2}(\xi_{1}) & = & -\ln\xi_{1}-\frac{ac}{\xi_{1}}-\xi_{1}\\
\bar{\alpha}_{3}^{\prime}(\xi_{1}) & = & \frac{a}{\xi_{1}}-\frac{c}{\xi_{1}^{2}} & \quad\Rightarrow\quad & \bar{\alpha}_{3}(\xi_{1}) & = & a\ln\xi_{1}+\frac{c}{\xi_{1}}
\end{array}
\]
der Aufschaltung. Mit $z_{3}=\lambda(\xi_{1})=\ln\xi_{1}$ bzw. $\xi_{1}=\gamma(z_{3})=\exp z_{3}$
erhält man (bis auf konstante Anteile) die in Gl.~(\ref{eq:roessler-alpha-gamma})
angegebene Ausgangsaufschaltung.
\end{example}

\begin{remark}
Das beschriebene Berechnungsverfahren kann auch auf Systeme mit mehreren
Ausgängen übertragen werden~\cite{keller86diss,phelps1988,phelps1991}.
Ist man nicht in der Lage, Gl.~(\ref{eq:ncg}) exakt zu lösen, dann
sind verschiedene Approximationen denkbar, z.\,B. mit Taylorreihen~\cite{jelali1995}
oder Spline-Funktionen~\cite{bortoff95,lynch97scl}.
\end{remark}

\subsection{Berechnung nach Li und Tao\label{subsec:Berechnung-nach-Li-Tao}}

Die Berechnung der Ausgangstransformation ist mit Gl.~(\ref{eq:keller-berechnung-inverse-ausgangstransformation})
möglich. Dadurch kann man ohne wesentliche Einschränkung auf die Transformation~$\gamma$
verzichten, d.\,h. man setzt $\beta(\cdot)=1$ und bekommt nach Gl.~(\ref{eq:dgamma-beta})
für $\gamma$ die identische Abbildung.\footnote{Die \emph{identische Abbildung}~$\id$ ist eine Funktion, die genau
ihr Argument zurückgibt, d.\,h. $\id(x)=x$.} Das Startvektorfeld~$v$ aus Gl.~(\ref{eq:v}) stimmt in diesem
Fall mit der letzten Spalte der inversen Beobachtbarkeitsmatrix überein.
Die Form~(\ref{eq:nf}) vereinfacht sich dann zur (eigentlichen)
Beobachternormalform\index{Beobachternormalform}\index{Normalform!Beobachter-}
\begin{equation}
\dot{z}=Az+\alpha(y,u),\quad y=c^{T}z.\label{eq:nf-ohne-gamma}
\end{equation}

Die Bedingungen aus Satz~\ref{thm:existence} seien für $\beta=1$
erfüllt. Für das nichtaffine System~(\ref{eq:plant}) mit der Zerlegung~(\ref{eq:plant-fg})
gilt wegen Bedingung~\ref{enu:ki3} der Zusammenhang
\[
\begin{array}{ccl}
\ad_{-F}v(x,u) & = & -[F,v](x,u)\\
 & = & -[f,v](x)-[g,v](x,u)\\
 & = & -[f,v](x)\\
 & = & \ad_{-f}v(x).
\end{array}
\]
Die gleiche Überlegung kann man bis zur Lie-Klammer
\begin{equation}
\begin{array}{ccl}
\ad_{-F}^{n-1}v(x,u) & = & -[F,\ad_{-F}^{n-2}v](x,u)\\
 & = & -[F,\ad_{-f}^{n-2}v](x,u)\\
 & = & -[f,\ad_{-f}^{n-2}v](x)-[g,\ad_{-f}^{n-2}v](x,u)\\
 & = & -[f,\ad_{-f}^{n-2}v](x)\\
 & = & \ad_{-f}^{n-1}v(x)
\end{array}\label{eq:hilfsrechnung-ad-n-1}
\end{equation}
fortführen. Mit Gln.~(\ref{eq:pde-T}) und~(\ref{eq:PI}) erhält
man 
\[
\begin{array}{lcl}
T_{*}\ad_{-F}^{n}v(x,u) & = & T^{\prime}(x)\cdot[-F,\ad_{-F}^{n-1}v](x,u)\\
 & = & -[T^{\prime}(x)F(x,u),T^{\prime}(x)\ad_{-f}^{n-1}v(x)]\\
 & = & -[Az+\alpha(z_{n},u),\frac{\partial}{\partial z_{n}}]\\
 & = & \frac{\partial}{\partial z_{n}}\alpha(z_{n},u).
\end{array}
\]
Dadurch bekommt man die Ableitung der Aufschaltung~$\alpha$, welche
selber eine vektorwertige Funktion ist:
\begin{equation}
\begin{array}{rll}
\frac{\partial}{\partial z_{n}}\alpha(z_{n},u) & = & T^{\prime}(x)\cdot\ad_{-F}^{n}v(x,u)\\
 & = & \Pi^{-1}(x)\cdot\ad_{-F}^{n}v(x,u)\\
 & =: & \left.b(y,u)\right|_{y=h(x)}.
\end{array}\label{eq:tao-dalpha}
\end{equation}
Die komponentenweise Integration
\begin{equation}
\alpha(y,u)=\int b(y,u)\,\d y\label{eq:tao-alpha}
\end{equation}
liefert (bis auf die Integrationskonstanten) die gesuchte Eingangs-Ausgangs-Aufschaltung~\cite{li1986}.
Ist das Parameterintegral nicht geschlossen zu ermitteln, so kann
man die numerische Lösung für jede Komponente $\alpha_{i}(y,u)$ in
einem zweidimensionalen Kennfeld erfassen. Damit lässt sich der Beobachter
in der Form~(\ref{eq:ons-orig}) implementieren, wobei in dem Korrekturterm~(\ref{eq:correction-term-nf-obs})
die inverse Jacobimatrix des Diffeomorphismus~$T$ wegen Gl.~(\ref{eq:pde-T})
durch die Matrix~$\Pi$ ersetzt werden kann.

\begin{example}
\label{exa:vdp-Tao}Beim van der Pol-System~(\ref{exa:van-der-Pol-Krener-Isidori})
aus Beispiel~\ref{exa:van-der-Pol-Krener-Isidori} wurde bereits
$\beta=1$ verwendet. Die in Gl.~(\ref{eq:tao-dalpha}) benötigte
Lie-Ableitung lautet
\begin{equation}
\ad_{-F}^{2}v(x,u)=\left(\begin{array}{c}
1-x_{1}^{2}\\
x_{1}^{4}-2x_{1}^{2}+u
\end{array}\right).\label{eq:vdp-ad-n}
\end{equation}
In Verbindung mit der Matrix~$\Pi^{-1}$ aus Gl.~(\ref{eq:vdp-nam-omega})
erhält man 
\[
\frac{\partial}{\partial y}\alpha(y,u)=\left.\Pi^{-1}(x)\cdot\ad_{-F}^{2}v(x,u)\right|_{y=x_{1}}=\left.\left(\begin{array}{c}
u-1\\
1-x_{1}^{2}
\end{array}\right)\right|_{y=x_{1}}=\left(\begin{array}{c}
u-1\\
1-y^{2}
\end{array}\right)
\]
und durch Integration nach~$y$ die Eingangs-Ausgangs-Aufschaltung
\[
\alpha(y,u)=\left(\begin{array}{c}
uy-y\\
y-\frac{1}{3}y^{3}
\end{array}\right),
\]
die bereits in Beispiel~\ref{exa:van-der-Pol-Krener-Isidori} angegeben
wurde.
\end{example}

\begin{example}
\label{exa:Roessler-Tao}Für das Rössler-System~(\ref{eq:roessler-attraktor})
aus Beispiel~\ref{exa:Roessler-Krener-Isidori} verwenden wir jetzt
unter Bezug auf die im Beispiel~\ref{exa:Roessler-Tao} berechnete
Transformation~$\lambda$ den neuen Ausgang 
\begin{equation}
y=h(x)=\ln x_{3}.\label{eq:roessler-ausgang2}
\end{equation}
Mit dem neuen Ausgang erhält man zwar eine andere Beobachtbarkeitsmatrix
als mit dem Ausgang~(\ref{eq:roessler-ausgang1}), durch die gleichzeitige
Normierung der Funktion~$\beta$ jedoch den bereits bekannten Startvektor~$v$
und damit die gleichen Lie-Klammern~(\ref{eq:Roessler-Lie-Klammern})
sowie die gleiche Matrix~$\Pi$ wie in Beispiel~\ref{exa:Roessler-Krener-Isidori}.
Mit der Matrix $\Pi^{-1}$ und der Lie-Klammer
\begin{equation}
\ad_{-f}^{3}v(x)=\left(\begin{array}{c}
a^{2}-1-x_{3}\\
a(2-a^{2})\\
ax_{3}-c
\end{array}\right)\label{eq:roessler-ad-n}
\end{equation}
erhält man die Ableitung der Ausgangsaufschaltung:
\[
\frac{\partial}{\partial z_{3}}\alpha(z_{3})=\Pi^{-1}(x)\cdot\ad_{-f}^{3}v(x)=\left(\begin{array}{c}
\frac{ax_{3}^{2}-c}{x_{3}}\\
\frac{ac-x_{3}-x_{3}^{2}}{x_{3}}\\
\frac{ax_{3}-c}{x_{3}}
\end{array}\right).
\]
Ersetzt man~$x_{3}$ gemäß Gl.~(\ref{eq:roessler-ausgang2}) durch
den Ausgang (also $x_{3}=\exp y$), dann liefert die Integration nach~$y$
die Aufschaltung
\begin{equation}
\alpha(y)=\left(\begin{array}{c}
a\e^{y}+c\e^{-y}\\
-\e^{y}-ac\e^{-y}-y\\
c\e^{-y}+ay
\end{array}\right).\label{eq:roessler-alpha-y2}
\end{equation}
Dieses Ergebnis stimmt bis auf die Integrationskonstanten mit der
Aufschaltung~(\ref{eq:roessler-alpha-gamma}) aus Beispiel~\ref{exa:Roessler-Krener-Isidori}
überein. Die Berechnung kann leicht mit \textsc{Maxima} nachvollzogen
werden:

\begin{maxima}\noindent
%%%%%%%%%%%%%%%
%%% INPUT:
\begin{minipage}[t]{8ex}\color{red}\bf
\begin{verbatim}
(%i5) 
\end{verbatim}
\end{minipage}
\begin{minipage}[t]{\textwidth}\color{blue}
\begin{verbatim}
f:[-x2-x3,x1+a*x2,c+x3*(x1-b)]$
h:log(x3)$
x:[x1,x2,x3]$
n:length(x)$
\end{verbatim}
\end{minipage}

\smallskip

\noindent
%%%%%%%%%%%%%%%
%%% INPUT:
\begin{minipage}[t]{8ex}\color{red}\bf
\begin{verbatim}
(%i9) 
\end{verbatim}
\end{minipage}
\begin{minipage}[t]{\textwidth}\color{blue}
\begin{verbatim}
Q:ObservabilityMatrix(f,h,x)$
v:list_matrix_entries(col(invert(Q),n))$
P:ControllabilityMatrix(f,v,x)$
dα:ratsimp(invert(P).LieBracket(-f,v,x,3));
\end{verbatim}
\end{minipage}
%%% OUTPUT:

\noindent
$\displaystyle
\parbox{8ex}{$\color{labelcolor}\mathrm{\tt (\%o9) }\quad $}
\begin{pmatrix}\frac{a\cdot {{\mathit{x3}}^{2}}-c}{\mathit{x3}}\cr -\frac{-a\cdot c+\mathit{x3}+{{\mathit{x3}}^{2}}}{\mathit{x3}}\cr \frac{a\cdot \mathit{x3}-c}{\mathit{x3}}\end{pmatrix}\mbox{}
$
%%%%%%%%%%%%%%%


\noindent
%%%%%%%%%%%%%%%
%%% INPUT:
\begin{minipage}[t]{8ex}\color{red}\bf
\begin{verbatim}
(%i11) 
\end{verbatim}
\end{minipage}
\begin{minipage}[t]{\textwidth}\color{blue}
\begin{verbatim}
subst(x3=exp(y),dα)$
integrate(%,y);
\end{verbatim}
\end{minipage}
%%% OUTPUT:

\noindent
$\displaystyle
\parbox{8ex}{$\color{labelcolor}\mathrm{\tt (\%o11) }\quad $}
\begin{pmatrix}a\cdot {{e}^{y}}+c\cdot {{e}^{-y}}\cr -{{e}^{y}}-a\cdot c\cdot {{e}^{-y}}-y\cr c\cdot {{e}^{-y}}+a\cdot y\end{pmatrix}\mbox{}
$
%%%%%%%%%%%%%%%
\end{maxima}

Mit der Aufschaltung~(\ref{eq:roessler-alpha-y2}) kann man den Korrekturterm~(\ref{eq:correction-term-nf-obs})
berechnen. Der Beobachter~(\ref{eq:ons-orig}) hat dann die Form
\[
\left(\begin{array}{c}
\dot{\hat{x}}_{1}\\
\dot{\hat{x}}_{2}\\
\dot{\hat{x}}_{3}
\end{array}\right)\!\!=\!\!\left(\begin{array}{c}
-\hat{x}_{2}-\hat{x}_{3}\\
\hat{x}_{1}+a\hat{x}_{2}\\
c+\hat{x}_{3}(\hat{x}_{1}-b)
\end{array}\right)\!+\!\left(\begin{array}{c}
\hat{x}_{3}-\e^{y}+(y-\ln\hat{x}_{3})(a_{1}+aa_{2}+a^{2}-1)\\
\!(y-\ln\hat{x}_{3})(2a-a_{0}+a_{2}-aa_{1}-a^{3}-a^{2}a_{2})\!\\
c\hat{x}_{3}(\e^{-y}-1)+(y-\ln\hat{x}_{3})\hat{x}_{3}(a+a_{2})
\end{array}\right)\!.
\]
\end{example}

\section{Taylorapproximation\label{sec:Approximation-of-observers}}

In diesem Abschnitt wird eine Klasse von Beobachtern vorgestellt,
für deren Auslegung weder die direkte Berechnung der Koordinatentransformation~(\ref{eq:TS})
noch der Normalform~(\ref{eq:nf}) erforderlich ist. Diese Beobachter
basieren auf verschiedenen Approximationen der Fehlerdynamik~(\ref{eq:error-dgl-nf}).
Zur Vereinfachung der Darstellung wird auf eine zusätzliche Ausgangstransformation
verzichtet, d.\,h. man setzt $\beta(\cdot)=1$ und bekommt damit
für~$\gamma$ die identische Abbildung (vgl. Abschnitt~\ref{subsec:Berechnung-nach-Li-Tao}).
Man erhält dadurch die eigentliche Beobachter-Normalform~(\ref{eq:nf-ohne-gamma}).

\subsection{Approximation nullter Ordnung mit einem High-Gain-Beobachter\label{subsec:Approximation-nullter-Ordnung}}

Der Beobachter 
\begin{equation}
\dot{\hat{x}}=F(\hat{x},u)+k_{0}(\hat{x})(y-h(\hat{x}))\label{eq:highgain}
\end{equation}
mit der Beobachterverstärkung $k_{0}:\mathcal{M}\to\R^{n}$ stellt
eine Erweiterung der klassischen Luenberger-Struktur dar. Die Existenzbedingungen
von Satz~\ref{thm:existence} seien erfüllt. Dann existiert ein Diffeomorphismus~(\ref{eq:TS}),
der das System~(\ref{eq:plant}) in die Beobachter-Normalform~(\ref{eq:nf-ohne-gamma})
überführt. Wendet man diese Transformation auf den Beobachter~(\ref{eq:highgain})
an, so erhält man 
\[
\dot{\hat{z}}=A\hat{z}+\alpha(\hat{z}_{n},u)+\left(S^{\prime}(\hat{z})\right)^{-1}k_{0}(S(\hat{z}))\,(z_{n}-\hat{z}_{n}).
\]
Der Beobachtungsfehler $\tilde{z}=z-\hat{z}$ genügt der Differentialgleichung
\begin{equation}
\dot{\tilde{z}}=A\tilde{z}+\alpha(z_{n},u)-\alpha(\hat{z}_{n},u)-\left(S^{\prime}(\hat{z})\right)^{-1}k_{0}(S(\hat{z}))\,\tilde{z}_{n}.\label{eq:error-high-gain1}
\end{equation}
Mit dem Vektorfeld 
\begin{equation}
k_{0}(\hat{x})=k_{0}(S(\hat{z}))=S^{\prime}(\hat{z})\,l,\label{eq:gain-hg-z}
\end{equation}
bekommt die Fehlerdynamik~(\ref{eq:error-high-gain1}) die Form 
\begin{equation}
\dot{\tilde{z}}=(A-lc^{T})\tilde{z}+\alpha(z_{n},u)-\alpha(\hat{z}_{n},u).\label{eq:error-high-gain-typical}
\end{equation}
Durch geeignete Wahl des Vektors $l\in\R^{n}$ kann man dem linearen
Teil $A-lc^{T}$ von~(\ref{eq:error-high-gain-typical}) ein vorgegebenes
charakteristisches Polynom~(\ref{eq:cp}) zuweisen. Die Fehlerdynamik~(\ref{eq:error-high-gain-typical})
besitzt die für High-Gain-Beobachter typische Form, vgl.~\cite{thau1973,raghavan1994,rajamani1998}
und Kapitel~\ref{chap:High-Gain-Beobachter}. Die in~(\ref{eq:error-high-gain-typical})
auftretenden Nichtlinearitäten lassen sich als Taylorentwicklung nullter
Ordnung der Ausgangsaufschaltung $\alpha(z_{n},u)$ bezüglich~$\tilde{z}_{n}$
auffassen, d.\,h. 
\[
\tilde{z}=(A-lc^{T})\tilde{z}+\mathcal{O}(\|\tilde{z}\|).
\]

Die unterschiedlichen Signalflüsse bei dem Normalform-Beobachter und
dem High-Gain-Beobachter~(\ref{eq:highgain}) werden aus dem in Abb.~\ref{fig:Wirkungsschemata-NF-und-High-Gain-Beobachter}
dargestellten Wirkungsschema deutlich.

\begin{figure}
\begin{centering}
\resizebox{0.75\textwidth}{!}{\input{bild-nf-high-gain1.pdftex_t}}
\par\end{centering}
\caption{Wirkungsschema für Normalform- und High-Gain-Beobachter\label{fig:Wirkungsschemata-NF-und-High-Gain-Beobachter}}
\end{figure}

Mit der Umkehrabbildung~$T$ von~$S$ kann man die partielle Differentialgleichung~(\ref{eq:pde-T})
auch gleichwertig in der Form 

\begin{equation}
S^{\prime}(z)=\left.\Pi(x)\right|_{x=S(z)}\label{eq:pde}
\end{equation}
angeben. Auf der Basis von~(\ref{eq:pde}) lässt sich die Beobachterverstärkung~(\ref{eq:gain-hg-z})
in den Originalkoordinaten angeben:
\begin{equation}
k_{0}(\hat{x})=\Pi(\hat{x})\,l=a_{0}v(\hat{x})+a_{1}\ad_{-f}v(\hat{x})+\cdots+a_{n-1}\ad_{-f}^{n-1}v(\hat{x}).\label{eq:gain-hg-x}
\end{equation}

\begin{example}
\label{exa:vdp-NF-high-gain}Beim van der Pol-System~(\ref{eq:vdp})
aus Beispiel~\ref{exa:van-der-Pol-Krener-Isidori} erhält man aus
den Lie-Klammern~(\ref{eq:vdp-Lie-Klammern2}) für den High-Gain-Beobachter~(\ref{eq:highgain})
die Beobachterverstärkung 
\begin{equation}
k_{0}(\hat{x})=a_{0}v(\hat{x})+a_{1}\ad_{-f}v(\hat{x})=\left(\begin{array}{c}
a_{1}\\
a_{0}+a_{1}(1-\hat{x}_{1}^{2})
\end{array}\right)\label{eq:vdp-k0}
\end{equation}
mit den Koeffizienten $a_{0},a_{1}>0$ des charakteristischen Polynoms~(\ref{eq:cp}).
\end{example}

\begin{example}
\label{exa:roessler-NF-high-gain}Im Fall des Rössler-Systems~(\ref{eq:roessler-attraktor})
aus Beispiel~\ref{exa:Roessler-Krener-Isidori} mit dem Ausgang~(\ref{eq:roessler-ausgang2})
aus Beispiel~\ref{exa:Roessler-Tao} berechnet man entsprechend Gl.~(\ref{eq:gain-hg-x})
die Beobachterverstärkung 
\begin{equation}
k_{0}(\hat{x})=\left(\begin{array}{c}
aa_{2}+a_{1}\\
(1-a^{2})a_{2}-aa_{1}-a_{0}\\
a_{2}\hat{x}_{3}
\end{array}\right).\label{eq:roessler-k0}
\end{equation}
Zur Berechnung dieser Verstärkung muss das in Beispiel~\ref{exa:Roessler-Tao}
verwendete \textsc{Maxima}-Skript nur leicht modifiziert werden:

\begin{maxima}\noindent
%%%%%%%%%%%%%%%
%%% INPUT:
\begin{minipage}[t]{8ex}\color{red}\bf
\begin{verbatim}
(%i5) 
\end{verbatim}
\end{minipage}
\begin{minipage}[t]{\textwidth}\color{blue}
\begin{verbatim}
f:[-x2-x3,x1+a*x2,c+x3*(x1-b)]$
h:log(x3)$
x:[x1,x2,x3]$
n:length(x)$
\end{verbatim}
\end{minipage}

\smallskip

\noindent
%%%%%%%%%%%%%%%
%%% INPUT:
\begin{minipage}[t]{8ex}\color{red}\bf
\begin{verbatim}
(%i9) 
\end{verbatim}
\end{minipage}
\begin{minipage}[t]{\textwidth}\color{blue}
\begin{verbatim}
Q:ObservabilityMatrix(f,h,x)$
v:list_matrix_entries(col(invert(Q),n))$
P:ControllabilityMatrix(f,v,x)$
k0:P.[a0,a1,a2];
\end{verbatim}
\end{minipage}
%%% OUTPUT:

\noindent
$\displaystyle
\parbox{8ex}{$\color{labelcolor}\mathrm{\tt (\%o9) }\quad $}
\begin{pmatrix}a\cdot \mathit{a2}+\mathit{a1}\cr \left( 1-{{a}^{2}}\right) \cdot \mathit{a2}-a\cdot \mathit{a1}-\mathit{a0}\cr \mathit{a2}\cdot \mathit{x3}\end{pmatrix}\mbox{}
$
%%%%%%%%%%%%%%%
\end{maxima}
\end{example}

im Folgenden wird gezeigt, dass der Beobachter~(\ref{eq:nf-ohne-gamma})
global konvergiert, sofern die Eingangs-Ausgangs-Aufschaltung einer
globalen Lipschitz-Bedingung\index{Lipschitz-Bedingung} bezüglich~$y$
gleichmäßig in~$u$ genügt, d.\,h. wenn es eine Konstante $\rho>0$
mit 
\begin{equation}
\forall y,\hat{y}\in\mathcal{Y}\,\forall u\in\mathcal{U}:\quad\left\Vert \alpha(y,u)-\alpha(\hat{y},u)\right\Vert \leq\rho\left\Vert y-\hat{y}\right\Vert \label{eq:Lipschitz-Bed-Ausgangsaufschaltung}
\end{equation}
gibt, wobei $\mathcal{Y}=\{h(x);\,x\in\mathcal{M}\}$.
\begin{theorem}
\label{thm:high-gain} Gegeben sei ein System~(\ref{eq:plant}) auf
$\mathcal{M}=\R^{n}$ zusammen mit dem Beobachter~(\ref{eq:highgain})
und der Beobachterverstärkung~(\ref{eq:gain-hg-x}). Die Bedingungen~\ref{enu:ki1}-\ref{enu:ki4}
aus Satz~\ref{thm:existence} seien für $\beta(\cdot)=1$ erfüllt.
Außerdem genüge die Eingangs-Ausgangs-Aufschaltung~$\alpha$ der
globalen Lipschitz-Bedingung~(\ref{eq:Lipschitz-Bed-Ausgangsaufschaltung}).
Dann existiert ein Vektor $l\in\R^{n}$ derart, dass 
\begin{equation}
\lim_{t\to\infty}\|\hat{x}(t)-x(t)\|=0\label{eq:nf-fehlerdgl-hg}
\end{equation}
 für alle Anfangswerte $x(0),\hat{x}(0)\in\mathcal{M}$ gilt. 
\end{theorem}

Der Satz wird in~\cite{roebenack2004nolcos,roebenack2007iet} für
eine spezielle Eigenwertlage ($n$-facher Pol in der linken Halbebene)
bewiesen. Der hier vorgestellte Beweis beruht auf einer allgemeineren
Variante der Eigenwertvorgabe~\cite[Abschn.~{IV.A}]{roebenack2012ssd}: 
\begin{svmultproof2}
Sind die Bedingungen~\ref{enu:ki1}-\ref{enu:ki4} aus Satz~\ref{thm:existence}
erfüllt, so existiert eine globale Transformation in die Beobachternormalform~(\ref{eq:nf-ohne-gamma}).
Die Konvergenzanalyse kann daher in der Beobachternormalform durchgeführt
werden. Wir zeigen, dass die Ruhelage $\tilde{z}=0$ der Fehlerdifferentialgleichung~(\ref{eq:error-high-gain-typical})
global asymp\-to\-tisch stabil ist. Die in~(\ref{eq:error-high-gain-typical})
auftretende Nichtlinearität
\begin{equation}
\Delta\alpha(\tilde{z},t):=\alpha(z_{n},u)-\alpha(\hat{z}_{n},u)\label{eq:beweis-nf-hg-nichtlinearitaet}
\end{equation}
genügt laut Voraussetzung einer globalen Lipschitz-Bedingung~(\ref{eq:Lipschitz-Bed-Ausgangsaufschaltung})
mit der Lipschitz-Konstanten $\rho>0$. Die Fehlerdynamik lässt sich
dann als Verknüpfung zwischen der zeitvarianten Nichtlinearität~(\ref{eq:beweis-nf-hg-nichtlinearitaet})
und einem linearen zeit\-invarianten System mit der Übertragungsfunktion
\begin{equation}
G(s)=c^{T}\left(sI-\left(A-lc^{T}\right)\right)^{-1}\label{eq:bweis-nf-hg-G}
\end{equation}
darstellen (siehe Abb.~\ref{fig:Fehlerdynamik-NF-HG}). 
\begin{figure}
\begin{centering}
\resizebox{0.75\textwidth}{!}{\input{bild-nf-high-error.pdftex_t}}
\par\end{centering}
\caption{Struktur der Fehlerdynamik~(\ref{eq:error-high-gain-typical}) des
High-Gain-Beobachters~(\ref{eq:highgain})\label{fig:Fehlerdynamik-NF-HG}}
\end{figure}

Für den linearen Teil gibt man zunächst (reelle oder paarweise konjugiert
komplexe) Eigen\-werte $s_{1},\ldots,s_{n}$ mit negativem Realteil
vor. Daraus erhält man durch Ausmultiplizieren 
\begin{equation}
\CP(s)=(s-s_{1})\cdots(s-s_{n})=a_{0}+a_{1}s+\cdots+a_{n-1}s^{n-1}+s^{n}\label{eq:cp-eigenwerte}
\end{equation}
der zugehörigen Linearfaktoren die Koeffizienten $a_{0},\ldots,a_{n-1}$
des charakteristischen Polynoms~(\ref{eq:cp}). Darauf aufbauend
führen wir die Beobachterverstärkung 
\begin{equation}
l^{\epsilon}=\left(\begin{array}{c}
a_{0}/\epsilon^{n}\\
a_{1}/\epsilon^{n-1}\\
\vdots\\
a_{n-1}/\epsilon
\end{array}\right)\label{eq:nf-verstaerkung-hg-eps}
\end{equation}
mit dem Parameter $\epsilon\in(0,1]$ ein. Für $\epsilon=1$ erhält
man das charakteristische Polynom~(\ref{eq:cp-eigenwerte}), für
$\epsilon<1$ dagegen
\[
\begin{array}{lrl}
\CP^{\epsilon}(s) & := & \det(sI-A+l^{\epsilon}c^{T})\\
 & = & \frac{a_{0}}{\epsilon^{n}}+\frac{a_{1}}{\epsilon^{n-1}}s+\cdots+\frac{a_{n-1}}{\epsilon}s^{n-1}+s^{n}\\
 & = & \left(s-\frac{s_{1}}{\epsilon}\right)\cdots\left(s-\frac{s_{n}}{\epsilon}\right),
\end{array}
\]
so dass der Parameter~$\epsilon$ für $\epsilon\to0$ eine Streckung
der Eigenwerte in Richtung~$-\infty$ (also nach links in der komplexen
Halbebene) bewirkt. Ersetzt man in der Übertragungsfunktion~(\ref{eq:bweis-nf-hg-G})
die Verstärkung~$l$ durch~$l^{\epsilon}$ aus Gl.~(\ref{eq:nf-verstaerkung-hg-eps}),
so erhält man die in~$\epsilon$ parametrierte Übertragungsfunktion
$G^{\epsilon}(s)$ mit $G^{1}(s)=G(s)$. Mit der Form~(\ref{eq:AC})
ergibt sich die Übertragungsfunktion
\begin{equation}
G^{\epsilon}(s)=\frac{1}{\CP^{\epsilon}(s)}\left(\begin{array}{cccc}
1 & s & \cdots & s^{n-1}\end{array}\right).\label{eq:bweis-nf-hg-G-eps}
\end{equation}
Für die $k$-te Komponente des Frequenzgangs~$G^{\epsilon}(j\omega)$
dieser Übertragungsfunktion gilt
\[
\begin{array}{ccl}
G_{k}^{\epsilon}(j\omega) & = & \frac{\left(j\omega\right)^{k-1}}{\left(j\omega-\frac{s_{1}}{\epsilon}\right)\cdots\left(j\omega-\frac{s_{n}}{\epsilon}\right)}\\
 & = & \frac{\left(j\omega\right)^{k-1}\epsilon^{n}}{\left(j\omega\epsilon-s_{1}\right)\cdots\left(j\omega\epsilon-s_{n}\right)}\\
 & = & \frac{\left(j\omega\epsilon\right)^{k-1}\epsilon^{n-k+1}}{\left(j\omega\epsilon-s_{1}\right)\cdots\left(j\omega\epsilon-s_{n}\right)}\\
 & = & \frac{\left(j\tilde{\omega}\right)^{k-1}\epsilon^{n-k+1}}{\left(j\tilde{\omega}-s_{1}\right)\cdots\left(j\tilde{\omega}-s_{n}\right)}\\
 & = & G(j\tilde{\omega})\epsilon^{n-k+1}
\end{array}
\]
mit $\tilde{\omega}=\omega\epsilon$. Bei Vorgabe eines stabilen charakteristischen
Polynoms~(\ref{eq:cp-eigenwerte}) gilt $\|G\|_{\infty}=\|G^{1}\|_{\infty}<\infty$.
Aus
\[
\|G_{k}^{\epsilon}\|_{\infty}=\sup_{\tilde{\omega}}|G_{k}^{\epsilon}(j\tilde{\omega})|=\sup_{\omega}|G_{k}(j\omega)|\epsilon^{n-k+1}=\|G_{k}\|_{\infty}\epsilon^{n-k+1}\to0
\]
für $\epsilon\to0$ folgt unmittelbar $\|G^{\epsilon}\|_{\infty}\to0$
für $\epsilon\to0$. Bei einer Lipschitz-Konstanten $\rho>0$ lässt
sich der Parameter~$\epsilon$ immer derart wählen, dass $\|G^{\epsilon}\|_{\infty}<1/\rho$,
so dass die Verstärkung des offenen Kreises kleiner als Eins ist.
Nach dem Satz von der kleinen Kreisverstärkung\index{Satz!von der kleinen Kreisverstärkung}
ist daher die Ruhelage $\tilde{z}=0$ des Gesamtsystems global asymptotisch
stabil (siehe Anhang~\ref{sec:Stabilitaet-im-geschlossenen-Regelkreis}).
\end{svmultproof2}

\begin{remark}
Der Beobachter~(\ref{eq:highgain}) mit der Verstärkung~(\ref{eq:gain-hg-x})
ist in gewisser Weise das Gegenstück zu dem in~\cite{gauthier92,ciccarella93,dalla-mora1997}
vorgeschlagenen High-Gain-Beobachter (vgl. Abschnitt~\ref{sec:High-Gain-Entwurf-Beobachtbarkeitsnormalform}),
wobei hier die Beobachternormalform anstelle der Beobachtbarkeitsnormalform
verwendet wird. Beide Beobachter erfordern eine reguläre Beobachtbarkeitsmatrix,
was der Bedingung~\ref{enu:ki1} aus Satz~\ref{thm:existence} entspricht.
Außerdem weisen beide Beobachter strukturelle Einschränkungen hinsichtlich
des Eingangs auf. Der Bedingung~\ref{enu:ki3} stehen die in~\cite{gauthier81}
und~\cite[Sect.~{III}]{gauthier92} angegebenen Bedingungen gegenüber.
Hinsichtlich der Existenz der jeweiligen Normalform weist der Beobachter
in~\cite{gauthier92} schwächere Existenzbedingungen auf, da dort
auf die Bedingung~\ref{enu:ki2} verzichtet wird. Der Beobachter~(\ref{eq:highgain})
mit~(\ref{eq:gain-hg-x}) ist dann besonders sinnvoll, wenn das System
in der Beobachternormalform einer Lipschitz-Bedingung genügt, nicht
aber in der Beobachtbarkeitsnormalform. Das wird in Abschnitt~\ref{subsec:Berechnung-nach-Keller}
deutlich, wo in der Beobachtbarkeitsnormalform~(\ref{eq:beobachtbarkeitsnormalform-keller})
u.\,a. Terme~$\xi_{2}^{2},\xi_{2}\xi_{3},\xi_{2}^{3}$ usw. auftreten,
die keiner globalen Lipschitz-Bedingung genügen (siehe Gl.~(\ref{eq:keller-f2})).
\end{remark}

\subsection{Approximation erster Ordnung mit einem erweiterten Luen\-berger-Beobachter\label{subsec:Luenberger}}

Die folgende Herleitung basiert unmittelbar auf~\cite{bestle83,zeitz87}.
Wir betrachten einen Beobachter
\begin{equation}
\dot{\hat{x}}=F(\hat{x},u)+k_{1}(\hat{x},u)(y-h(\hat{x}))\label{eq:luenberger}
\end{equation}
mit Luenberger-Struktur und der Beobachterverstärkung $k_{1}:\mathcal{M}\times\mathcal{U}\to\R^{n}$,
welche jetzt auch vom Eingang~$u$ abhängen kann. Unter den Bedingungen
von Satz~\ref{thm:existence} für $\beta(\cdot)=1$ existiert die
Transformation~(\ref{eq:TS}) in die Beobachternormalform~(\ref{eq:nf-ohne-gamma}).
Diese Transformation wird auf den Beobachter~(\ref{eq:luenberger})
angewendet:
\[
\dot{\hat{z}}=A\hat{z}+\alpha(\hat{z}_{n},u)+\left(S^{\prime}(\hat{z})\right)^{-1}k_{1}(S(\hat{z}),u)\,(z_{n}-\hat{z}_{n}).
\]
In den transformierten Koordinaten erhält man die Fehlerdifferentialgleichung
\begin{equation}
\dot{\tilde{z}}=A\tilde{z}+\alpha(z_{n},u)-\alpha(\hat{z}_{n},u)-\left(S^{\prime}(\hat{z})\right)^{-1}k_{1}(S(\hat{z}),u)\,\tilde{z}_{n}.\label{eq:error-luen1}
\end{equation}

Ohne explizite Kenntnis der Transformation~(\ref{eq:TS}) muss man
die Abbildung $\alpha(z_{n},u)$ als unbekannt ansehen, obwohl $z_{n}=y$
und~$u$ gemessen werden. Daher wird $\alpha(z_{n},u)$ entlang der
Schätztrajektorie~$\hat{z}$ des Beobachters~(\ref{eq:luenberger})
linearisiert: 
\[
\alpha(z_{n},u)=\alpha(\hat{z}_{n},u)+\frac{\partial}{\partial\hat{z}_{n}}\alpha(\hat{z}_{n},u)\tilde{z}_{n}+\mathcal{O}(|\tilde{z}_{n}|^{2}).
\]
Die sich daraus ergebende Linearisierung der Fehlerdifferentialgleichung~(\ref{eq:error-luen1})
hat die Form 
\begin{equation}
\dot{\tilde{z}}=A\tilde{z}-\left(\left(S^{\prime}(\hat{z})\right)^{-1}k_{1}(S(\hat{z}),u)-\frac{\partial}{\partial\hat{z}_{n}}\alpha(\hat{z}_{n},u)\right)c^{T}\tilde{z}+\mathcal{O}(\|\tilde{z}\|^{2}).\label{eq:error-luen-linearized}
\end{equation}
Die Beobachterverstärkung~$k_{1}$ wird so gewählt, dass die Jacobimatrix
der linearisierten Fehlerdifferentialgleichung~(\ref{eq:error-luen-linearized})
das charakteristische Polynom~(\ref{eq:cp}) besitzt, und zwar unabhängig
vom jeweiligen Arbeitspunkt. Setzt man 
\begin{equation}
k_{1}(\hat{x},u)=k_{1}(S(\hat{z}),u)=S^{\prime}(\hat{z})\,l+S^{\prime}(\hat{z})\,\frac{\partial}{\partial\hat{z}_{n}}\alpha(\hat{z}_{n},u),\label{eq:k1z}
\end{equation}
 so erzielt man mit der Fehlerdifferentialgleichung 
\begin{equation}
\dot{\tilde{z}}=(A-lc^{T})\,\tilde{z}+\mathcal{O}(\|\tilde{z}\|^{2})\label{eq:error-dgl-ordnung-1}
\end{equation}
eine approximative Linearisierung des Beobachtungsfehlers.\index{Linearisierung!approximative}
Dieses Vorgehen ähnelt der in~\cite{baumann86} beschriebenen \emph{erweiterten
Taylor-Lineariserung}.\index{Linearisierung!erweiterte}

im Folgenden soll die Beobachterverstärkung~(\ref{eq:k1z}) in den
Originalkoordinaten ausgedrückt werden. Dazu betrachte von dem System
partieller Differentialgleichungen~(\ref{eq:pde-T}) die einzelnen
Spalten 
\[
T_{*}\ad_{-f}^{i}v=\frac{\partial}{\partial z_{i+1}}\quad\textrm{für}\quad i=0,\ldots,n-1.
\]
Mit Gl.~(\ref{eq:hilfsrechnung-ad-n-1}) erhält man 
\[
\begin{array}{ccl}
T_{*}\ad_{-F}^{n}v(\hat{x}) & = & T_{*}\ad_{-f}^{n}v(\hat{x})+T_{*}\ad_{-g}\ad_{-f}^{n-1}v(\hat{x})\\
 & = & T_{*}[-(f+g),\ad_{-f}^{n-1}v](\hat{x},u)\\
 & = & [-A\hat{z}+\alpha(\hat{z}_{n},u),\frac{\partial}{\partial\hat{z}_{n}}](\hat{z},u)\\
 & = & \frac{\partial}{\partial\hat{z}_{n}}\alpha(\hat{z}_{n},u)
\end{array}
\]
und
\begin{equation}
\ad_{-F}^{n}v(\hat{x},u)=\ad_{-f}^{n}v(\hat{x})+\ad_{-g}\ad_{-f}^{n-1}v(\hat{x},u)=S^{\prime}(\hat{z})\,\frac{\partial}{\partial\hat{z}_{n}}\alpha(\hat{z}_{n},u).\label{eq:ad-n}
\end{equation}
Zusammen mit~(\ref{eq:gain-hg-z}) und~(\ref{eq:gain-hg-x}) ergibt
sich die Beobachterverstärkung 
\begin{equation}
\begin{array}{lcl}
k_{1}(\hat{x},u) & = & p_{0}v(\hat{x})+p_{1}\ad_{-f}v(\hat{x})+\cdots+p_{n-1}\ad_{-f}^{n-1}v(\hat{x})\\
 &  & +\,\ad_{-f}^{n}v(\hat{x})+\ad_{-g}\ad_{-f}^{n-1}v(\hat{x},u).
\end{array}\label{eq:ackermann}
\end{equation}
Die Abhängigkeit der Beobachterverstärkung~(\ref{eq:ackermann})
vom Eingang~$u$ resultiert aus dem eingangsabhängigen Vektorfeld~$g$.
Gl.~(\ref{eq:ackermann}) lässt sich als eine Verallgemeinerung der
Ackermann-Formel~\cite{ackermann77} für nichtlineare Systeme auffassen~\cite{bestle83,zeitz87}.
Der Beobachter~(\ref{eq:luenberger}) mit der Verstärkung~(\ref{eq:ackermann})
heißt \emph{erweiterter Luenberger-Beobachter}\index{Beobachter!erweiterter Luenberger-}
(engl. \emph{extended Luenberger observer}).

Der hier vorgestellte Ansatz unterscheidet sich geringfügig von~\cite{zeitz87}:
Obwohl das \hbox{System}~(\ref{eq:plant}) ggf. über den Eingang~$u$
beeinflusst werden kann, hängt die Transformation~(\ref{eq:TS})
nur vom Zustand~$x$ ab. Dieser Zugang ist aufgrund von Bedingung~\ref{enu:ki3}
aus Satz~\ref{thm:existence} möglich. Dadurch hängt die Beobachterverstärkung~(\ref{eq:ackermann})
auch nicht von Zeitableitungen des Eingangs~$u$ ab.

\medskip{}

\begin{example}
\label{exa:vdp-NF-ELO}Man betrachte das van der Pol-System~(\ref{eq:vdp})
aus Beispiel~\ref{exa:van-der-Pol-Krener-Isidori}. Die Verstärkung~(\ref{eq:ackermann})
des erweiterten Luenberger-Beobachters~(\ref{eq:luenberger}) erhält
man aus der Verstärkung~(\ref{eq:vdp-k0}) des High-Gain-Beobachters
und der Lie-Klammer~(\ref{eq:vdp-ad-n}): 
\begin{equation}
k_{1}(\hat{x},u)=\left(\begin{array}{c}
a_{1}-\hat{x}_{1}+1\\
a_{0}+a_{1}(1-\hat{x}_{1}^{2})+\hat{x}_{1}^{4}-2\hat{x}_{1}^{2}+u
\end{array}\right).\label{eq:k1-vdp}
\end{equation}
\end{example}

\begin{example}
\label{exa:roessler-NF-ELO}Beim Rössler-System~(\ref{eq:roessler-attraktor})
aus Beispiel~\ref{exa:Roessler-Krener-Isidori} mit dem Ausgang~(\ref{eq:roessler-ausgang2})
aus Beispiel~\ref{exa:Roessler-Tao} erhält man für den erweiterten
Luenberger-Beobachter~(\ref{eq:luenberger}) die Beobachterverstärkung
\[
k_{1}(\hat{x})=\left(\begin{array}{c}
a\,a_{2}+a_{1}\\
(1-a^{2})a_{2}-a\,a_{1}-a_{0}\\
a_{2}\hat{x}_{3}
\end{array}\right)+\left(\begin{array}{c}
a^{2}-1-\hat{x}_{3}\\
a(2-a^{2})\\
a\hat{x}_{3}-c
\end{array}\right),
\]
die sich aus Gln.~(\ref{eq:roessler-ad-n}) und~(\ref{eq:roessler-k0})
ergibt. Die Berechnung dieser Verstärkung mit \textsc{Maxima} erfolgt
ähnlich wie in Beispiel~\ref{exa:roessler-NF-high-gain}:

\begin{maxima}\noindent
%%%%%%%%%%%%%%%
%%% INPUT:
\begin{minipage}[t]{8ex}\color{red}\bf
\begin{verbatim}
(%i5) 
\end{verbatim}
\end{minipage}
\begin{minipage}[t]{\textwidth}\color{blue}
\begin{verbatim}
f:[-x2-x3,x1+a*x2,c+x3*(x1-b)]$
h:log(x3)$
x:[x1,x2,x3]$
n:length(x)$
\end{verbatim}
\end{minipage}

\smallskip

\noindent
%%%%%%%%%%%%%%%
%%% INPUT:
\begin{minipage}[t]{8ex}\color{red}\bf
\begin{verbatim}
(%i10) 
\end{verbatim}
\end{minipage}
\begin{minipage}[t]{\textwidth}\color{blue}
\begin{verbatim}
Q:ObservabilityMatrix(f,h,x)$
v:list_matrix_entries(col(invert(Q),n))$
cp:a0+a1*s+a2*s^2+s^3$
k1:sum(coeff(cp,s,i)*LieBracket(-f,v,x,i),i,0,n)$
transpose(ratsimp(k1));
\end{verbatim}
\end{minipage}
%%% OUTPUT:

\noindent
$\displaystyle
\parbox{8ex}{$\color{labelcolor}\mathrm{\tt (\%o10) }\quad $}
\begin{pmatrix}-\mathit{x3}+a\cdot \mathit{a2}+\mathit{a1}+{{a}^{2}}-1\cr \left( 1-{{a}^{2}}\right) \cdot \mathit{a2}-a\cdot \mathit{a1}-\mathit{a0}-{{a}^{3}}+2\cdot a\cr \left( a+\mathit{a2}\right) \cdot \mathit{x3}-c\end{pmatrix}\mbox{}
$
%%%%%%%%%%%%%%%
\end{maxima}
\end{example}

Aufgrund seiner Herleitung mittels Linearisierung wird der erweiterte
Luenberger-Beobachter oft als ein rein lokal konvergierender Beobachter
verstanden. Ähnlich wie bei Satz~\ref{thm:high-gain} kann man zeigen,
dass der erweiterte Luenberger-Beobachter~(\ref{eq:luenberger})
auch ein globaler Beobachter ist, sofern die Eingangs-Ausgangs-Aufschaltung~$\alpha$
einer globalen Lipschitz-Bedingung~(\ref{eq:Lipschitz-Bed-Ausgangsaufschaltung})
genügt~\cite{roebenack2007iet}:
\begin{theorem}
\label{thm:elo-global} Gegeben sei ein System~(\ref{eq:plant})
auf $\mathcal{M}=\R^{n}$ zusammen mit dem erweiterten Luenberger-Beobachter~(\ref{eq:luenberger})
und der Beobachterverstärkung~(\ref{eq:ackermann}). Die Bedingungen
\ref{enu:ki1}-\ref{enu:ki4} aus Satz~\ref{thm:existence} seien
für $\beta(\cdot)=1$ erfüllt. Außerdem genüge die Eingangs-Ausgangs-Aufschaltung~$\alpha$
der globalen Lipschitzbedingung~(\ref{eq:Lipschitz-Bed-Ausgangsaufschaltung}).
Dann existiert ein Vektor $l\in\R^{n}$ derart, dass 
\[
\lim_{t\to\infty}\|\hat{x}(t)-x(t)\|=0
\]
 für alle Anfangswerte $x(0),\hat{x}(0)\in\mathcal{M}$ gilt. 
\end{theorem}

Angenommen, die Voraussetzungen aus Satz~\ref{thm:existence} sind
für eine geeignete Funktion $\beta\not\equiv1$ erfüllt. Dann überführt
die Transformation~(\ref{eq:TS}) den Beobachter~(\ref{eq:luenberger})
in die Form 
\[
\dot{\hat{z}}=A\hat{z}+\alpha(\hat{z}_{n},u)+\left(S^{\prime}(\hat{z})\right)^{-1}k_{1}(S(\hat{z}))\,(\gamma(z_{n})-\gamma(\hat{z}_{n}))
\]
mit der Ausgangstransformation~$\gamma$. Die Linearisierung der
zugehörigen Fehlerdifferentialgleichung 
\[
\dot{\tilde{z}}=A\tilde{z}+\alpha(z_{n},u)-\alpha(\hat{z}_{n},u)-\left(S^{\prime}(\hat{z})\right)^{-1}k_{1}(S(\hat{z}))\,(\gamma(z_{n})-\gamma(\hat{z}_{n}))
\]
entlang der Beobachtertrajektorie~$\hat{z}$ liefert 
\[
\dot{\tilde{z}}=A\tilde{z}-\left(\left(S^{\prime}(\hat{z})\right)^{-1}k_{1}(S(\hat{z}),u)\frac{\partial}{\partial\hat{z}_{n}}\gamma(\hat{z}_{n})\!-\!\frac{\partial}{\partial\hat{z}_{n}}\alpha(\hat{z}_{n},u)\right)c^{T}\tilde{z}\!+\!\mathcal{O}\left(\|\tilde{z}\|^{2}\right).
\]
Mit der Wahl 
\[
k_{1}(\hat{x},u)=k_{1}(S(\hat{z}),u)=\frac{1}{\gamma^{\prime}(\hat{z}_{n})}\left(S^{\prime}(\hat{z})\,l+S^{\prime}(\hat{z})\,\frac{\partial}{\partial\hat{z}_{n}}\alpha(\hat{z}_{n},u)\right)
\]
erhält man die näherungsweise lineare Fehlerdifferentialgleichung~(\ref{eq:error-dgl-ordnung-1}).
Die Terme in der großen Klammer lassen sich mittels Gl.~(\ref{eq:gain-hg-x})
und~(\ref{eq:ad-n}) in den Orinigalkoordinaten ausdrücken. Zusammen
mit Gl.~(\ref{eq:dgamma-beta}) erhält man die Beobachterverstärkung
\begin{equation}
\begin{array}{lcl}
k_{1}(\hat{x},u) & = & \frac{1}{\beta(h(\hat{x}))}\left(p_{0}v(\hat{x})+\cdots+p_{n-1}\ad_{-f}^{n-1}v(\hat{x})\right.\\
 &  & \left.\quad\quad+\,\ad_{-f}^{n}v(\hat{x})+\ad_{-g}\ad_{-f}^{n-1}v(\hat{x},u)\right)
\end{array}\label{eq:ackermann-mit-ausgangstransformation}
\end{equation}
ähnlich wie in~\cite{schaffner99}.

\begin{example}
\label{exa:Synchronmaschine-ELO}Wir betrachten das Zustandsraummodell
\begin{equation}
\begin{array}{lcl}
\dot{x}_{1} & = & x_{2}\\
\dot{x}_{2} & = & B_{1}-A_{1}x_{2}-A_{2}x_{3}\sin x_{1}-\tfrac{1}{2}B_{2}\sin(2x_{1})\\
\dot{x}_{3} & = & u-D_{1}x_{3}+D_{2}\cos x_{1}\\
y & = & x_{1}
\end{array}\label{eq:synchronmotor-nfb}
\end{equation}
der Synchronmaschine\index{Synchronmaschine} aus Beispiel~\ref{exa:Synchronmaschine-HG}.
Aus~(\ref{eq:synchronmotor-nfb}) berechnet man die Beobachtbarkeitsmatrix
und deren Inverse:
\begin{equation}
Q_{B}^{-1}(x)=\left(\begin{array}{ccc}
1 & 0 & 0\\
0 & 1 & 0\\
-\frac{A_{2}x_{3}\cos x_{1}+B_{2}\cos(2x_{1})}{A_{2}\sin x_{1}} & -\frac{A_{1}}{A_{2}\sin x_{1}} & -\frac{1}{A_{2}\sin x_{1}}
\end{array}\right).\label{eq:synchronmotor-inv-beob-mat}
\end{equation}
Die einfachste Wahl für die Funktion~$\beta$ ist die einer konstanten
Funktion: $\beta(x_{1})\equiv1$. Das daraus resultierende Startvektorfeld
\begin{equation}
v(x)=\left(\begin{array}{c}
0\\
0\\
-\frac{1}{A_{1}\sin x_{1}}
\end{array}\right)\label{eq:synchronmotor-va}
\end{equation}
ist die letzte Spalte von~(\ref{eq:synchronmotor-inv-beob-mat}).
Es gilt $[v,\ad_{-f}v]=0$ und $[v,\ad_{-f}^{2}v]=0$, aber 
\[
[\ad_{-f}v,\ad_{-f}^{2}v](x)=\left(\begin{array}{c}
0\\
\cot x_{1}\\
*
\end{array}\right).
\]
Die letzte Lie-Klammer ist nur in der ersten Komponente identisch
Null. Die dritte Komponente ist ein komplizierter Ausdruck. 

Die Bedingung~\ref{enu:ki2} aus Satz~\ref{thm:existence} ist für
$\beta(x_{1})\equiv1$ offensichtlich nicht erfüllt. Allerdings kann
man die Vektorfelder $v$, $\ad_{-f}v$ und $\ad_{-f}^{2}v$ symbolisch
bestimmen, ohne sich hinsichtlich der Wahl der Funktion~$\beta$
festlegen zu müssen. Für eine beliebige glatte Funktion~$\beta$
gilt $[v,\ad_{-f}v]=[v,\ad_{-f}^{2}v]=0$ und 
\end{example}
\begin{equation}
[\ad_{-f}v,\ad_{-f}^{2}v](x)=\left(\begin{array}{c}
0\\
\frac{\beta(x_{1})}{\sin(x_{1})}\left(\cos(x_{1})\beta(x_{1})-3\sin(x_{1})\beta^{\prime}(x_{1})\right)\\
*
\end{array}\right)\label{eq:synchronmotor-Lie-Klammer-VarianteC-1}
\end{equation}
mit einem komplizierten Ausdruck in der dritten Komponente. Die zweite
Komponente verschwindet identisch, wenn~$\beta$ die gewöhnliche
Differentialgleichung 
\[
\cos(x_{1})\beta(x_{1})-3\sin(x_{1})\beta^{\prime}(x_{1})=0
\]
erfüllt. Die allgemeine Lösung dieser Differentialgleichung lautet
$\beta(x_{1})=C_{1}(\sin(x_{1}))^{1/3}$ mit der Konstanten~$C_{1}$.
Mit $\beta(x_{1})=(\sin(x_{1}))^{1/3}$ vereinfacht sich die Lie-Klammer~(\ref{eq:synchronmotor-Lie-Klammer-VarianteC-1})
zu 
\begin{equation}
[\ad_{-f}v,\ad_{-f}^{2}v](x)=\left(\begin{array}{c}
0\\
0\\
*
\end{array}\right),\label{eq:motor-Lie-best}
\end{equation}
wobei die dritte Komponente weiterhin von Null verschieden ist. Damit
ist offensichtlich, dass die Integrabilitätsbedingung (Bedingung~\ref{enu:ki2}
aus Satz~\ref{thm:existence}) mit dem Ausgang $y=x_{1}$ nicht zu
erfüllen ist.\footnote{Die exakte Linearisierung des Beobachtungsfehlers ist möglich, wenn
man statt des skalaren Ausgangs $y=x_{1}$ den vektoriellen Ausgang
$y=(x_{1},x_{2})^{T}$ verwendet~\cite{keller86diss}. Für diesen
vektoriellen Ausgang wird in~\cite{birk88} ein erweiterten Luenberger-Beobachter
entworfen. Für den skalaren Ausgang wird in~\cite{roebenack2013wseas}
eine approximative Linearisierung des Beobachtungsfehlers vorgeschlagen.} (Mit zwei Null\-ein\-trägen in Gl.~(\ref{eq:motor-Lie-best})
erzielt in gewissem Sinne eine bestmögliche Approximation des Integrabilitätsbedingung.)
Daher lässt sich kein Normalform-Beobachter~(\ref{eq:ons-orig})
entwerfen. Allerdings kann man trotzdem einen erweiterten Luenberger-Beobachter~(\ref{eq:luenberger})
mit der Beobachterverstärkung~(\ref{eq:ackermann-mit-ausgangstransformation})
angeben. 

\begin{remark}
Ist das zu überwachende System~(\ref{eq:plant}) (differentiell)
flach (siehe Abschnitt~\ref{sec:Flache-Systeme}, dann kann man den
Verlauf des Ausgangs durch eine hinreichend glatte Kurve vorgeben
(beispielsweise für einen Arbeitspunktwechsel) und daraus die zugehörigen
Referenzverläufe $x_{\text{ref}}:[0,T]\to\mathcal{M}$ und $u_{\text{ref}}:[0,T]\to\mathcal{U}$
von Zustand und Eingang auf dem Zeitintervall $[0,T]$ berechnen~\cite{FLMR95ijc,rothfuss97flach,rudolph2003habil}.
Statt wie in Gl.~(\ref{eq:luenberger}) bzw.~(\ref{eq:ackermann})
die Beobachterverstärkung $k_{1}(\hat{x},u)$ entlang der Schätztrajektorie~$\hat{x}$
zu bestimmen, kann man bei einem flachen System auch die (vorab berechneten)
Referenzverläufe einsetzen, wodurch sich eine zeitvariante Beobachterverstärkung
$t\mapsto k_{1}(x_{\text{ref}}(t),u_{\text{ref}}(t))$ ergibt. Daraus
resultiert ein \emph{flachheitsbasierter Folgebeobachter}\index{Folgebeobachter},
der mittlerweile bei zahlreichen Anwendungen zum Einsatz kommt~\cite{fliess96,bindel2000}.
\end{remark}

\begin{remark}
Die beschriebene Entwurfsidee, nämlich die Approximation einer linearen
Fehlerdynamik entlang der Schätztrajektorie, lässt sich auch auf Systeme
mit $p>1$ Ausgängen übertragen. Die Linearisierung der Fehlerdynamik
entlang der Beobachtertrajektorie führt auf einen erweiterten Luenberger-Beobachter~\cite{birk88,birk92},
dessen (zustands- und eingangsabhängige) Beobachterverstärkung $K_{1}:\mathcal{M}\times\mathcal{U}\to\R^{n\times p}$
matrixwertig ist.
\end{remark}

\begin{remark}
Die Beobachterverstärkungen~(\ref{eq:gain-hg-x}), (\ref{eq:ackermann})
und~(\ref{eq:ackermann-mit-ausgangstransformation}) mit dem Startvektorfeld~(\ref{eq:v})
werden durch verschiedene Lie-Ableitungen bzw. Lie-Klammern beschrieben.
Implementierungen mit Hilfe von Computer-Algebra-Software werden beispielsweise
in~\cite{birk89,birk90,birk92,rothfuss95} vorgestellt. Die symbolische
Berechnung dieser Beobachterverstärkungen kann jedoch auf außerordentlich
große Ausdrücke führen. In diesen Fällen sollte man die benötigten
Ableitungen nicht durch symbolisches Differenzieren berechnen, sondern
mit algorithmischem bzw. automatischem Differenzieren~\cite{griewank2008}.
Die Berechnung der in den Beobachterverstärkungen~(\ref{eq:gain-hg-x}),
(\ref{eq:ackermann}) und~(\ref{eq:ackermann-mit-ausgangstransformation})
auftretenden Lie-Klammern mit Hilfe des algorithmuischen Differenzierens
wird in~\cite{roebenack2003buch,roebenack2004ima,roebenack2004nolcos,roebenack2005habil}
beschrieben. Ein Anwendungsbeispiel aus der Kristallzüchtung ist in~\cite{lindert-gamm2004,winkler2012ssd}
zu finden.
\end{remark}

\subsection{Approximation höherer Ordnung\label{sec:Higher-Order-Approximation}}

Der erweiterte Luenberger-Beobachter liefert die Approximation erster
Ordnung von einer exakt linearen Fehlerdynamik. Nachfolgend wird ein
Ansatz zum Erreichen einer höheren Approximationsordnung vorgestellt~\cite{roebenack2004ijc,roebenack2004pamm,roebenack2007ima}.
Dazu betrachten wir einen Beobachter der Form 
\begin{equation}
\dot{\hat{x}}=F(\hat{x},u)+\sum_{i=1}^{N}k_{i}(\hat{x},u)(y-h(\hat{x}))^{i}\label{eq:obs-new}
\end{equation}
mit glatten Vektorfeldern $k_{1},\ldots,k_{N}:\mathcal{M}\times\mathcal{U}\to\R^{n}$.
Sind die Voraussetzungen von Satz~\ref{thm:existence} für $\beta(\cdot)=1$
erfüllt, dann kann man die Transformation~(\ref{eq:TS}) auf~(\ref{eq:obs-new})
anwenden und erhält damit die Fehlerdifferentialgleichung 
\begin{equation}
\dot{\tilde{z}}=A\tilde{z}+\alpha(z_{n},u)-\alpha(\hat{z}_{n},u)-\left(S^{\prime}(\hat{z})\right)^{-1}\sum_{i=1}^{N}k_{i}(S(\hat{z}),u)\tilde{z}_{n}^{i}.\label{eq:error-dgl-new}
\end{equation}
Ähnlich wie im Abschnitt~\ref{subsec:Luenberger} wird jetzt eine
Reihenentwicklung von $\alpha(z_{n},u)$ entlang der Trajektorie~$\hat{z}$
des Beobachters betrachtet: 
\begin{equation}
\alpha(z_{n},u)=\alpha(\hat{z}_{n},u)+\sum_{i=1}^{N}\frac{\partial^{i}}{i!\,\partial\hat{z}_{n}^{i}}\alpha(\hat{z}_{n},u)\,\tilde{z}_{n}^{i}+\mathcal{O}\left(|\tilde{z}_{n}|^{N+1}\right).\label{eq:alpha-series-expansion}
\end{equation}
 Wählt man die Beobachterverstärkung~$k_{1}$ entsprechend Gl.~(\ref{eq:k1z}),
so erhält man 
\[
\begin{array}{lcl}
\dot{\tilde{z}} & = & \left(A-lc^{T}\right)\tilde{z}\\
 &  & +\sum_{i=2}^{N}\left[\left(\frac{\partial^{i}}{i!\,\partial\hat{z}_{n}^{i}}\alpha(\hat{z}_{n},u)-\left(S^{\prime}(\hat{z})\right)^{-1}k_{i}(S(\hat{z}),u)\right)\tilde{z}_{n}^{i}\right]+\mathcal{O}\left(\|\tilde{z}\|^{N+1}\right).
\end{array}
\]
Die Nichtlinearitäten sollen bis zur Ordnung~$N$ kompensiert werden.
Mit 
\begin{equation}
k_{i}(\hat{x},u)=\frac{1}{i!}\left.S^{\prime}(\hat{z})\frac{\partial^{i}}{\partial\hat{z}_{n}^{i}}\alpha(\hat{z}_{n},u)\right|_{\hat{z}=T(\hat{x})}\quad\textrm{für}\quad i=2,\ldots,N\label{eq:ki-z-new}
\end{equation}
erhält man die gewünschte Fehlerdynamik 
\begin{equation}
\dot{\tilde{z}}=\left(A-lc^{T}\right)\tilde{z}+\mathcal{O}\left(\|\tilde{z}\|^{N+1}\right).\label{eq:error-o-N}
\end{equation}
 

Die Beobachterverstärkung~(\ref{eq:ki-z-new}) soll in den Originalkoordinaten
ausgedrückt werden. Zur Vereinfachung wird die Notation 
\[
\alpha^{(j)}(z_{n},u)=\frac{\partial^{j}}{\partial z_{n}^{j}}\alpha(z_{n},u)
\]
verwendet und zusätzlich das Vektorfeld~$w$ durch 
\[
w(x):=\ad_{-f}^{n-1}v(x)
\]
definiert. Die Behauptung 
\begin{equation}
S^{\prime}(z)\alpha^{(j)}(z_{n},u)=\ad_{w}^{j-1}\ad_{-F}^{n}v(x,u)\quad\textrm{für}\quad j\geq1\label{eq:claim}
\end{equation}
wird mit vollständiger Induktion bewiesen. Wegen~(\ref{eq:ad-n})
gilt Gl.~(\ref{eq:claim}) für $j=1$:
\[
S^{\prime}(z)\alpha^{(1)}(z_{n},u)=\ad_{-F}^{n}v(x,u)=\ad_{w}^{0}\ad_{-F}^{n}v(x,u).
\]
Gl.~(\ref{eq:claim}) gelte nun für $j=1,\ldots,i$. Die Differentiation
liefert 
\begin{equation}
\frac{\partial}{\partial z_{n}}\left(S^{\prime}(z)\alpha^{(i)}(z_{n},u)\right)=S^{\prime}(z)\alpha^{(i+1)}(z_{n},u)+\left(\frac{\partial}{\partial z_{n}}S^{\prime}(z)\right)\alpha^{(i)}(z_{n},u).\label{eq:partial}
\end{equation}
Da Gl.~(\ref{eq:claim}) nach Induktionsvoraussetzung für $j=i$
gilt, kann man die linke Seite von Gl.~(\ref{eq:partial}) folgendermaßen
umformen: 
\begin{equation}
\begin{array}{ccl}
\frac{\partial}{\partial z_{n}}\left(S^{\prime}(z)\alpha^{(i)}(z_{n},u)\right) & = & \frac{\partial}{\partial z_{n}}\left.\ad_{w}^{i-1}\ad_{-F}^{n}v(x,u)\right|_{x=S(z)}\\
 & = & \left(\frac{\partial}{\partial x}\ad_{w}^{i-1}\ad_{-F}^{n}v(x,u)\right)\frac{\partial}{\partial z_{n}}S(z)\\
 & = & \left(\frac{\partial}{\partial x}\ad_{w}^{i-1}\ad_{-F}^{n}v(x,u)\right)w(x).
\end{array}\label{eq:X1}
\end{equation}
Außerdem gilt 
\begin{equation}
\begin{array}{lcl}
\frac{\partial}{\partial z_{n}}S^{\prime}(z) & = & \frac{\partial}{\partial z_{n}}\frac{\partial}{\partial z}S(z)\\
 & = & \frac{\partial}{\partial z}\frac{\partial}{\partial z_{n}}S(z)\\
 & = & \frac{\partial}{\partial z}\left.w(x)\right|_{x=S(z)}\\
 & = & w^{\prime}(x)S^{\prime}(z).
\end{array}\label{eq:NR}
\end{equation}
Nutzt man die Induktionsvoraussetzung~(\ref{eq:claim}) ein weiteres
Mal für $j=i$ sowie Gl.~(\ref{eq:NR}), dann bekommt man für den
zweiten Summanden auf der rechten Seite von Gl.~(\ref{eq:partial})
den Ausdruck 
\begin{equation}
\begin{array}{ccl}
\left(\frac{\partial}{\partial z_{n}}S^{\prime}(z)\right)\alpha^{(i)}(z_{n},u) & = & w^{\prime}(x)S^{\prime}(z)\alpha^{(i)}(z_{n},u)\\
 & = & w^{\prime}(x)S^{\prime}(z)\left(S^{\prime}(z)\right)^{-1}\ad_{w}^{i-1}ad_{-F}^{n}v(x,u)\\
 & = & w^{\prime}(x)\,\ad_{w}^{i-1}\ad_{-F}^{n}v(x,u).
\end{array}\label{eq:X2}
\end{equation}
Die Gln.~(\ref{eq:partial}), (\ref{eq:X1}) und~(\ref{eq:X2})
liefern
\[
\begin{array}{ccl}
S^{\prime}(z)\alpha^{(i+1)}(z_{n},u) & = & \frac{\partial}{\partial z_{n}}\left(S^{\prime}(z)\alpha^{(i)}(z_{n},u)\right)-\left(\frac{\partial}{\partial z_{n}}S^{\prime}(z)\right)\alpha^{(i)}(z_{n},u)\\
 & = & \left(\frac{\partial}{\partial x}\ad_{w}^{i-1}\ad_{-F}^{n}v(x,u)\right)w(x)-w^{\prime}(x)\,\ad_{w}^{i-1}\ad_{-F}^{n}v(x,u)\\
 & = & \ad_{w}^{i}\ad_{-F}^{n}v(x,u),
\end{array}
\]
womit Gl.~(\ref{eq:claim}) bewiesen wäre. Folglich hat die Beobachterverstärkung~(\ref{eq:ki-z-new})
in den Originalkoordinaten die Form 
\begin{equation}
\begin{array}{lcl}
k_{i}(\hat{x},u) & = & \frac{1}{i!}\ad_{w}^{i-1}\ad_{-F}^{n}v(\hat{x},u)\\
 & = & \frac{1}{i!}\left(\ad_{w}^{i-1}\ad_{-f}^{n}v(\hat{x})+\ad_{w}^{i-1}\ad_{-g}\ad_{-f}^{n-1}v(\hat{x},u)\right).
\end{array}\label{eq:ki-x-new}
\end{equation}
für $i\geq2$. Für die Beobachterverstärkung~$k_{2}$, die einer
Approximation zweiter Ordnung entspricht, erhält man 
\[
\begin{array}{lcl}
k_{2}(\hat{x},u) & = & \frac{1}{2}\ad_{w}\ad_{-F}^{n}v(\hat{x},u)\\
 & = & \frac{1}{2}[w,\ad_{-F}^{n}v](\hat{x},u)\\
 & = & \frac{1}{2}[\ad_{-f}^{n-1}v,\ad_{-f}^{n}v+\ad_{-g}\ad_{-f}^{n-1}v](\hat{x},u).
\end{array}
\]
Bei einem System ohne Stelleingriff vereinfacht sich diese Beobachterverstärkung
zu 
\begin{equation}
k_{2}(\hat{x})=\frac{1}{2}[\ad_{-f}^{n-1}v,\ad_{-f}^{n}v](\hat{x}),\label{eq:k2-unforced}
\end{equation}
siehe~\cite{roebenack2004pamm}. Ist~(\ref{eq:k2-unforced}) identisch
mit dem Nullvektor, dann ist die Ausgangsaufschaltung des betrachteten
Systems linear~\cite[Prop.~2]{krener83}.

\begin{example}
\label{exa:vdp-hoehere-Ordnung}Beim van der Pol-System~(\ref{eq:vdp})
aus Beispiel~\ref{exa:van-der-Pol-Krener-Isidori} erhält man mit
$w(\hat{x})=\ad_{-f}v(\hat{x})$ aus Gl.~(\ref{eq:vdp-Lie-Klammern2})
die Beobachterverstärkungen
\begin{equation}
k_{2}(\hat{x})=\left(\begin{array}{c}
-\hat{x}_{1}\\
\hat{x}_{1}(\hat{x}_{1}^{2}-1)
\end{array}\right),\;k_{3}(\hat{x})=\frac{1}{3}\left(\begin{array}{c}
-1\\
\hat{x}_{1}^{2}-1
\end{array}\right)\label{eq:vdp-k23}
\end{equation}
und $k_{j}(\hat{x})=(0,0)^{T}$ für $j\geq4$. Da $\alpha(y,u)$ ein
Polynom in~$y$ ist, bricht die Taylor\-entwicklung von~(\ref{eq:alpha-series-expansion})
nach endlich vielen Termen ab. Außerdem hängen die Vektor\-felder
von~(\ref{eq:vdp-k23}) nicht vom Eingang ab, da $u$ nur als Faktor
zur ersten Potenz von~$y$ auftritt (siehe Gl.~(\ref{eq:vdp-beobachternormalform})).
Man kann unmittelbar nachrechnen, dass der Korrektorterm~(\ref{eq:vdp-kinf})
des Normalformbeobachters aus Beispiel~\ref{exa:van-der-Pol-Krener-Isidori}
übereinstimmt mit dem Korrekturterm des Beobachters~(\ref{eq:obs-new})
bei Verwendung der Beobachterverstärkungen~(\ref{eq:k1-vdp}) und~(\ref{eq:vdp-k23}).
In diesem Beispiel führt der beschriebene Reihenansatz mit endlich
vielen Termen auf eine exakt lineare Fehlerdynamik.
\end{example}

\begin{example}
\label{exa:roessler-hoehere-Ordnung}Im Fall des Rössler-Systems~(\ref{eq:roessler-attraktor})
aus Beispiel~\ref{exa:Roessler-Krener-Isidori} mit dem Ausgang~(\ref{eq:roessler-ausgang2})
aus Beispiel~\ref{exa:Roessler-Tao} erhält man für den Beobachter~(\ref{eq:obs-new})
die Beobachterverstärkungen 
\[
k_{i}(\hat{x})=\frac{1}{i!}\left(\begin{array}{c}
-\hat{x}_{3}\\
0\\
(-1)^{i}c
\end{array}\right)\quad\mbox{für}\quad i\geq2.
\]
\end{example}

Mit Gl.~(\ref{eq:ki-x-new}) lassen sich die Beobachterverstärkungen
$k_{2},\ldots,k_{N}$ direkt bestimmen. Zur Berechnung dieser Beobachterverstärkungen
ist weder eine explizite Berechnung der Koordinatentransformation~(\ref{eq:TS})
noch die konkrete Kenntnis der Normalform~(\ref{eq:nf}) bzw.~(\ref{eq:nf-ohne-gamma})
erforderlich. Die Integrabilitätsbedingung~\ref{enu:ki2} aus Satz~\ref{thm:existence}
wird benötigt, um die Jacobimatrix~$S^{\prime}(\hat{z})$ entsprechend
Gl.~(\ref{eq:pde}) durch~$\Pi(\hat{x})$ ersetzen zu können. Die
Beobachterverstärkungen $k_{1},\ldots,k_{N}$ lassen sich auch dann
mit den Gln.~(\ref{eq:ackermann}) und~(\ref{eq:ki-x-new}) berechnen,
wenn die Bedingung~\ref{enu:ki2} verletzt ist.

\medskip{}

\begin{remark}
Aus theoretischer Sicht ist es problemlos möglich, die vorgestellte
Approximation höherer Ordnung auch bei Systemen mit mehreren Ausgängen
durchzuführen~\cite{roebenack2005ifac}. Bei der Taylorentwicklung
der Eingangs-Ausgangs-Aufschaltung müssen dann allerdings auch die
partiellen Ableitungen nach den verschiedenen Ausgängen berücksichtigt
werden. Dadurch erhält man tensorwertige Beobachterverstärkungen
\[
K_{2}:\mathcal{M}\times\mathcal{U}\to\R^{n\times p\times p},\;K_{3}:\mathcal{M}\times\mathcal{U}\to\R^{n\times p\times p\times p},\;\ldots,
\]
deren Speicherbedarf nur in wenigen Anwendungesfällen zu rechtfertigen
ist.
\end{remark}

\nocite{nijmeijer99,Corliss2001}

\nocite{jakubczyk85}
\nocite{meurer2005syncod}

\bibliographystyle{babalpha}
\bibliography{dynamic}

