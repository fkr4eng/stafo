
\preface{}

Nichtlineare Regelungs- und Steuerungsverfahren kommen überall dort
zum Einsatz, wo die Leistungsfähigkeit linearer Methoden an ihre Grenzen
stößt. An- und Abfahrvorgänge bzw. Arbeitspunktwechsel in chemischen
Reaktoren, die hochpräzise Lageregelung von Starrkörpern im Raum (beispielsweise
Effektoren von Industrierobotern oder autonome Flugkörper) oder die
Regelung leistungselektronischer Baugruppen mögen als Beispiele dienen.
Vor diesem Hintergrund verwundert es nicht, dass Ansätze zur nichtlinearen
Regelung in der industriellen Praxis zunehmend auf Interesse stoßen,
um die Ausbeute und Zuverlässigkeit der zugrundliegenden technischen
Prozesse zu erhöhen. Gleichwohl ist der in der nichtlinearen Regelungstheorie
verwendete mathematische Apparat gefürchtet. Genau an dieser Stelle
setzt dieses Lehrbuch an: Die Darstellung ist einerseits in sich mathematisch
schlüssig und nachvollziehbar, andererseits aber auch in einer für
Ingenieure verständlichen Sprache formuliert. Der inhaltliche Fokus
des Buches liegt dabei auf der Methode der exakten Linearisierung
zum Regler- und Beobachterentwurf.

Das Buch richtet sich maßgeblich an Studierende der Elektrotechnik
und Mechatronik in der Vertiefungsrichtung Automatisierungs- bzw.
Regelungstechnik, die bereits über fortgeschrittene regelungstechnische
Kenntnisse verfügen. Doktoranden können anhand des Buches ihr systemtheoretisches
bzw. regelungstechnisches Verständnis nichtlinearer Systeme vertiefen.
Für Ingenieure in der Industrie, die mit der Regelung nichtlinearer
Systeme konfrontiert sind, werden verschiedene Entwurfsansätze detailliert
beschrieben.

Im Teil~\ref{part:Vorbereitung} des Buches werden nach einigen Einführungsbeispielen
die benötigten mathematischen Konzepte aus den Bereichen der linearen
Algebra, der Vektoranalysis bzw. der Differentialgeometrie vermittelt.
Zur Festigung der Lehrinhalte sind am Ende der Kapitel~\ref{cha:Grundlagen}
und~\ref{chap:Diff-Geo} Übungsaufgaben vorgesehen. Die Teile~\ref{part:Reglerentwurf}
und~\ref{part:Beobachterentwurf} behandeln den Regler- bzw. Beobachterentwurf.
Die jeweilige Herangehensweise bzw. Entwurfsmethodik wird an verschiedenen
Beispielen veranschaulicht bzw. durch den Einsatz des Open-Source-Computeralgebrasystems
\textsc{Maxima} illustriert.

Erste Skizzen zum vorliegenden Buch entstanden in Verbindung mit der
Vorlesung ,,Mathematische Grundlagen der nichtlinearen Regelungstheorie``,
die ich im Sommersemester 2006 an der Fakultät Mathematik und Naturwissenschaften
der TU Dresden hielt. Diese Lehrveranstaltung führe ich seit 2007
als Wahlpflichtfach ,,Nichtlineare Regelungstechnik 2`` für verschiedene
Ingenieurstudiengänge fort. Meine zehnjährige Lehrtätigkeit in diesem
Gebiet hat das vorliegende Buch maßgeblich geprägt.

Es ist mir ein Bedürfnis, zuallererst meinem Lehrer, Herrn Prof. Kurt
Reinschke, für seine langjährige Unterstützung zu danken. Besonderer
Dank gilt auch Herrn Prof. Andreas Griewank für seine Anregungen zur
Beschäftigung mit dem algorithmischen Differenzieren. Ebenso möchte
ich Herrn Prof. Wolfgang V. Walter danken, durch den ich in den Jahren
2006 bis 2007 die Vertretung der Professur ,,Wissenschaftliches Rechnen
und Angewandte Mathematik`` wahrnehmen konnte.

Bei der Gestaltung der dem Buch zugrundeliegenden Vorlesung und der
Konzeption des Manuskripts haben mich zahlreiche Diplomanden, heutige
und ehemalige Mitarbeiter und Doktoranden unterstützt. Besonderer
Dank gilt Dr. Carsten Knoll, der das Buch durch zahlreiche Verbesserungsvorschläge
erheblich bereichert hat. Außerdem danke ich Dr. Jan Winkler, Dr.
Matthias Franke und Prof. Frank Woittennek für die angenehme Zusammenarbeit
und die interessanten Diskussionen. Herrn M.Sc. Rick Voßwinkel danke
ich für die sehr gewissenhafte Durchsicht des Manuskripts. Mein Dank
gebührt auch Dr. Carsten Collon, Dipl.-Ing. Mirko Franke, Dipl.-Ing.
Chenzi Huang, Dipl.-Ing. Fabian Paschke, Dipl.-Ing. Klemens Fritzsche,
Dipl.-Ing. Gunter Nitzsche und Dipl.-Ing. Robert Huber.

Zahlreiche Diskussionen bei Konferenzen, Tagungen und Workshops haben
mein Verständnis für nichtlineare Systeme vertieft. Für fachliche
Anregungen möchte ich daher Dr. Albrecht Gensior (TU Dresden), Associate
Prof. Pranay Goel (Indian Institute of Science Education and Research,
Pune), PD Dr. Lutz Gröll (Karlsruher Institut für Technologie), Prof.
Alan F. Lynch (University of Alberta, Kanada), Prof. Jaime A. Moreno
(Universidad Nacional Autónoma de México), Prof. Joachim Rudolph (Universität
des Saarlandes), Prof. Andrea Walther (Universität Paderborn) und
Prof. Zeitz (Universität Stuttgart) danken.

Frau Eva Hestermann-Beyerle und Frau Birgit Kollmar-Thoni vom Springer-Verlag
danke ich für die angenehme Zusammenarbeit. Mein Dank gilt auch dem
\hbox{De~Gruyter}-Verlag für die Nutzungsgenehmigung im Zusammenhang
mit Kapitel~\ref{cha:Beobachter-Normalform}. Meiner Frau und meinen
Kindern danke ich für ihre Geduld und ihr Verständnis.

\vspace{\baselineskip}

\noindent \begin{flushright}
Dresden, März 2017\hfill{}Klaus Röbenack
\par\end{flushright}
