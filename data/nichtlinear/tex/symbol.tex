
\chapter*{Symbolverzeichnis\addcontentsline{toc}{chapter}{Symbolverzeichnis}}

Die meisten der verwendeten Symbole werden im laufenden Text eingeführt.
Die nachfolgende Übersicht ist auf die gebräuchlichsten Symbole beschränkt:

\bigskip{}

\begin{tabular}{ll}
$\NN$, $\N$ & Menge der natürlichen Zahlen ab~$0$ bzw.~1\tabularnewline
$\Z$ & Ring der ganzen Zahlen\tabularnewline
$\R$, $\C$ & Körper der reellen bzw. komplexen Zahlen\tabularnewline
$\R^{n}$ & $n$-dimensionaler reeller Vektorraum\tabularnewline
$(\R^{n})^{*}$ & Dualraum des Vektorraums~$\R^{n}$\tabularnewline
$\R^{n\times m}$ & Vektorraum der reellen $n\times m$-Matrizen\tabularnewline
$\spann\{v_{1},\ldots,v_{r}\}$ & lineare Hülle der Vektoren $v_{1},\ldots,v_{r}$\tabularnewline
$\in$, $\notin$ & ist Element, ist nicht Element\tabularnewline
$\forall$, $\exists$  & Generalisator (,,für alle``), Partikularistor (,,es exisiert``)\tabularnewline
$\times$ & kartesisches Produkt oder Kreuzprodukt\tabularnewline
$\subseteq$, $\subset$ & Teilmenge, echte Teilmenge\tabularnewline
$\cup$, $\cap$ & Vereinigung, Schnitt\tabularnewline
$e_{i}$, $e_{i}^{*}$ & $i$-ter Einheitsvektor des~$\R^{n}$ bzw. des Dualraums~$(\R^{n})^{*}$\tabularnewline
$I$, $I_{n}$ & Einheitsmatrix passender Dimension, $n\times n$-Einheitsmatrix\tabularnewline
$M^{T}$, $M^{-1}$ & transponierte bzw. inverse Matrix\tabularnewline
$\im\,M$, $\ker\,M$ & Bild bzw. Kern einer Matrix~$M$\tabularnewline
$\rank\,M$ & Rang der Matrix~$M$\tabularnewline
$\dim\mathbb{U}$ & Dimension eines Unterraums~$\mathbb{U}$\tabularnewline
$\mathbb{U}^{\perp}$ & othogonales Komplement eines Unterraums~$\mathbb{U}$\tabularnewline
$\Delta^{\perp}$, $\Omega^{\perp}$ & Annihilator einer Distribution~$\Delta$ bzw. einer Kodistribution~$\Omega$\tabularnewline
$\dot{x}$, $\ddot{x}$, $x^{(k)}$ & erste, zweite, $k$-te Zeitableitung von~$x$\tabularnewline
$\frac{\partial}{\partial x_{i}}$, $\d x_{i}$ & $i$-tes Einheitsvektorfeld bzw. -kovektorfeld\tabularnewline
$\d\Psi$, $\Psi^{\prime}$ & Gradient bzw. Jacobimatrix einer Abbildung~$\Psi$\tabularnewline
$\Psi_{*}$, $\Psi^{*}$ & Pushforward bzw. Pullback zu einer Abbildung~$\Psi$\tabularnewline
$\mathcal{O}(\cdot)$ & Landau-Symbol: Asymptotische obere Schranke\tabularnewline
\end{tabular}
