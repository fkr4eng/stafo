
\appendix

\chapter{Stabilität\label{chap:introA}}

\section{Stabilität autonomer Systeme\label{sec:Stabilitaet-autonomer-Systeme}}

In diesem Abschnitt sind wesentliche Aussagen der klassischen Stabilitätstheorie
zusammengefasst~\cite{hahn1967,slotine1991,sastry1999,khalil2002}.
Gegeben sei ein autonomes System
\begin{equation}
\dot{x}=f(x)\label{eq:anhang-system-autonom}
\end{equation}
mit einem hinreichend glatten Vektorfeld $f:\mathcal{M}\to\R^{n}$
und der offenen Menge $\mathcal{M}\subseteq\R^{n}$. Bei globalen
Betrachtungen gehen wir von $\mathcal{M}=\R^{n}$ aus. Der Fluss des
Differentialgleichungssystems~(\ref{eq:anhang-system-autonom}) sei
mit $\varphi_{t}(\cdot)$ bezeichnet.

Das System hat eine Ruhelage\index{Ruhelage} im Punkt $x^{0}\in\mathcal{M}$,
falls $f(x^{0})=0$ (siehe Abschnitt~\ref{sec:Vektorfelder-und-Fluesse}).
Ohne Einschränkung liege die Ruhelage im Ursprung, d.\,h. $f(0)=0$.
(Liegt die Ruhelage~$x^{0}$ nicht im Ursprung, so kann man sie mit
der Koordinatentransformation $\bar{x}=x-x^{0}$ in den Ursprung $\bar{x}^{0}=0$
verschieben.)

Eine Ruhelage $x^{0}=0$ von System~(\ref{eq:anhang-system-autonom})
heißt\emph{ stabil} (im Sinne von Ljapunov, engl. \emph{stable}),
wenn für alle $\varepsilon>0$ ein $\delta>0$ existiert, so dass
\begin{equation}
\|p\|<\delta\quad\Longrightarrow\quad\forall t\geq0:\;\|\varphi_{t}(p)\|<\varepsilon.\label{eq:anhang-stabil}
\end{equation}
Das bedeutet, dass bei einer hinreichend kleinen Abweichung des Anfangswerts
von der Ruhelage die Systemtrajektorie für alle fortlaufenden Zeiten
$t\geq0$ in den Nähe der Ruhelage bleibt. Andernfalls heißt die Ruhelage
\emph{instabil} (engl. \emph{unstable}). Die Bedingung~(\ref{eq:anhang-stabil})
wird in Abb.~\ref{fig:stabil-im-Sinne-Lyapunov} veranschaulicht.

\begin{figure}
\begin{centering}
\resizebox{0.87\textwidth}{!}{\input{lyap-stabil-asymp.pdftex_t}}
\par\end{centering}
\caption{Illustration der Stabilitätsbedingung~(\ref{eq:anhang-stabil}) in
Anlehnung an~\cite{slotine1991}\label{fig:stabil-im-Sinne-Lyapunov}}

\end{figure}

Eine stabile Ruhelage $x^{0}=0$ heißt (\emph{lokal}) \emph{asymptotisch
stabil}, wenn zusätzlich zu~(\ref{eq:anhang-stabil}) auch 
\begin{equation}
\|p\|<\delta\quad\Longrightarrow\quad\lim_{t\to\infty}\varphi_{t}(p)=0\label{eq:anhang-asymptotisch}
\end{equation}
erfüllt ist. Gilt bei einer stabilen Ruhelage die Bedingung~(\ref{eq:anhang-asymptotisch})
für alle Anfangswerte $p\in\mathcal{M}$, dann heißt die Ruhelage
\emph{global asymptotisch stabil}. Eine Ruhelage mit der Eigenschaft~(\ref{eq:anhang-asymptotisch})
nennt man \emph{attraktiv}. Aus der Attraktivität einer Ruhelage folgt
nicht zwangsläufig deren Stabilität~\cite{vinograd1957,hahn1967}.

Zur übersichtlichen Formulierung von Stabilitätsaussagen ist es hilfreich,
bestimmte Klassen von Vergleichsfunktionen\index{Vergleichsfunktion}
einzuführen. Eine Funktion $\gamma:[0,a)\to[0,\infty)$ mit $a>0$
gehört zur Klasse~$\mathcal{K}$ (Notation: $\gamma\in\mathcal{K}$),
wenn sie streng monoton wachsend ist und $\gamma(0)=0$ gilt. Eine
Funktion~$\gamma$ mit $a=\infty$ gehört zur Klasse~$\mathcal{K}_{\infty}$,
wenn zusätzlich $\gamma(r)\to\infty$ für $r\to\infty$ gilt.

Sei $V:\mathcal{M}\to\R$ eine stetig differenzierbare Funktion mit
$V(0)=0$. Die Funktion~$V$ heißt (\emph{lokal}) \emph{positiv definit}\index{Funktion!positiv definite},
falls es Funktionen $\gamma_{1},\gamma_{2}\in\mathcal{K}$ gibt, so
dass 
\begin{equation}
\gamma_{1}\left(\|x\|\right)\leq V(x)\leq\gamma_{2}\left(\|x\|\right)\label{eq:Ljap2}
\end{equation}
für alle~$x$ aus einer Umgebung von Null gilt (siehe Abb.~\ref{fig:Positiv-definit}).
Die Funktion~$V$ heißt \emph{global positiv definit}, wenn die Umgleichung~(\ref{eq:Ljap2})
für Funktionen $\gamma_{1},\gamma_{2}\in\mathcal{K}_{\infty}$ und
alle $x\in\R^{n}$ gilt. Eine Funktion $V:\R^{n}\to\R$ mit $V(x)>0$
für alle $x\neq0$ ist genau dann global positiv definit, wenn sie
radial unbeschränkt ist, d.\,h. $V(x)\to\infty$ für alle $\|x\|\to\infty$
gilt.

\begin{figure}
\begin{centering}
\resizebox{0.5\textwidth}{!}{\input{positiv-definit.pdftex_t}}
\par\end{centering}
\caption{Bedingung~(\ref{eq:Ljap2}) der positiven Definitheit in Anlehnung
an~\cite[Fig.~{4.1}]{slotine1991}\label{fig:Positiv-definit}}

\end{figure}

Auf Basis der Zeitableitung 
\[
\dot{V}(x)=\frac{\partial V(x)}{\partial x}\dot{x}=\frac{\partial V(x)}{\partial x}f(x)=L_{f}V(x)
\]
von~$V$ entlang~(\ref{eq:anhang-system-autonom}) lassen sich folgende
Stabilitätsaussagen treffen~\cite{hahn1967}:
\begin{theorem}
\label{thm:Lyapunov-Theorem}Sei $V$ positiv definit und $x^{0}=0$
eine isolierte Ruhelage von System~(\ref{eq:anhang-system-autonom}).
Die Ruhelage $x^{0}=0$ ist
\begin{enumerate}
\item (lokal) stabil, falls 
\begin{equation}
\dot{V}(x)\leq0\label{eq:Ljap3}
\end{equation}
für alle~$x$ aus einer Umgebung von $x^{0}=0$ gilt.
\item (lokal) asymptotisch stabil, falls es eine Funktion $\gamma_{3}\in\mathcal{K}$
gibt, so dass 
\begin{equation}
\dot{V}(x)\leq-\gamma_{3}\left(\|x\|\right)\label{eq:Ljap4}
\end{equation}
für alle~$x$ aus einer Umgebung von $x^{0}=0$ gilt.
\item global asymptotisch stabil, falls $V$ global positiv definit ist
und die Ungleichung~(\ref{eq:Ljap4}) für ein $\gamma_{3}\in\mathcal{K}_{\infty}$
und alle $x\in\R^{n}$ gilt.
\end{enumerate}
\end{theorem}

Eine Funktion~$\dot{V}$, die Gl.~(\ref{eq:Ljap3}) bzw.~(\ref{eq:Ljap4})
erfüllt, nennt man \emph{negativ semi\-definit} bzw. \emph{negativ
definit}\index{Funktion!negativ definite}. Ist die Bedingung~(\ref{eq:Ljap4})
für eine Funktion $\gamma_{3}\in\mathcal{K}_{\infty}$ erfüllt, dann
nennt man~$\dot{V}$ \emph{global negativ definit}. Eine positiv
definite Funktion~$V$, die Gl.~(\ref{eq:Ljap3}) erfüllt, heißt
\emph{Ljapunov-Funktion}\index{Ljapunov-Funktion}\index{Funktion!Ljapunov-}
(engl. \emph{Lyapunov function}). Eine Ljapunov-Funktion, die zusätzlich
die Bedingung~(\ref{eq:Ljap4}) für ein $\gamma_{3}\in\mathcal{K}$
erfüllt, nennt man \emph{strenge Ljapunov-Funktion} (engl. \emph{strict
Lyapunov function}).\index{Funktion!strenge Ljapunov-} 

\medskip{}

Als Spezialfall von System~(\ref{eq:anhang-system-autonom}) betrachten
wir das lineare zeitinvariante Differentialgleichungssystem
\begin{equation}
\dot{x}=A\,x\label{eq:anhang-LTI-ohne-erregung}
\end{equation}
mit $A\in\R^{n\times n}$. Das System hat im Punkt $x^{0}=0$ eine
Ruhelage. Gilt $\rank\,A=n$, so liegt eine isolierte Ruhelage vor.
Die Ruhelage $x^{0}=0$ ist genau dann global asymptotisch stabil
im Sinne von Ljapunov, wenn alle Eigenwerte der Matrix~$A$ negativen
Realteil besitzen. Eine solche Matrix nennt man \emph{stabil}.\footnote{Eine stabile Matrix wird von manchen Autoren auch als \emph{Hurwitz-Matrix}
bezeichnet. Das kann allerdings zur Verwechslung mit der beim Stabilitätskriterium
von Hurwitz auftretenden Koeffizientenmatrix führen~\cite{gantmacher86,reinschke2014buch}.}\index{Matrix!stabile} Für die Stabilitätsuntersuchung nichtlinearer
Systeme, die ein lineares Teilsystem enthalten, ist es für die Konstruktion
einer Ljapunov-Funktion für das Gesamtsystem oft hilfreich, wenn man
für das lineare Teilsystem eine Ljapunov-Funktion angeben kann. Deren
Existenz und Berechnung werden nachfolgend betrachtet. 

Für lineare Systeme gibt man die Vergleichsfunktionen\index{Vergleichsfunktion}
typischerweise in Matrixnotation an. In diesem Sinne werden nachfolgend
Definitheitsbegriffe für quadratische Matrizen formuliert. Wir beschränken
uns dabei auf symmetrische Matrizen $P=P^{T}$. Eine Matrix $P\in\R^{n\times n}$
heißt \emph{positiv semidefinit}\index{Matrix!positiv semidefinite}
($P\succeq0$) bzw. \emph{negativ semidefinit}\index{Matrix!negativ semidefinite}
($P\preceq0$), wenn $x^{T}Px\geq0$ bzw. $x^{T}Px\leq0$ für alle
$x\in\R^{n}$ gilt. Die Matrix~$P$ heißt \emph{positiv definit}\index{Matrix!positiv definite}
($P\succ0$) bzw. \emph{negativ definit}\index{Matrix!negativ definite}
($P\prec0$), wenn $x^{T}Px>0$ bzw. $x^{T}Px<0$ für alle $x\neq0$
gilt~\cite{zhou1998}. Die Verbindung zur positiven Definitheit einer
Funktion entsprechend Gl.~(\ref{eq:Ljap2}) stellt der Satz von Courant-Fischer\index{Satz!von Courant-Fischer}
(Minimum-Maximum-Prinzip) her:
\[
\lambda_{\min}(P)\cdot\left\Vert x\right\Vert ^{2}\leq x^{T}Px\leq\lambda_{\max}(P)\cdot\left\Vert x\right\Vert ^{2}.
\]
Dabei bezeichnen $\lambda_{\min}(P)$ bzw. $\lambda_{\max}(P)$ den
kleinsten bzw. größten Eigenwert der symmetrischen Matrix~$P$. Bei
einer positiv definiten Matrix sind alle Eigenwerte positiv~\cite{zeidler2003}.

Die potentielle\footnote{Setzt man eine Funktion $V:\mathcal{M}\to\R$ an, bei der noch nicht
klar ist, ob bzw. unter welchen Bedingungen es sich um eine Ljapunov-Funktion
handelt, so spricht man auch von einem \emph{Kandidaten} für eine
Ljapunov-Funktion (engl. \emph{candidate Lyapunov function}).} (strenge) Ljapunov-Funktion für System~(\ref{eq:anhang-LTI-ohne-erregung})
setzt man als quadratische Form $V(x)=x^{T}Px$ mit einer noch zu
bestimmenden positiv definiten Matrix $P\in\R^{n\times n}$ an. Für
eine positiv definite Matrix $P\succ0$ ist die Funktion~$V$ global
positiv definit. Die Zeitableitung von~$V$ entlang~(\ref{eq:anhang-LTI-ohne-erregung})
besitzt die Form
\begin{equation}
\begin{array}{ccl}
\dot{V}(x) & = & \frac{\d}{\d t}\left(x^{T}Px\right)\\
 & = & \dot{x}^{T}Px+x^{T}P\dot{x}\\
 & = & x^{T}\left(A^{T}P+PA\right)x\\
 & = & -x^{T}Qx.
\end{array}\label{eq:dV-LTI}
\end{equation}
Die letzte Zeile definiert eine Matrix $Q\in\R^{n\times n}$. Im Fall
einer positiv definiten Matrix~$Q$ ist die Bedingung~(\ref{eq:Ljap4})
für alle $x\in\R^{n}$ erfüllt, so dass die Ruhelage $x=0$ (global)
asymptotisch stabil\footnote{Bei einem linearen System~(\ref{eq:anhang-LTI-ohne-erregung}) sind
lokale und globale asymptotische Stabilität gleichwertig.} ist. Die Matrix~$P$ erhält man nach~(\ref{eq:dV-LTI}) aus der
(matrixwertigen) \emph{Ljapunov-Gleichung}\index{Ljapunov-Gleichung}
\begin{equation}
A^{T}P+PA+Q=0.\label{eq:anhang-Lyap-Gl}
\end{equation}
Damit ist folgende Stabilitätsaussage möglich~\cite[Abschnitt~{16.5}]{gantmacher86}:
\begin{theorem}
\label{thm:Lyapunov-Stabilitaet-LTI}Die Ruhelage $x^{0}=0$ des linearen
Systems~(\ref{eq:anhang-LTI-ohne-erregung}) ist genau dann asymptotisch
stabil, wenn für eine beliebige Matrix $Q\succ0$ die Ljapunov-Gleichung~(\ref{eq:anhang-Lyap-Gl})
eine eindeutig bestimmte positiv definite Lösung $P\succ0$ besitzt. 
\end{theorem}
\begin{svmultproof2}
\hinreichend\ Die Ruhelage $x^{0}=0$ sei asymptotisch stabil. Dann
ist die Matrix~$A$ stabil. Dadurch existiert das Integral
\begin{equation}
P:=\int_{0}^{\infty}\e^{A^{T}t}Q\e^{At}\d t.\label{eq:anh-LTI-Integral}
\end{equation}
Ferner gelte $Q\succ0$. Daraus folgt $P\succ0$ für~(\ref{eq:anh-LTI-Integral}).
Mit $\e^{0}=I$ und $\e^{At}\to0$ für $t\to\infty$ gilt außerdem
\begin{eqnarray*}
-Q & = & \int_{0}^{\infty}\frac{\d\left[\e^{A^{T}t}Q\e^{At}\right]}{\d t}\,\d t\\
 & = & \int_{0}^{\infty}\left[A^{T}\e^{A^{T}t}Q\e^{At}+\e^{A^{T}t}Q\e^{At}A\right]\d t\\
 & = & A^{T}\int_{0}^{\infty}\e^{A^{T}t}Q\e^{At}\d t+\int_{0}^{\infty}\e^{A^{T}t}Q\e^{At}\d t\,A\\
 & = & A^{T}P+PA,
\end{eqnarray*}
so dass die Ljapunov-Gleichung~(\ref{eq:anhang-Lyap-Gl}) eine positiv
definite Lösung besitzt. Die Eindeutigkeit zeigt man durch einen Vergleich
zweier Lösungen.

\notwendig\ Für eine beliebige Matrix $Q\succ0$ habe die Ljapunov-Gleichung~(\ref{eq:anhang-Lyap-Gl})
eine positiv definite Lösung $P\succ0$. Dann ist die Funktion $V(x)=x^{T}Px$
global positiv definit. Die Zeitableitung $\dot{V}(x)=-x^{T}Qx$ nach
Gl.~(\ref{eq:dV-LTI}) ist wegen $Q\succ0$ global negativ definit.
Nach Satz~\ref{thm:Lyapunov-Theorem} ist die Ruhelage $x^{0}=0$
global asymptotisch stabil.
\end{svmultproof2}

Für den Stabilitätsnachweis mittels Ljapunov-Gleichung~(\ref{eq:anhang-Lyap-Gl})
kann man ohne Einschränkung $Q=I_{n}$ wählen.
\begin{remark}
Die Ljapunov-Gleichung~(\ref{eq:anhang-Lyap-Gl}) stellt ein lineares
Gleichungssystem dar, welches man in die Form 
\begin{equation}
\left(I\otimes A^{T}+A^{T}\otimes I\right)\VEC(P)=-\VEC(Q)\label{eq:anhang-lyap-gleichung-kroneckerformulierung}
\end{equation}
überführen kann~\cite{marcusminc1992}. Dabei bezeichnet $\otimes$
das Kronecker-Tensor-Produkt. Mit $\VEC(\cdot)$ ordnet man die Spalten
einer $n\times n$-Matrix übereinander zu einem Vektor mit $n^{2}$
Einträgen an. Nach dem Lösen von Gl.~(\ref{eq:anhang-lyap-gleichung-kroneckerformulierung})
arrangiert man den Vektor $\VEC(P)$ wieder zu der $n\times n$-Matrix~$P$
um. Alg.~\ref{alg:Loesung-Lyapunov-Gleichung} zeigt eine einfache
\textsc{Maxima}-Implementierung dieser Vorgehensweise.
\end{remark}
\begin{algorithm}
\begin{raggedright}
\noindent
%%%%%%%%%%%%%%%
%%% INPUT:
\begin{minipage}[t]{\textwidth}\color{blue}
\begin{verbatim}
Lyap(A,Q):=block([n,I,At,M,Qvec,Pvec,P],
    n:length(A),
    I:identfor(A),
    At:transpose(A),
    M:kronecker_product(I,At)+kronecker_product(At,I),
    Qvec:list_matrix_entries(transpose(Q)),
    Pvec:list_matrix_entries(-invert(M).Qvec),
    P:genmatrix(lambda([i,j],Pvec[i+(j-1)*n]),n,n)
    )$
\end{verbatim}
\end{minipage}

\par\end{raggedright}
\caption{Einfache \textsc{Maxima}-Implementierung zur Lösung der Ljapunov-Gleichung~(\ref{eq:anhang-Lyap-Gl})\label{alg:Loesung-Lyapunov-Gleichung}
auf Basis von Gl.~(\ref{eq:anhang-lyap-gleichung-kroneckerformulierung})}

\end{algorithm}

Existiert für ein System~(\ref{eq:anhang-system-autonom}) eine Ljapunov-Funktion,
so folgt daraus nur die Stabilität der Ruhelage. Unter bestimmten
Umständen kann man allerdings auch mit einer (nicht strengen) Ljapunov-Funktion
die asymptotische Stabilität nachweisen. Dazu wird der Begriff der
Invarianz benötigt. Eine Teilmenge $\mathcal{N\subseteq\mathcal{M}}$
heißt \emph{positiv} bzw. \emph{negativ invariant} bezüglich~(\ref{eq:anhang-system-autonom}),
falls $\varphi_{t}(p)\in\mathcal{N}$ für alle $p\in\mathcal{N}$
und alle $t\geq0$ bzw. $t\leq0$. Das bedeutet, dass jede Lösung
von~(\ref{eq:anhang-system-autonom}), die mit einem Anfangswert
aus der Menge~$\mathcal{N}$ startet, für alle zukünftigen bzw. zurückliegenden
Zeiten in dieser Menge verbleibt. Eine \emph{invariante Menge}\index{invariant!Menge}
ist sowohl positiv als auch negativ invariant. Auf Basis dieser Vorbetrachtungen
kann folgende Stabilitätsaussage getroffen werden \cite{slotine1991,khalil2002}:
\begin{theorem}
[Invarianzprinzip von LaSalle]\label{thm:LaSalle}Die Teilmenge $\mathcal{K}\subset\mathcal{M}$
sei kompakt und positiv invariant bezüglich~(\ref{eq:anhang-system-autonom}).
Sei $V:\mathcal{M}\to\R$ positiv definit mit $L_{f}V(x)\leq0$ für
alle $x\in\mathcal{K}$. Außerdem sei $\mathcal{N}\subseteq\mathcal{Z}$
die größte invariante Teilmenge von 
\[
\mathcal{Z}:=\left\{ x\in\mathcal{K};\;L_{f}V(x)=0\right\} .
\]
Dann strebt jede Lösung von~(\ref{eq:anhang-system-autonom}) mit
einem Anfangswert aus der Menge~$\mathcal{K}$ für $t\to\infty$
zur Menge~$\mathcal{N}$.
\end{theorem}

Die Teilmenge~$\mathcal{K}$ wird oft über eine Subniveaumenge der
Funktion~$V$ konstruiert, d.\,h. für eine Zahl $V_{0}>0$ definiert
man 
\[
\mathcal{K}:=\left\{ x\in\mathcal{M};\;V(x)\leq V_{0}\right\} .
\]
Die Menge~$\mathcal{K}$ ist als Urbild\footnote{Sei $\psi:\mathcal{A}\to\mathcal{B}$ eine Abbildung und $\mathcal{C}\subseteq\mathcal{B}$
eine Teilmenge des Bildbereichs. Das \emph{Urbild} der Menge~$\mathcal{C}$
unter der Abbildung~$\psi$ besteht aus den Elementen des Definitionsbereichs~$\mathcal{A}$,
die durch~$\psi$ in die Menge~$\mathcal{C}$ abgebildet werden:
$\psi^{-1}(\mathcal{C}):=\{x\in\mathcal{A};\,\psi(x)\in\mathcal{C}\}$.} des abgeschlossenen Intervalls $[0,V_{0}]$ auch abgeschlossen. Ist~$\mathcal{K}$
zusätzlich beschränkt, so ist~$\mathcal{K}$ kompakt (Satz von Heine-Borel\index{Satz!von Heine-Borel}).
Mit $L_{f}V(x)\leq0$ für alle $x\in\mathcal{K}$ ist diese Menge
positiv invariant. Schränkt man das System~\ref{eq:anhang-system-autonom}
auf die Menge~$\mathcal{K}$ ein, so ist~$V$ eine Ljapunov-Funktion.
Im Spezialfall $\mathcal{N}=\{0\}$ streben alle Lösungen aus~$\mathcal{K}$
gegen den Ursprung, so dass die Ruhelage $x^{0}=0$ asymptotisch stabil
ist~\cite[{Cor.~2.22}]{sepulchre97}. Generell lässt sich das Invarianzprinzip
von LaSalle allerdings auch in Verbindung mit anderen Attraktoren
(z.\,B. Grenz\-zyklen) nutzen.

\medskip{}

Existiert für System~(\ref{eq:anhang-system-autonom}) eine Ljapunov-Funktion~$V$,
dann ist die Ruhe\-lage $x^{0}=0$ stabil bzw. im Fall einer strengen
Ljapunov-Funktion asymptotisch stabil. Allerdings gilt auch die Umkehrung
(siehe z.\,B.~\cite{massera56}, \cite[Abschnitt~{4.4}]{slotine1991}):
\begin{theorem}
\label{thm:Lyapunov-Converse}Man betrachte die Ruhelage $x=0$ des
Systems~(\ref{eq:anhang-system-autonom}).
\begin{enumerate}
\item Ist die Ruhelage stabil. Dann exisitert eine Ljapunov-Funktion~$V$. 
\item Ist die Ruhelage asymptotisch stabil, dann existiert eine strenge
Ljapunov-Funktion~$V$.
\end{enumerate}
\end{theorem}

\section{Stabilität erregter Systeme\label{sec:Stabilitaet-erregter-Systeme}}

In diesem Abschnitt wird das Stabilitätskonzept auf Systeme mit Eingang
übertragen~\cite{sontag1995ejc,isidori1999,sontag2000,sontag2008,malisoff2009}.
Man betrachte ein System 
\begin{equation}
\dot{x}=F(x,u)\label{eq:anhang-system-mit-eingang}
\end{equation}
mit hinreichend glattem Vektorfeld $F:\R^{n}\times\R^{m}\to\R^{n}$.
Der Punkt $x^{0}=0$ sei für $u=0$ eine Ruhelage, d.\,h. $F(0,0)=0$.
Außerdem sei das Eingangssignal $u:[0,\infty)\to\R^{m}$ bezüglich
der Supremumsnorm beschränkt: 
\[
\|u\|_{\infty}:=\sup_{t\geq0}\|u(t)\|<\infty.
\]
Für ein Eingangssignal~$u(\cdot)$ bezeichne $\varphi_{t}^{u}$ die
allgemeine Lösung von System~(\ref{eq:anhang-system-mit-eingang}). 

Für die Stabilitätsanalyse von Systemen mit Eingang ist es sinnvoll,
eine weitere Klasse von Vergleichsfunktionen\index{Vergleichsfunktion}
einzuführen. Eine Funktion $\beta:[0,\infty)\times[0,\infty)\to[0,\infty)$
gehört zur Klasse~$\mathcal{KL}$ (kurz $\beta\in\mathcal{KL}$),
falls $\beta(\cdot,t)\in\mathcal{K}$ für jedes~$t$ gilt und $\beta(r,\cdot)$
für jedes~$r$ monoton fallend mit $\beta(r,t)\to0$ für $t\to\infty$
ist. Das System~(\ref{eq:anhang-system-mit-eingang}) heißt \emph{eingangs-zustands-stabil}\index{eingangs-zustands-stabil}
(engl. \emph{input-to-state stable}, kurz \emph{ISS}), falls es Funktionen
$\beta\in\mathcal{KL}$ und $\gamma\in\mathcal{K}$ gibt, so dass
für jeden Anfangswert $p\in\R^{n}$ und jedes beschränkte Eingangssignal
$u(\cdot)$ gilt 
\begin{equation}
\forall t\geq0:\quad\|\varphi_{t}^{u}(p)\|\leq\beta\left(\|p\|,t\right)+\gamma\left(\|u\|_{\infty}\right).\label{eq:anhang-def-ISS}
\end{equation}
Ist ein System eingangs-zustands-stabil, dann bleibt eine Trajektorie
bei kleiner Abweichung des Anfangswertes von der Ruhelage und bei
kleinem Eingangssignal für alle Zeiten $t\geq0$ in der Nähe der Ruhelage
$x^{0}=0$. 

Die Bedingung~(\ref{eq:anhang-def-ISS}) lässt sich mittels einer
Verallgemeinerung der direkten Methode von Ljapunov überprüfen. Eine
global positiv definite Funktion~$V$ heißt \emph{ISS-Ljapunov-Funktion}\index{ISS-Ljapunov-Funktion}\index{Ljapunov-Funktion!ISS},
falls Funktionen $\gamma_{3},\gamma_{4}\in\mathcal{K}_{\infty}$ existieren,
so dass gilt
\begin{equation}
\forall x\in\R^{n},\;\forall u\in\R^{m}:\quad\frac{\partial V(x)}{\partial x}F(x,u)\leq-\gamma_{3}\left(\|x\|\right)+\gamma_{4}\left(\|u\|\right).\label{eq:dV-ISS}
\end{equation}
Es besteht folgender Zusammenhang~\cite{sontag95scl}:
\begin{theorem}
\label{them:ISS-Lyapunov}Das System~(\ref{sec:Stabilitaet-erregter-Systeme})
ist genau dann eingangs-zustands-stabil, wenn eine ISS-Ljapunov-Funktion
existiert.
\end{theorem}
Für $u=0$ geht die Bedingung~(\ref{eq:dV-ISS}) mit $f(x):=F(x,0)$
in die Bedingung~(\ref{eq:Ljap4}) über, so dass die Ruhelage $x^{0}=0$
global asymptotisch stabil ist. Die ISS-Ljapunov-Funktion ist dann
auch Ljapunov-Funktion des autonomen Systems~(\ref{eq:anhang-system-autonom}). 

\medskip{}

Als Spezialfall von System~(\ref{eq:anhang-system-mit-eingang})
betrachte man das lineare zeit\-invariante System
\begin{equation}
\dot{x}=A\,x+B\,u\label{eq:anhang-LTI-mit-Eingang}
\end{equation}
mit den Matrizen $A\in\R^{n\times n}$ und $B\in\R^{n\times m}$.
Die Eingangs-Zustands-Stabilität des Systems~(\ref{eq:anhang-LTI-mit-Eingang})
ist äquivalent zur globalen asymptotischen Stabilität der Ruhelage
$x^{0}=0$ des zugehörigen autonomen Systems~(\ref{eq:anhang-LTI-ohne-erregung}):
\begin{theorem}
Das System~(\ref{eq:anhang-LTI-mit-Eingang}) ist genau dann eingangs-zustands-stabil,
wenn die Systemmatrix~$A$ stabil ist.
\end{theorem}
\begin{svmultproof2}
\hinreichend\ System~(\ref{eq:anhang-LTI-mit-Eingang}) sei eingangs-zustands-stabil.
Aus Gl.~(\ref{eq:anhang-def-ISS}) folgt für $u\equiv0$ die asymptotische
Stabilität der Ruhelage $x^{0}=0$ des resultierenden autonomen Systems~(\ref{eq:anhang-LTI-ohne-erregung}).
Damit muss die Matrix~$A$ stabil sein.

\notwendig\ Sei~$A$ eine Hurwitz-Matrix und $u(\cdot)$ beschränkt.
Aus der allgemeinen Lösung
\[
\varphi_{t}^{u}(p)=\e^{At}p+\int_{0}^{t}\e^{A(t-\tau)}Bu(\tau)\,\d\tau.
\]
des Systems~(\ref{eq:anhang-LTI-mit-Eingang}) erhält man unmittelbar
die Abschätzung
\[
\left\Vert \varphi_{t}^{u}(p)\right\Vert \leq\left\Vert \e^{At}\right\Vert \left\Vert p\right\Vert +\left(\left\Vert B\right\Vert \int_{0}^{\infty}\left\Vert \e^{A\tau}\right\Vert \d\tau\right)\left\Vert u\right\Vert _{\infty}.
\]
Mit $\left\Vert \e^{At}\right\Vert \to0$ für $t\to\infty$ ist das
eine Abschätzung der Form~(\ref{eq:anhang-def-ISS}), so dass das
System eingangs-zustands-stabil ist.
\end{svmultproof2}

Zerfällt ein gegebenes Gesamtsystem in zwei Teilsysteme, so kann die
Stabilitätsanalyse mitunter auf die Teilsysteme reduziert werden.
Dazu betrachte man das in Kaskadenstruktur vorliegende System 
\begin{equation}
\begin{array}{lcl}
\dot{x} & = & F(x,u)\\
\dot{z} & = & G(z,x)
\end{array}\label{eq:anhang-kaskade}
\end{equation}
mit $F(0,0)=0$ und $G(0,0)=0$ (Abb.~\ref{fig:Systeme-in-Kaskadenstruktur}).
In~\cite{sontag95iss} wird folgende Stabilitätsaussage getroffen:
\begin{theorem}
\label{thm:ISS-Kaskade}Beide Teilsysteme in~(\ref{eq:anhang-kaskade})
seien eingangs-zustands-stabil bezüglich des jeweils zweiten Arguments.
Dann ist das Gesamtsystem~(\ref{eq:anhang-kaskade}) eingangs-zustands-stabil
bezüglich des Eingangs~$u$. 

\begin{proofsketch}Für das erste Teilsystem von~(\ref{eq:anhang-kaskade})
existiert eine ISS-Ljapunov-Funktion $V=:V_{1}$, so dass Bedingung~(\ref{eq:dV-ISS})
erfüllt ist. Die $\mathcal{K}_{\infty}$-Funktionen der ISS-Ljapunov-Funktion
des zweiten Teilsystems von~(\ref{eq:anhang-kaskade}) können entsprechend
\[
\forall z\in\R^{q},\;\forall x\in\R^{n}:\quad\frac{\partial V_{2}(z)}{\partial z}G(z,x)\leq-\gamma_{5}\left(\|z\|\right)+\frac{1}{2}\gamma_{3}\left(\|x\|\right)
\]
gewählt werden~\cite{sontag95iss}. Dann ist $W(x,z):=V_{1}(x)+V_{2}(z)$
ISS-Ljapunov-Funktion des Gesamtsystems~(\ref{eq:anhang-kaskade})
bezüglich des Eingangs~$u$.\end{proofsketch}
\end{theorem}
\begin{figure}
\begin{centering}
\input{Kaskade1.pdftex_t}
\par\end{centering}
\caption{Systeme in Kaskadenstruktur\label{fig:Systeme-in-Kaskadenstruktur}}

\end{figure}

Für $u\equiv0$ geht System~(\ref{eq:anhang-kaskade}) mit $f(x):=F(x,u)$
in Form

\begin{equation}
\begin{array}{lcl}
\dot{x} & = & f(x)\\
\dot{z} & = & G(z,x)
\end{array}\label{eq:anhang-kaskade2}
\end{equation}
über. Ist die Ruhelage $x^{0}=0$ des ersten Teilsystems global asymptotisch
stabil und das zweite Teilsystem eingangs-zustands-stabil bezüglich~$x$,
dann ist die Ruhelage $(x^{0},z^{0})=(0,0)$ des Gesamtsystems~(\ref{eq:anhang-kaskade2})
global asymptotisch stabil.


\section{Stabilität im geschlossenen Regelkreis\label{sec:Stabilitaet-im-geschlossenen-Regelkreis}}

Zur Stabilitätsanalyse des geschlossenen Regelkreises wird zunächst
für lineare Systeme ein geeignetes Maß zur Beschreibung der Verstärkung
eingeführt~\cite{doyle1990,zhou1998}. Wir betrachten ein lineares
zeit\-invariantes System

\begin{equation}
\dot{x}=A\,x+B\,u,\quad y=C\,x\label{eq:anhang-LTI-mit-Ein-und-Ausgang}
\end{equation}
mit den Matrizen $A\in\R^{n\times n}$, $B\in\R^{n\times m}$, $C\in\R^{q\times n}$.
Das Eingangs-Ausgangs-Verhalten des Systems~(\ref{eq:anhang-LTI-mit-Ein-und-Ausgang})
lässt sich durch die Übertragungsfunktion 
\begin{equation}
G(s):=C\left(sI-A\right)^{-1}B\label{eq:anhang-Uebertragungsfunktion}
\end{equation}
beschreiben~\cite{lunze2007}. Die Systemmatrix~$A$ sei stabil.
Dann besitzen alle Polstellen der Übertragungsfunktion~(\ref{eq:anhang-Uebertragungsfunktion})
einen negativen Real\-teil, so dass die Übertragungsfunktion \emph{eingangs-ausgangs-stabil}
(engl. \emph{bounded-input bounded output stable}, kurz \emph{BIBO
stable}) ist~\cite{lunze2007,reinschke2014buch}. Die Menge der stabilen
Übertragungsfunktionen bildet den \emph{Hardy-Raum}~$\mathcal{H}_{\infty}$\index{Hardy-Raum}
mit der Norm
\begin{equation}
\left\Vert G\right\Vert _{\infty}:=\sup_{\omega\in\R}\left\Vert G(j\omega)\right\Vert _{2}=\sup_{\omega\in\R}\sigma_{\max}\left(G(j\omega)\right).\label{eq:anhang-hinf}
\end{equation}
Dabei wird punktweise (für alle $\omega\in\R$) die von der euklidischen
Norm induzierte Operatornorm $\|G(j\omega)\|_{2}$ der komplexen Matrix~$G(j\omega)$
betrachtet, die durch den größten Singulärwert $\sigma_{\max}(G(j\omega))$
berechnet wird. Im Falle eines Eingrößensystems, also einer skalarwertigen
Übertragungsfunktion~$G(s)$, ist die $\mathcal{H}_{\infty}$-Norm
die kleinste obere Schranke des Amplitudenfrequenzgangs $|G(j\omega)|$
(siehe Abb.~\ref{fig:Norm-Amplitudenfrequenzgang}).

\begin{figure}
\begin{centering}
\resizebox{0.6\textwidth}{!}{\input{Norm_Bode.pdftex_t}}
\par\end{centering}
\caption{Amplitudenfrequenzgang und $\mathcal{H}_{\infty}$-Norm für eine skalare
Übertragungsfunktion~$G$\label{fig:Norm-Amplitudenfrequenzgang}}
\end{figure}

Für eine näherungsweise Bestimmung der $\mathcal{H}_{\infty}$-Norm
kann man den Frequenzgang $G(j\omega)$ an verschiedenen diskreten
Frequenzen $\omega_{1},\ldots,\omega_{N}$ ermitteln und damit folgende
Abschätzung erhalten:
\begin{equation}
\left\Vert G\right\Vert _{\infty}\geq\max_{i=1,\ldots,N}\left\Vert G(j\omega_{i})\right\Vert _{2}\label{eq:anhang-hinf-diskrete-freq}
\end{equation}
Diesen Ansatz nutzt beispielsweise die \textsc{Scilab}-Funktion \hbox{\texttt{h\_norm}}
zur numerischen Berechnung von~(\ref{eq:anhang-hinf}). Mit~(\ref{eq:anhang-hinf-diskrete-freq})
erhält man allerdings nur eine untere Schranke der Norm. Für Stabilitätsaussagen
ist dagegen die nachfolgend angegebene obere Schranke hilfreich~\cite{hinrichsen1990,hinrichsen1990ijc}:
\begin{lemma}
[Real Bounded Lemma]\label{lem:real-bounded-lemma}\index{Lemma!Real Bounded}Gegeben
sei System~(\ref{eq:anhang-LTI-mit-Ein-und-Ausgang}) mit der Übertragungsfunktion~(\ref{eq:anhang-Uebertragungsfunktion}).
Die Systemmatrix~$A$ sei stabil. Für beliebige Zahlen $\rho>0$
sind folgende Aussagen äquivalent:
\begin{enumerate}
\item $\left\Vert G\right\Vert _{\infty}<1/\rho$,
\item Die algebraische Riccati-Gleichung\index{Riccati-Gleichung}
\begin{equation}
A^{T}P+PA+\rho^{2}PBB^{T}P+C^{T}C=0\label{eq:anhang-Ricc1}
\end{equation}
besitzt eine positiv definite Lösung $P\succ0$.
\end{enumerate}
\end{lemma}
Damit ist man in der Lage, die $\mathcal{H}_{\infty}$-Norm über ein
Bisektionsverfahren zu berechnen. 

\bigskip{}

Nach diesen Vorbetrachtungen gehen wir zu dem in Abb.~\ref{fig:Regelkreis-kleine-Kreisverstaerkung}
dargestellten Regelkreis, der aus einem linearen System mit der Übertragungsfunktion~(\ref{eq:anhang-Uebertragungsfunktion})
und einer (möglicherweise zeit\-varianten) Nichtlinearität $g:\R^{q}\times\R\to\R^{m}$
mit $g(0,\cdot)=0$ besteht. Die Abbildung~$N$ genüge der Bedingung
\begin{equation}
\forall y\in\R^{q}\;\forall t\in\R:\quad\left\Vert N(y,t)\right\Vert _{2}\leq\rho\left\Vert y\right\Vert _{2}\label{eq:anhang-verstaerkung-NL}
\end{equation}
für eine Zahl $\rho>0$. Die Bedingung~(\ref{eq:anhang-verstaerkung-NL})
ist beispielsweise dann erfüllt, wenn~$N$ einer Lipschitz-Bedingung
mit der Lipschitz-Konstanten~$\rho$ erfüllt. Für den geschlossenen
Regelkreis ist folgende Stabilitätsaussage möglich~\cite{hinrichsen1986,hinrichsen1989}:

\begin{figure}
\begin{centering}
\input{Regelkreis_kleine_Kreisverstaerkung.pdftex_t}
\par\end{centering}
\caption{Geschlossener Regelkreis\label{fig:Regelkreis-kleine-Kreisverstaerkung}}
\end{figure}

\begin{theorem}
[Satz von der kleinen Kreisverstärkung]\label{thm:Small-Gain-Theorem}\index{Satz!von der kleinen Kreisverstärkung}Man
betrachte den in Abb.~\ref{fig:Regelkreis-kleine-Kreisverstaerkung}
dargestellten Regelkreis. Die Abbildung~$N$ erfülle die Bedingung~(\ref{eq:anhang-verstaerkung-NL})
für eine Zahl $\rho>0$. Die Matrix~$A$ des linearen Teil\-systems~(\ref{eq:anhang-LTI-mit-Ein-und-Ausgang})
sei stabil. Für die Übertragungsfunktion~(\ref{eq:anhang-Uebertragungsfunktion})
gelte $\left\Vert G\right\Vert _{\infty}<1/\rho$. Dann ist die Ruhelage
$x=0$ des resultierenden Systems
\begin{equation}
\dot{x}=A\,x+B\,N(C\,x,t)\label{eq:anhang-system-linear-nichtlinear}
\end{equation}
global asymptotisch stabil.
\end{theorem}
\begin{svmultproof2}
Die Voraussetzungen des Satzes seien erfüllt. Mit $\left\Vert G\right\Vert _{\infty}<1/\rho$
besitzt die algebraische Riccati-Gleichung~(\ref{eq:anhang-Ricc1})
eine positiv definite Lösung (Lemma~\ref{lem:real-bounded-lemma}).
Aus Stetigkeitsgründen hat auch die (gegenüber Gl.~(\ref{eq:anhang-Ricc1})
leicht modifizierte) Riccati-Gleichung 
\begin{equation}
A^{T}P+PA+\rho^{2}PBB^{T}P+C^{T}C+\epsilon I=0\label{eq:anhang-Ricc2}
\end{equation}
für eine hinreichend kleine Zahl $\epsilon>0$ eine positiv definite
Lösung $P\succ0$ (siehe~\cite{rajamani1998}). Mit dieser Matrix~$P$
setzen wir $V(x)=x^{T}Px$ als Kandidaten für eine Ljapunov-Funktion
an. Damit ist die Funktion~$V$ global positiv definit. Mit Gln.~(\ref{eq:anhang-verstaerkung-NL})
und~(\ref{eq:anhang-Ricc2}) ist die Zeitableitung global negativ
definit:
\[
\begin{array}{lcl}
\dot{V} & = & \frac{\d}{\d t}\left(x^{T}Px\right)\\
 & = & \dot{x}^{T}Px+x^{T}P\dot{x}\\
 & = & x^{T}\left(A^{T}P+PA\right)x+2x^{T}PBN(C\,x,t)\\
 & \leq & x^{T}\left(A^{T}P+PA\right)x+\left\Vert 2x^{T}PBN(C\,x,t)\right\Vert \\
 & \leq & x^{T}\left(A^{T}P+PA\right)x+2\rho\left\Vert x^{T}PB\right\Vert \cdot\left\Vert C\,x\right\Vert \\
 & \leq & x^{T}\left(A^{T}P+PA\right)x+\rho\,x^{T}PBB^{T}P+x^{T}C^{T}C\,x\\
 & \leq & x^{T}\left(A^{T}P+PA+\rho^{2}PBB^{T}P+C^{T}C\right)x\\
 & \leq & -\epsilon\,x^{T}x\\
 & \leq & -\epsilon\left\Vert x\right\Vert ^{2}.
\end{array}
\]
Folglich ist die Ruhelage $x^{0}=0$ global asymptotisch stabil.
\end{svmultproof2}

Anstelle der Riccati-Gleichung~(\ref{eq:anhang-Ricc2}) kann man
auch die Riccati-Ungleichung\index{Riccati-Ungleichung} 
\begin{equation}
A^{T}P+PA+\rho^{2}PBB^{T}P+C^{T}C\prec0\label{eq:anhang-Ricc3}
\end{equation}
lösen. Für die numerische Lösung überführt man diese quadratische
Ungleichung~(\ref{eq:anhang-Ricc3}) in die lineare Matrixungleichung
(LMI)\index{LMI} 
\[
\left(\begin{array}{cc}
-A^{T}P+PA-C^{T}C & \rho PB\\
\rho B^{T}P & I
\end{array}\right)\succ0,\quad P\succ0,
\]
siehe Anmerkung~\ref{rem:High-Gain-Beobachter-LMI}.

\medskip{}

Für den Stabilitätsnachweis eines Regelkreises mit der Struktur nach
Abb.~\ref{fig:Regelkreis-kleine-Kreisverstaerkung} kann man anstelle
des Satzes von der kleinen Kreisverstärkung (engl. \emph{Small Gain
Theorem}) auch das Stabilitätskriterium von Popov bzw. das Kreiskriterium
(engl. \emph{Circle Criterion}) her\-an\-ziehen~\cite{slotine1991,sepulchre97,khalil2002,adamy2014}.

\bibliographystyle{babalpha}
\bibliography{dynamic}

