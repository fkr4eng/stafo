\snippet{1}
\chapter{Sets, Logic, Numbers, Relations, Orderings, Graphs, and Functions}
\snippet{2i}


%change odot to circledast----for khatri rao----done

%change circ to odot-----for Schur prod

%change bullet to circ-----for composition






%$f\bulled g$


\indent\indent In this chapter we review basic terminology
and results concerning sets, logic, numbers, relations, orderings, graphs, and functions. This material is used throughout the book.





%\cite{CraWal93}%%%%%%%%%%%%%%%%%%%



\snippet{3}
\section{Sets}


\snippet{4}
A {\it set} $\{x,y,\ldots\}$ is a collection of elements.
A set can include either a finite or infinite number of elements.
The set $\SX$ is {\it finite} if it has a finite number of elements; otherwise, $\SX$ is {\it infinite}.
%
\snippet{5}
The set $\SX$ is {\it countably infinite} if $\SX$ is infinite and its elements are in one-to-one correspondence with the positive integers. The set $\SX$ is {\it countable} if it is either finite or countably infinite.

\snippet{6}
Let $\SX$ be a set.
Then, \begin{align}x\in \SX\end{align} means that $x$ is an {\it element}
\label{insym}%
\index{element!definition}%
of $\SX$. If $w$ is not an element of $\SX$, then we write
\begin{align}w\not\in \SX.\end{align}
\label{notinsym}

\snippet{7}
No set can be an element of itself.  Therefore, there does not exist a set that includes every set.  The set with no elements, denoted by $\varnothing\mspace{-1mu},$ is the {\it empty set}.
%
\label{varnothingsym}%
\index{empty set!definition}%
\index{nonempty set!definition}%
If $\SX\not=\varnothing\mspace{-1mu},$ then $\SX$ is {\it nonempty}.

\snippet{8}
Let $\SX$ and $\SY$ be sets. The {\it intersection}
\index{intersection!definition}%
of $\SX$ and $\SY$ is the set of common elements of $\SX$ and
$\SY$, which is given by
\begin{align}
\SX\cap \SY
%
\isdef\{x\mspace{-1mu}\colon\ x\in \SX \mbox{ and } x\in \SY \}
%
=  \{ x\in \SX \mspace{-2mu}\colon\ x\in \SY \}
%
=\{x\in \SY\mspace{-2mu}\colon\ x\in \SX \}
%
= \SY \cap \SX,\end{align}

\label{capsym}%
%
\snippet{9}
The {\it union} of $\SX $ and $\SY$ is the set of
elements in either $\SX $ or $\SY$, which is the set
\index{union!definition}%
%
\begin{align}\SX \cup \SY \isdef\{x\mspace{-1mu}\colon\ x\in \SX \mbox{ or } x\in \SY \}=\SY \cup
\SX.\end{align}
\label{cupsym}%
%
%
\snippet{10}
The {\it complement}
\label{backslashsym}%
\index{relative complement!definition}%
\index{$\SY\backslash\SX$!relative complement!definition}%
of  $\SX$ {\it relative} to $\SY$ is
%
\begin{align}\SY  \backslash\SX \isdef \{x \in \SY
\mspace{-2mu}\colon\ x \not\in \SX \}.\end{align}
%
\label{simsym}%
\index{complement!definition}%
\index{$\SX^\sim$!complement!definition}%
%
\snippet{11}
If  $\SY$ is specified, then the {\it complement} of $\SX$ is
%
\begin{align}\SX^\sim \isdef \SY \backslash  \SX.\end{align}
%
\snippet{12}
The {\it symmetric difference}
\index{symmetric difference!definition}%
of $\SX$ and $\SY$ is the set of elements that are in either $\SX$ or $\SY$ but not both, which is given by
%
\begin{align}
\SX\ominus \SY
%
\isdef (\SX \cup \SY)\backslash(\SX \cap \SY).
%
\end{align}
\label{symdiffsym}%
%
\snippet{13}
If $x\in \SX $ implies that $x\in \SY\mspace{-1mu},$ then
%
$\SX$ is a {\it subset}
\label{subseteqsym}%
\index{subset!definition}%
of $\SY$ (equivalently, $\SY$ {\it contains} $\SX$), which is written as
\index{contains!definition}%
%
\begin{equation}
\SX \subseteq\SY\mspace{-1mu}.\end{equation}
%
\snippet{14}
Equivalently,
%
\begin{equation}
\SY \supseteq\SX\mspace{-1mu}.\end{equation}
%
\snippet{15}
Note that $\SX \subseteq\SY$ if and only if $\SX\backslash\SY=\varnothing.$
%
\snippet{16}
Furthermore, $\SX =\SY$ if and only if
$\SX \subseteq \SY $ and $\SY \subseteq \SX $. If $\SX \subseteq
\SY $ and $\SX \not= \SY $, then $\SX$ is a {\it proper subset}
\label{subsetneqsym}%
\index{proper subset!definition}%
of $\SY$ and we write $\SX \subset   \SY $.
%
\snippet{17}
The sets $\SX$ and
$\SY$ are {\it disjoint}
\index{disjoint!definition}%
if $\SX \cap \SY =\varnothing\mspace{-1mu}.$
%
\snippet{18}
A {\it partition}
\index{partition!definition}%
of $\SX$ is a set of pairwise-disjoint and nonempty subsets of $\SX$
whose union is equal to $\SX$.

\snippet{19}
The symbols $\BBN$, $\BBP,$ $\BBZ$, $\BBQ,$ and $\BBR$ denote the
\label{BBNsym}%
\label{BBPsym}%
\label{BBZsym}%
\label{BBQsym}%
\label{BBRsym}%
\index{$\BBZ$!real numbers!definition}%
\index{$\BBN$!real numbers!definition}%
\index{$\BBP$!real numbers!definition}%
\index{$\BBQ$!real numbers!definition}%
\index{$\BBR$!real numbers!definition}%
sets of nonnegative integers, positive integers, integers, rational numbers, and real numbers,
respectively.





\snippet{20}
A set cannot have repeated elements.
Therefore, $\{x,x\}=\{x\}.$
%
\snippet{21}
A {\it multiset}
\index{multiset!definition}%
is a finite collection of elements that allows for repetition.  The
multiset consisting of two copies of $x$ is written as
\label{multisetsym}%
$\{x,x\}_{\rmms}$.
\snippet{22i}
For example, the roots of the polynomial $p(x) = (x-1)^2$ are the elements of the multiset $\{1,1\}_{\mspace{-1mu}_\rmms},$
while the prime factors of 72 are the elements of the multiset $\{2,2,2,3,3\}_\rmms.$

\snippet{23}
The operations ``$\cap,$'' ``$\cup,$''
``$\mspace{1mu}\backslash,$'' ``$\ominus,$'' and ``$\times$'' and the relations ``$\subset$''
and ``$\subseteq$'' extend to multisets.
\snippet{24}
For example,
\begin{align}
\{x,x\}_{\rmms}\cup\{x\}_{\rmms}
=\{x,x,x\}_{\rmms}.\end{align}
\snippet{25}
By ignoring
repetitions, a multiset can be converted to a set, while a set can
be viewed as  a multiset with distinct elements.

\snippet{26}
The {\it Cartesian product}
\index{Cartesian product!definition}%
$\SX_1\mspace{-2mu} \times\cdots \times \SX_n$ of sets
$\SX_1\mspace{-1mu},\ldots,\SX_n$ is the set consisting of {\it
tuples}
\index{tuple!definition}%
\label{tuplesym}%
of the form $(x_1,\ldots, x_n)$, where, for all $i\in\{1,\ldots,n\},$ $x_i\in \SX_i.$
A tuple with $n$ components is an $n$-{\it tuple}.
\index{$n$-tuple!definition}%
\snippet{27}
The components of a tuple are ordered but need not be distinct.  Therefore, a tuple can be viewed as an ordered multiset.
\snippet{28}
We thus write
%
\begin{align}
(x_1,\ldots, x_n)\in{\textstyle\varprod_{i=1}^n} \SX_i \isdef \SX_1\mspace{-2mu} \times\cdots \times \SX_n.
\end{align}
%
$\SX^n$ denotes $\varprod_{i=1}^n \SX.$
\label{cartprodsym}%

\snippet{29}
\begin{defin}  \label{defin:nine5} {\rm
%
\index{sequence!definition!Definition \ref{defin:nine5}}%
\label{sequencesym}%
%
A {\it sequence} $(x_i)^{\infty}_{i=1} = (x_1,x_2,\ldots)$ is a tuple with a
countably infinite number of components. Now, let $i_1 < i_2 < \cdots.$ Then, $(x_{i_j} )^{\infty}_{j=1}$ is a {\it subsequence} of
$(x_i)^{\infty}_{i=1}.$
%
}\end{defin}


\snippet{30i}
Let $\SX$ be a set, and let $X\isdef(x_i)_{i=1}^\infty$ be a sequence whose components are elements of $\SX;$ that is, $\{x_1,x_2,\ldots\}\subseteq\SX.$
%
For convenience, we write either $X\subseteq \SX$ or $X\subset \SX,$  where $X$ is viewed as a set and the multiplicity of the components of the sequence is ignored.
%
\snippet{31}
For sequences $X,Y\subset\BBF^n,$ define $X + Y\isdef (x_i + y_i)_{i=1}^\infty$ and $X\odot Y\isdef (x_i\odot y_i)_{i=1}^\infty,$ where ``$\odot$'' denotes component-wise multiplication.  In the case $n=1,$ we define $XY\isdef (x_i y_i)_{i=1}^\infty.$

\snippet{32i}











%
%just defins of sum and product of sequences

%sequence componentwise multiplication!  AX = {Aixi)

\snippet{33}
\section{Logic}


\snippet{34}
Every {\it statement} is either true or false, and no statement is both true and false.
\snippet{35i}
\index{statement!definition}%
%
%
%{\it Logic} is a collection of rules used to analyze statements.
%\index{logic!definition}%
%
\snippet{36}
A {\it proof} is a collection of statements that verify that a statement is true.
\index{proof!definition}%
%
\snippet{37}
A {\it conjecture} is a statement that is believed to be true but whose proof is not known.
\index{conjecture!definition}%
%
%An {\it assumption} is a true statement whose truth is not subject to proof.----hypothesis deals with this
%


\snippet{38}
Let $A$ and $B$ be statements.  %Then, $A$ {\it holds} if $A$ is true.
\index{holds!definition}%
%
\snippet{39}
The {\it not} of $A$ is the
statement $(\mbox{not } A),$ the {\it and} of $A$ and $B$ is the
statement
%
\index{not!definition}%
\index{and!definition}%
\index{or!definition}%
\index{inclusive or!definition}%
\index{xor!definition}%
\index{exclusive or!definition}%
%
$(A\mbox{ and } B),$ and the {\it or} of $A$ and $B$ is the
statement $(A\mbox{ or } B).$
\snippet{40}
The statement $(A\mbox{ or } B)$ does
not contradict the statement $(A\mbox{ and } B);$ hence, the word ``or'' is
inclusive.
\snippet{41}
The {\it exclusive or} of $A$ and $B$ is the statement $(A\mbox{ xor } B),$
which is $[(A\mbox{ and not } B) \mbox{ or }(B\mbox{ and not } A)].$
%
\snippet{42}
Equivalently, $(A\mbox{ xor } B)$ is the statement [$(A\mbox{ or } B)\mbox{ and not}(A\mbox{ and }B)]$, that is, $A$ or $B,$ but not both.
%
\snippet{43}
Note that $(A\mbox{ and } B)= (B\mbox{ and } A),$  $(A\mbox{ or } B)= (B\mbox{ or } A),$ and $(A\mbox{ xor } B)= (B\mbox{ xor } A).$


\snippet{44i}
Let $A$, $B,$ and $C$ be statements.  Then, the statements ($A$ and $B$ or $C$) and ($A$ or $B$ and $C$) are
ambiguous. For clarity, we thus write, for example, [$A$ and ($B$ or $C$)] and [$A$ or ($B$ and $C$)].  In words, we write
``$A$ and either $B$ or $C$'' and ``$A$ or both $B$ and $C$,'' respectively, where ``either'' and ``both'' signify parentheses.
%
\snippet{45i}
Furthermore,
%
%
%
\begin{gather}
%
(A\mbox{ and } B) \mbox{ or } C = (A\mbox{ and } C)\mbox{ or }(B\mbox{ and }C),\\
%
(A\mbox{ or } B) \mbox{ and } C = (A\mbox{ or } C)\mbox{ and }(B\mbox{ or }C).
%
\end{gather}


%does xor distribute???

%could be 4 more equalities

%also and/and, or/or, xor/xor






\snippet{46}
Let $A$ be a statement.  To analyze statements involving logic operators, define $\truth(A) = 1$ if $A$ is true, and $\truth(A) = 0$ if $A$ is false.  Then,
%
%
\begin{align}
%
\truth(\mbox{not }A) = \truth(A)+1,
%
\end{align}
%
\snippet{47}
where $0+0=0,$ $1+0 = 0+1 = 1,$ and $1+1=0.$
%
\snippet{48i}
Therefore, $A$ is true if and only if $(\mbox{not }A)$ is false, while $A$ is false if and only if $(\mbox{not }A)$ is true.
%
\snippet{49i}
Note that
%
%
\begin{align*}
%
\truth[\mbox{not}(\mbox{not }A)] &= \truth(\mbox{not }A)+1\\
%
&= [\truth(A)+1]+1\\
%
&= \truth(A).
%
\end{align*}
%
\snippet{50}
Furthermore, note that $\truth(A)+\truth(A)=0$ and $\truth(A)\truth(A)=\truth(A).$




\snippet{51}
Let $A$ and $B$ be statements.
%
%
Then,
%
%
\begin{gather}
%
\truth(A\mbox{ and }B) = \truth(A)\truth(B),\\
%
%
\truth(A\mbox{ or }B) = \truth(A)\truth(B)+\truth(A)+\truth(B),\\
%
%
\truth(A\mbox{ xor }B) = \truth(A)+\truth(B).
%
\end{gather}
%
\snippet{52}
Hence,
%
\begin{gather}
%
\truth(A\mbox{ and }B) = \min\,\{\truth(A),\truth(B)\},\\
%
%
\truth(A\mbox{ or }B) = \max\,\{\truth(A),\truth(B)\}.
%
\end{gather}
\snippet{53i}
%
%Consequently, if $A$ and $B$ are true, then $(A\mbox{ and } B)$ and
%$(A\mbox{ or } B)$ are true;
%
%if $A$ is true and $B$ is false, then $(A\mbox{ and } B)$ is false and
%$(A\mbox{ or } B)$ is true;
%
%if $A$ is false and $B$ is true, then $(A\mbox{ and } B)$ is false and
%$(A\mbox{ or } B)$ is true;
%
%and, if $A$ and $B$ are false, then $(A\mbox{ and } B)$ and
%$(A\mbox{ or } B)$ are false.
%
%Note that $(A\mbox{ and } B)$ is true if and only if $(B\mbox{ and } A)$ is true, and that
%$(A\mbox{ or } B)$ is true if and only if $(B\mbox{ or } A)$ is true.
%
\snippet{54i}
Consequently, $\truth(A\mbox{ and }B) = \truth(B\mbox{ and }A),$ $\truth(A\mbox{ or }B) = \truth(B\mbox{ or }A),$ and $\truth(A\mbox{ xor }B) = \truth(B\mbox{ xor }A).$
%
\snippet{55}
Furthermore,
%
$\truth(A\mbox{ and }A) = \truth(A\mbox{ or }A) =  \truth(A),$ and
%
%
%
$\truth(A\mbox{ xor }A) = 0.$
%







\snippet{56}
Let $A$ and $B$ be statements.  The {\it implication}
\index{implication!definition}%
%
$(A\Longrightarrow B)$ is the statement $[(\mbox{not }A) \mbox{ or }B].$
%
\snippet{57}
Therefore,
%
\begin{align}
%
\truth(A\Longrightarrow B) = \truth(A)\truth(B) + \truth(A)+1.\label{truthAimpliesB}
%
\end{align}
%
\snippet{58}
The implication $(A\Longrightarrow B)$ is read as either ``if $A$, then $B$,'' ``if $A$ holds, then $B$ holds,'' or
``$A$ implies $B\mspace{-1mu}.$''
\snippet{59}
The statement $A$ is the {\it hypothesis}, while the statement $B$ is the {\it conclusion}.
\index{hypothesis!definition}%
\index{conclusion!definition}%
%
\snippet{60}
If $(A\Longrightarrow B),$  then $A$ is a {\it sufficient condition} for $B,$ and $B$ is a {\it necessary condition} for $A.$
\index{sufficient condition!definition}%
\index{necessary condition!definition}%
%
%
\snippet{61}
It follows from (\ref{truthAimpliesB}) that,
%
if $A$ and $B$ are true, then $(A\Longrightarrow B)$ is true;
%
if $A$ is true and $B$ is false, then $(A\Longrightarrow B)$ is false;
%
and, if $A$ is false, then $(A\Longrightarrow B)$ is true whether or not $B$ is true.
%
\snippet{62i}
For example, both implications [$(2+2=5) \Longrightarrow (3+3=6)$] and [$(2+2=5) \Longrightarrow (3+3=8)$] are true.
%
\snippet{63}
Finally, note that $[(A\Longrightarrow B)\mbox{ and } A] = A \mbox{ and } B.$

\snippet{64i}
%how do we know this?  truth[A(x)] = truth[B(x)] for all x implies A = B?  No, doubt it

%how do we show 2 statements are equal withOUT Venn diagrams???













\snippet{65}
A {\it predicate} is a statement that depends on a variable.  Let $\SX$ be a set, let $x\in\SX,$ and let $A(x)$ be a predicate.
%
%Then, for each $x\in\SX,$ $A(x)$ is a statement.
%
%The predicate $A(x)$ may be true for some values of $x\in\SX$ and false for other values of $x\in\SX.$
\index{predicate!definition}%
%
\snippet{66}
There are two ways to use a predicate to create a statement.   An {\it existential statement}
\index{existential statement!definition}%
has the form
\begin{align}\mbox{there exists}\  x\in \SX \mbox{ such that } A(x) \mbox{ holds,} \label{0.1}
\end{align} whereas a {\it universal statement}
\index{universal statement!definition}%
has the form
\begin{equation}\mbox{for all } x\in \SX,  A(x) \mbox{ holds.}\label{0.2X}
\end{equation}
\snippet{67i}
%or, equivalently,
%
%
%\begin{equation} A(x)   \mbox{ holds for all } x\in \SX.\label{0.2}
%\end{equation}
%
\snippet{68}
Note that   %$[\mbox{for all }x\in\SX, A(x)\mbox{ holds}]$ is the statement whose truth is given by
%
\begin{gather}
%
%
\truth[\mbox{there exists }x\in\SX \mbox{ such that } A(x) \mbox{ holds}] = \max_{x\in\SX} \truth[A(x)],\\
%
\truth[\mbox{for all }x\in\SX, A(x)\mbox{ holds}] = \min_{x\in\SX} \truth[A(x)].       %min is and
%
%\end{align}
%
%while $[\mbox{there exists }x\in\SX \mbox{ such that } A(x)\mbox{ holds}]$ is the statement whose truth is given by
%
%\begin{align}          %max is or
%
\end{gather}
%




\snippet{69}
An {\it argument} is an implication whose hypothesis and conclusion are predicates that depend on the same variable.
In particular, letting $x$ denote a variable, and letting $A(x)$ and $B(x)$ be predicates, the implication $[A(x)\Longrightarrow B(x)]$ is an argument.
\index{argument!definition}%
%
\snippet{70}
For example, for each real number $x,$ the implication [$(x=1)\Longrightarrow (x+1=2)$] is an argument.
\snippet{71}
Note that the variable $x$ links the hypothesis and the conclusion, thereby making this implication useful for the purpose of {\it inference}.
\index{inference!definition}%
In particular, for all real numbers $x,$ $\truth[(x=1)\Longrightarrow (x+1=2)]=1.$  The statements $(\mbox{for all }x, [A(x)\Longrightarrow B(x)]\mbox{ holds})$ and $(\mbox{there exists }x \mbox{ such that } [A(x)\Longrightarrow B(x)]\mbox{ holds})$ are inferences.





\snippet{72}
Let $A$ and $B$ be statements.
The {\it bidirectional implication} $(A\Longleftrightarrow B)$ is the statement
\index{bidirectional implication!definition}%
$[(A\Longrightarrow B) \mbox{ and } (A\Longleftarrow B)],$  where
$(A\Longleftarrow B)$ means $(B\Longrightarrow A)$.
%
If $(A\Longleftrightarrow B)$, then $A$ and $B$ are {\it equivalent}.
%
%
\snippet{73}
Furthermore,
%
\begin{align}
%
\truth(A\Longleftrightarrow B) = \truth(A) + \truth(B) + 1.
%
\end{align}
%
\snippet{74i}
Therefore, $A$ and $B$ are equivalent if and only if either both $A$ and $B$
are true or both $A$ and $B$ are false.
\snippet{75i}


%Note that
%
%\begin{align}
%
%\truth(A\Longrightarrow B)  = \truth[(\mbox{not }A)\mbox{ or } B].
%
%\end{align}
%
%Therefore, $(A\Longrightarrow B)$ is equivalent to $[(\mbox{not }A)\mbox{ or } B].$










\snippet{76}
Let $A$ and $B$ be statements, and assume that $(A\Longleftrightarrow B).$
%
Then, $A$ holds {\it if and only if} $B$ holds.  The
implication $A\Longrightarrow B$ (the ``only if'' part) is {\it necessity},
\index{necessity!definition}%
\index{sufficiency!definition}%
while $B\Longrightarrow A$ (the ``if'' part) is {\it sufficiency}.







\snippet{77}
Let $A$ and $B$ be statements.  The {\it converse}
\index{converse!definition}%
of $(A\Longrightarrow B)$ is $(B \Longrightarrow A)$.
\snippet{78}
Note that
%
\begin{align*}
%
(A\Longrightarrow B) &\Longleftrightarrow [(\mbox{not }A)\mbox{ or } B]\\
%
&\Longleftrightarrow [(\mbox{not }A) \mbox{ or } \mbox{not}(\mbox{not }B)]\\
%
&\Longleftrightarrow [ \mbox{not}(\mbox{not }B) \mbox{ or } \mbox{not }A ]\\
%
%&\Longleftrightarrow \mbox{not}[\mbox{not }A \mbox{ and not} B]\\
%
&\Longleftrightarrow(\mbox{not }B\Longrightarrow \mbox{not }A).
%
\end{align*}
%
\snippet{79}
Therefore, the statement
$(A\Longrightarrow B)$ is equivalent to its {\it contrapositive}
\index{contrapositive!definition}%
[(not $B$)  $\Longrightarrow$ (not $A$)].


\snippet{80}
Let $A$, $B$, $A',$ and $B'$ be statements, and assume that $(A^{\mspace{-1mu}\prime}\Longrightarrow A \Longrightarrow B\Longrightarrow B^{\prime}).$ Then, $(A^{\mspace{-1mu}\prime}
\Longrightarrow B^{\prime})$ is a {\it corollary} of $(A\Longrightarrow
B).$

\snippet{81}
Let $A$, $B,$ and $A'$ be statements, and assume that
$A\Longrightarrow B\mspace{-1mu}.$ Then, $(A\Longrightarrow B)$ is a
{\it strengthening} of $[(A\mbox{ and
}A')\Longrightarrow B].$
\index{strengthening!definition}%
\index{redundant assumptions!definition}%
If, in addition, $(A\Longrightarrow A'),$ then the
statement $[(A\mbox{ and }A')\Longrightarrow B]$ has a {\it redundant
assumption}.

\snippet{82}
An {\it interpretation} is a feasible assignment of true or false to all statements that comprise a statement.
\snippet{83}
For example, there are four interpretations of  the statement $(A\mbox{ and }B)$, depending on whether $A$ is assigned to be true or false and $B$ is assigned to be true or false.
%
Likewise, $[(x=1)\mbox{ and }(x=2)]$ has three interpretations, which depend on the value of $x.$




\snippet{84}
Let $A_1,A_2,\ldots$ be statements, and let $B$ be a statement that depends on $A_1,A_2,\ldots$   Then, $B$ is a {\it tautology} if $B$ is true whether or not $A_1,A_2,\ldots$ are true.
%
\index{tautology!definition}%
%
\snippet{85}
For example, let $B$ denote the statement $(A\mbox{ or not } A).$  Then,
\begin{align}
%
\truth(A\mbox{ or not } A) = 1,
%
\end{align}
%
and thus the statement $(A\mbox{ or not } A)$ is true whether or not $A$ is true.  Hence, $(A\mbox{ or not } A)$ is a tautology.
%
%Furthermore, $\truth(A\mbox{ or not }A)=1$, and thus this statement is a tautology.
%
Likewise, %the implication $(A\Longrightarrow B)$ is a tautology if it is true whether or not $A$ and $B$ are true.
%
%For example,
%
$(A\Longrightarrow A)$ is a tautology.
%
Furthermore, since
%
\begin{align}
%
\truth[(A\mbox{ and } B)\Longrightarrow A] = \truth(A)^2\truth(B)+\truth(A)\truth(B)+1=1,
%
\end{align}
%
it follows that $[(A\mbox{ and } B)\Longrightarrow A]$ is a tautology.
%
\snippet{86}
Likewise, $\truth([A\mbox{ and not }A]\Longrightarrow B)=1,$ and thus $([A\mbox{ and not }A]\Longrightarrow B)$ is a tautology.
\snippet{87i}
%
%
%
%



%when is an argument a tautology?
%I am not sure an argument can be a tautology!  This is just a true contingency
%For each real number $x,$ $\truth[(x=1)\Longrightarrow (x+1=2)]=1,$ and thus the argument
%
%[$(x=1)\Longrightarrow (x+1=2)$] is a tautology.




\snippet{88}
Let $A_1,A_2,\ldots$ be statements, and let $B$ be a statement that depends on $A_1,A_2,\ldots$   Then, $B$ is a {\it contradiction} if $B$ is false whether or not $A_1,A_2,\ldots$ are true.
%
\index{contradiction!definition}%
%
\snippet{89}
For example, let $B$ denote the statement $(A\mbox{ and not } A).$  Then,
%
\begin{align}
%
\truth(A\mbox{ and not } A) = 0,
%
\end{align}
%
and thus the statement $(A\mbox{ and not } A)$ is false whether or not $A$ is true.    Hence, $(A\mbox{ and not } A)$ is a contradiction.
\snippet{90i}
%
%






%This is just a false statement-----its truth is zero----but it is not structurally false
%As another example, for each real number $x,$ the argument $[(x=1)\mbox{ and }(x=2)]$ is a contradiction.
%
%
%
%%%%%%%%%%%%%  too complicated
%Likewise, if $A'$ is a statement, then
%
%
%
%\begin{align}
%
%\truth([A\mbox{ or not }A]\Longrightarrow [A'\mbox{ and not }A'])=0,
%
%\end{align}
%
%and thus $B=([A\mbox{ or not }A]\Longrightarrow [A'\mbox{ and not }A'])$ is a contradiction.
%%%%%%%%%%%%%%%%%%%%%%%%
%
%
%
%
%As a final example, for each real number $x,$  $\truth[(x=x)\Longrightarrow (x<x)]=0,$ and thus the argument
%
%[$(x=x)\Longrightarrow (x<x)$] is a contradiction.
%




\snippet{91}
Let $A$ and $B$ be statements.  If the implication $(A\Longrightarrow B)$ is neither a tautology nor a contradiction, then $\truth(A\Longrightarrow B)$ depends on the truth of the statements that comprise $A$ and $B\mspace{-1mu}.$
\snippet{92}
For example, $\truth(A\Longrightarrow \mbox{not }A)=\truth(A)+1,$ and thus the statement $(A\Longrightarrow \mbox{not }A)$ is true if and only if $A$ is false, and false if and only if $A$ is true.  Hence, $(A\Longrightarrow \mbox{not }A)$ is neither a tautology nor a contradiction.
\snippet{93}
A statement that is neither a tautology nor a contradiction is a {\it contingency}.
\index{contingency!definition}%
%
\snippet{94}
For example, the implication $[A\Longrightarrow (A\mbox{ and }B)]$ is a contingency.  Likewise, for each real number $x,$
%
$\truth[(x=1)\Longrightarrow (x=2)]=\truth(x\ne1),$ and thus the statement
%
[$(x=1)\Longrightarrow (x=2)$] is a contingency.
%






\snippet{95}
An argument that is a contingency is a {\it theorem}, {\it proposition}, {\it corollary}, or {\it lemma}.
%
\snippet{96}
A theorem is a significant result; a proposition is a theorem of less significance.  The primary role of
a lemma is to support the proof of a theorem or a proposition.
\index{theorem!definition}%
\index{proposition!definition}%
\index{lemma!definition}%
\index{corollary!definition}%
\index{fact!definition}%
A {\it corollary} is a consequence of a theorem or a proposition.
A {\it fact} is either a theorem, proposition, lemma, or
corollary.










\snippet{97}
In order to visualize logic operations on predicates, it is helpful to replace statements with sets and logic operations by set operations; the truth of a statement can then be visualized in terms of Venn diagrams.
%
%represent the statement $A$ as a set $\SA,$ where the elements of $\SA$ correspond to ${\rm truth}(A)=1$ and the elements of $\SA^\sim$ correspond to ${\rm truth}(A)=0.$
%
\snippet{98}
To do this, let $\SX$ be a set, for all $x\in\SX,$ let $A(x)$ and $B(x)$ be predicates, and
%
define $\SA\isdef \{x\in\SX\colon {\rm truth}[A(x)] = 1\}$ and $\SB\isdef \{x\in\SX\colon {\rm truth}[B(x)] = 1\}.$
%
%
Then, the logic operations ``and,'' ``or,'' ``xor,'' and ``not'' are equivalent to ``$\cap$,'' ``$\cup$,'' ``$\ominus$,'' and
``${}^\sim$,'' respectively.
%
%
%
%For example, let $A$ and $B$ be statements with corresponding sets $\SA$ and $\SB,$ respectively.
%
\snippet{99}
For example, $\{x\in\SX\colon {\rm truth}[(\mbox{not }A(x)) \mbox{ and }B(x)]=1\} = \SA^\sim \cap\SB.$
%
Furthermore, since $[A(x) \Longrightarrow B(x)]$ is equivalent to $[(\mbox{not }A(x))\mbox{ or }B(x)],$ it follows that $\{x\in\SX\colon {\rm truth}[A(x) \Longrightarrow B(x)]=1\} = \SA^\sim\cup\SB.$
%
Similarly, since $[A(x) \Longleftrightarrow B(x)]$ is equivalent to $[(A(x) \mbox{ or } \mbox{not } B(x)) \mbox{ and } ( [\mbox{not }A(x)]  \mbox{ or }  B(x) )],$ it follows that
%
$\{x\in\SX\colon A(x) \Longleftrightarrow B(x)\} = (\SA \cup\SB^\sim)\cap (\SA^\sim\cup  \SB) = (\SA\cap\SB)\cup(\SA\cup\SB)^\sim.$







\snippet{100}
Now, define $\SX,$ $A(x),$ $B(x)$, $\SA,$ and $\SB$ as in the previous paragraph, and assume that, for all $x\in\SX,$ $A(x)\Longrightarrow B(x).$
%
Therefore, $\SA^\sim\cup\SB= \{x\in\SX\colon{\rm truth}[(\mbox{not }A(x))\mbox{ or }B(x)] = 1\} = \SX,$ and thus
%
$\SA\backslash\SB = (\SA^\sim\cup\SB)^\sim = \{x\in\SX\colon {\rm truth}[(\mbox{not }A(x))\mbox{ or }B(x)] = 0\} = \varnothing.$
%
Consequently, $\SA \subseteq \SB.$
\snippet{101i}
%
%
%Therefore, $\SX = \{x\in\SX\colon {\rm truth}[A(x)\Longrightarrow B(x)] = 1\}.$
%
%$ = \SA^\sim\cup\SB
%
%Now, define $\SX_0\isdef\{x\in\SX\colon {\rm truth}[A(x)] = 1\}.$
%
%Then, $\{x\in\SX_0\colon {\rm truth}[A(x)] = 1\}$ is empty.
%
%Consequently,
%
%Then, ${\rm truth}[(\mbox{not }A)\mbox{ or }B)]=1$ if and only if $\SA\backslash\SB=\varnothing,$ which, in turn, is equivalent to $\SA\subseteq\SB.$
%
%
%
%
\snippet{102}
This means that the logic operator ``$\Longrightarrow$'' is represented by ``$\subseteq$.''
%
\snippet{103}
For example, for all $x\in\SX,$ let $C(x)$ be a predicate, and define $\SC\isdef \{x\in\SX\colon {\rm truth}[C(x)] = 1\}.$
%
Then, for all $x\in\SX,$
%
${\rm truth}[( A(x) \mbox{ and }B(x))\Longrightarrow C(x)]=1$
%
if and only if $\SA\cap\SB \subseteq \SC.$
%
\snippet{104}
Likewise, for all $x\in\SX,$
%
\begin{align}{\rm truth}([\mbox{$A(x)$ and ($B(x)$ or $C(x)$)] $\Longleftrightarrow$ [($A(x)$ and $B(x)$) or ($A(x)$ and $C(x)$)}])=1\end{align}
%
if and only if
%
\begin{align}\SA \cap(\SB \cup \SC )=(\SA \cap \SB )\cup(\SA \cap \SC ).\label{seteq234}\end{align}
%
Note that \eqref{seteq234} represents a tautology.
\snippet{105i}


%Finally, $A$ is a tautology if and only if the set $\SA$ representing $A$ is the universal set; $A$ is a contradiction if and only if $\SA$ is the empty set; and $A$ is a contingency if and only if $\SA$ is neither the universal set nor the empty set.









\snippet{106}
\section{Relations and Orderings}
