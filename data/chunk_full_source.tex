% snippet(1)
A {\it set} $\{x,y,\ldots\}$ is a collection of elements.
A set can include either a finite or infinite number of elements.
The set $\SX$ is {\it finite} if it has a finite number of elements; otherwise, $\SX$ is {\it infinite}.
%
% snippet(2)
The set $\SX$ is {\it countably infinite} if $\SX$ is infinite and its elements are in one-to-one correspondence with the positive integers. The set $\SX$ is {\it countable} if it is either finite or countably infinite.

% snippet(3)
Let $\SX$ be a set.
Then, \begin{align}x\in \SX\end{align} means that $x$ is an {\it element}
\label{insym}%
\index{element!definition}%
of $\SX$. If $w$ is not an element of $\SX$, then we write
\begin{align}w\not\in \SX.\end{align}
\label{notinsym}

% snippet(4)
No set can be an element of itself.  Therefore, there does not exist a set that includes every set.  The set with no elements, denoted by $\varnothing\mspace{-1mu},$ is the {\it empty set}.
%
\label{varnothingsym}%
\index{empty set!definition}%
\index{nonempty set!definition}%
If $\SX\not=\varnothing\mspace{-1mu},$ then $\SX$ is {\it nonempty}.

% snippet(5)
Let $\SX$ and $\SY$ be sets. The {\it intersection}
\index{intersection!definition}%
of $\SX$ and $\SY$ is the set of common elements of $\SX$ and
$\SY$, which is given by
\begin{align}
\SX\cap \SY
%
\isdef\{x\mspace{-1mu}\colon\ x\in \SX \mbox{ and } x\in \SY \}
%
=  \{ x\in \SX \mspace{-2mu}\colon\ x\in \SY \}
%
=\{x\in \SY\mspace{-2mu}\colon\ x\in \SX \}
%
= \SY \cap \SX,\end{align}

\label{capsym}%
%
% snippet(6)
The {\it union} of $\SX $ and $\SY$ is the set of
elements in either $\SX $ or $\SY$, which is the set
\index{union!definition}%
%
\begin{align}\SX \cup \SY \isdef\{x\mspace{-1mu}\colon\ x\in \SX \mbox{ or } x\in \SY \}=\SY \cup
\SX.\end{align}
\label{cupsym}%
%
%
% snippet(7)
The {\it complement}
\label{backslashsym}%
\index{relative complement!definition}%
\index{$\SY\backslash\SX$!relative complement!definition}%
of  $\SX$ {\it relative} to $\SY$ is
%
\begin{align}\SY  \backslash\SX \isdef \{x \in \SY
\mspace{-2mu}\colon\ x \not\in \SX \}.\end{align}
%
\label{simsym}%
\index{complement!definition}%
\index{$\SX^\sim$!complement!definition}%
%
% snippet(8)
If  $\SY$ is specified, then the {\it complement} of $\SX$ is
%
\begin{align}\SX^\sim \isdef \SY \backslash  \SX.\end{align}
%
% snippet(9)
The {\it symmetric difference}
\index{symmetric difference!definition}%
of $\SX$ and $\SY$ is the set of elements that are in either $\SX$ or $\SY$ but not both, which is given by
%
\begin{align}
\SX\ominus \SY
%
\isdef (\SX \cup \SY)\backslash(\SX \cap \SY).
%
\end{align}
\label{symdiffsym}%
%
% snippet(10)
If $x\in \SX $ implies that $x\in \SY\mspace{-1mu},$ then
%
$\SX$ is a {\it subset}
\label{subseteqsym}%
\index{subset!definition}%
of $\SY$ (equivalently, $\SY$ {\it contains} $\SX$), which is written as
\index{contains!definition}%
%
\begin{equation}
\SX \subseteq\SY\mspace{-1mu}.\end{equation}
%
% snippet(11)
Equivalently,
%
\begin{equation}
\SY \supseteq\SX\mspace{-1mu}.\end{equation}
%
% snippet(12)
Note that $\SX \subseteq\SY$ if and only if $\SX\backslash\SY=\varnothing.$
%
% snippet(13)
Furthermore, $\SX =\SY$ if and only if
$\SX \subseteq \SY $ and $\SY \subseteq \SX $. If $\SX \subseteq
\SY $ and $\SX \not= \SY $, then $\SX$ is a {\it proper subset}
\label{subsetneqsym}%
\index{proper subset!definition}%
of $\SY$ and we write $\SX \subset   \SY $.
%
% snippet(14)
The sets $\SX$ and
$\SY$ are {\it disjoint}
\index{disjoint!definition}%
if $\SX \cap \SY =\varnothing\mspace{-1mu}.$
%
% snippet(15)
A {\it partition}
\index{partition!definition}%
of $\SX$ is a set of pairwise-disjoint and nonempty subsets of $\SX$
whose union is equal to $\SX$.

% snippet(16)
The symbols $\BBN$, $\BBP,$ $\BBZ$, $\BBQ,$ and $\BBR$ denote the
\label{BBNsym}%
\label{BBPsym}%
\label{BBZsym}%
\label{BBQsym}%
\label{BBRsym}%
\index{$\BBZ$!real numbers!definition}%
\index{$\BBN$!real numbers!definition}%
\index{$\BBP$!real numbers!definition}%
\index{$\BBQ$!real numbers!definition}%
\index{$\BBR$!real numbers!definition}%
sets of nonnegative integers, positive integers, integers, rational numbers, and real numbers,
respectively.





% snippet(17)
A set cannot have repeated elements.
Therefore, $\{x,x\}=\{x\}.$
%
% snippet(18)
A {\it multiset}
\index{multiset!definition}%
is a finite collection of elements that allows for repetition.  The
multiset consisting of two copies of $x$ is written as
\label{multisetsym}%
$\{x,x\}_{\rmms}$.
% snippet(19)
For example, the roots of the polynomial $p(x) = (x-1)^2$ are the elements of the multiset $\{1,1\}_{\mspace{-1mu}_\rmms},$
while the prime factors of 72 are the elements of the multiset $\{2,2,2,3,3\}_\rmms.$

% snippet(20)
The operations ``$\cap,$'' ``$\cup,$''
``$\mspace{1mu}\backslash,$'' ``$\ominus,$'' and ``$\times$'' and the relations ``$\subset$''
and ``$\subseteq$'' extend to multisets.
% snippet(21)
For example,
\begin{align}
\{x,x\}_{\rmms}\cup\{x\}_{\rmms}
=\{x,x,x\}_{\rmms}.\end{align}
% snippet(22)
By ignoring
repetitions, a multiset can be converted to a set, while a set can
be viewed as  a multiset with distinct elements.

% snippet(23)
The {\it Cartesian product}
\index{Cartesian product!definition}%
$\SX_1\mspace{-2mu} \times\cdots \times \SX_n$ of sets
$\SX_1\mspace{-1mu},\ldots,\SX_n$ is the set consisting of {\it
tuples}
\index{tuple!definition}%
\label{tuplesym}%
of the form $(x_1,\ldots, x_n)$, where, for all $i\in\{1,\ldots,n\},$ $x_i\in \SX_i.$
A tuple with $n$ components is an $n$-{\it tuple}.
\index{$n$-tuple!definition}%
% snippet(24)
The components of a tuple are ordered but need not be distinct.  Therefore, a tuple can be viewed as an ordered multiset.
% snippet(25)
We thus write
%
\begin{align}
(x_1,\ldots, x_n)\in{\textstyle\varprod_{i=1}^n} \SX_i \isdef \SX_1\mspace{-2mu} \times\cdots \times \SX_n.
\end{align}
%
$\SX^n$ denotes $\varprod_{i=1}^n \SX.$
\label{cartprodsym}%

% snippet(26)
\begin{defin}  \label{defin:nine5} {\rm
%
\index{sequence!definition!Definition \ref{defin:nine5}}%
\label{sequencesym}%
%
A {\it sequence} $(x_i)^{\infty}_{i=1} = (x_1,x_2,\ldots)$ is a tuple with a
countably infinite number of components. Now, let $i_1 < i_2 < \cdots.$ Then, $(x_{i_j} )^{\infty}_{j=1}$ is a {\it subsequence} of
$(x_i)^{\infty}_{i=1}.$
%
}\end{defin}


% snippet(27)
Let $\SX$ be a set, and let $X\isdef(x_i)_{i=1}^\infty$ be a sequence whose components are elements of $\SX;$ that is, $\{x_1,x_2,\ldots\}\subseteq\SX.$
%
For convenience, we write either $X\subseteq \SX$ or $X\subset \SX,$  where $X$ is viewed as a set and the multiplicity of the components of the sequence is ignored.
%
% snippet(28)
For sequences $X,Y\subset\BBF^n,$ define $X + Y\isdef (x_i + y_i)_{i=1}^\infty$ and $X\odot Y\isdef (x_i\odot y_i)_{i=1}^\infty,$ where ``$\odot$'' denotes component-wise multiplication.  In the case $n=1,$ we define $XY\isdef (x_i y_i)_{i=1}^\infty.$

% snippet(29)